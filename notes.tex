\documentclass[a4paper]{article}

\usepackage[margin=3.5cm]{geometry}

\usepackage{amsmath}
\usepackage{amsthm}
\usepackage{amssymb}
\usepackage[table,x11names]{xcolor}
\usepackage{logicproof}
\usepackage{url}
\usepackage{enumitem}
\usepackage{graphicx}
\usepackage{xypic}

\usepackage{titlesec}
\usepackage{longtable}

% For restatables
\usepackage{thmtools}
\usepackage{thm-restate}
\usepackage[hidelinks]{hyperref}
\usepackage{cleveref}

\theoremstyle{definition}
\newtheorem{definition}{Definition}
\newtheorem{lemma}{Lemma}
\newtheorem{example}{Example}
\newtheorem*{remark}{Remark}
\declaretheorem[name=Exercise,numberwithin=section]{exc}

% Writing
\newcommand{\ie}{{i.e.}}
\newcommand{\eg}{{e.g.}}

\renewcommand\thepart{\Alph{part}}
\titleformat{\part}{\Large\filcenter\scshape}{\textnormal{\textbf{Part \thepart}}}{1em}{\\}

\title{\vspace{-2.5em}CO519 - Theory of Computing - Logic}
\author{Dominic Orchard \\
  {\small{School of Computing, University of Kent}}}

\date{Last updated on \today}

\begin{document}
\maketitle

\noindent
These are the accompanying notes for the logic part of CO519. They
provide a counterpart to the lectures, but they do not replace them; the
lectures will provide content, detail, and discussion not given in
these notes.  However, the notes also provide additional exercises and
examples, and some additional detail beyond what is assessed on this
course-- this will be pointed out when it happens.

These notes also act as a counterpart to the \textbf{course textbook},
\emph{Logic in Computer Science} (by Huth and Ryan) which I
recommend. We will only cover material from the first two chapters due
to the short length of this part of the course. I recommend reading
these chapters, but the assessable material for this course is
covered in lectures and these notes (excluding the appendices); the
textbook contains extra detail which is not assessed, but
worth learning.

If you spot any errors or have suggested edits, the notes are written
in LaTeX and are available on GitHub at
\url{http://github.com/dorchard/co519-logic}. Please fork and submit a
pull request with any suggested changes.

The \textbf{last page of these notes} provides a section for you to
write down any topics you do not yet understand, or feel like you need to
work on more. By writing here during the course, you can offload the task of remembering
what you need to work on. You can then look back at this list when you
are studying, and eventually cross of items as you get to grips with
the material.

\part{Propositional logic \& its natural deduction
      proofs}

\documentclass{article}

\usepackage{amsmath}
\usepackage{amsthm}
\usepackage{amssymb}
\usepackage[table,x11names]{xcolor}
\usepackage{logicproof}
\usepackage{url}
\usepackage{fancyhdr}
\usepackage{enumitem}

% For restatables
\usepackage{thmtools}
\usepackage{thm-restate}
\usepackage[hidelinks]{hyperref}
\usepackage{cleveref}

\theoremstyle{definition}
\newtheorem{definition}{Definition}
\newtheorem{lemma}{Lemma}
\newtheorem{example}{Example}
\newtheorem*{remark}{Remark}
\declaretheorem[name=Exercise,numberwithin=section]{exc}

% Writing
\newcommand{\ie}{\emph{i.e.}}
\newcommand{\eg}{\emph{e.g.}}

\title{\vspace{-3em}CO519 - Theory of Computing - Logic \\
  {\large{Part A : Propositional logic and its natural deduction
      proofs}}}
\author{Dominic Orchard \\
  {\small{School of Computing, University of Kent}}}

\date{Last updated on \today}

\begin{document}
\maketitle

These are the accompanying notes for CO519 (the logic part). They provide
a counterpart to the lectures, but do not replace them; the lectures
will provide content, detail, and discussion not given in these notes.

These notes also provide a counterpart to the course textbook,
\emph{Logic in Computer Science} (by Huth and Ryan) which I
recommend. We will only cover material from the first two chapters due
to the short length of this part of the course. I recommend reading
these chapters, but the assessable material for this course is
covered in lectures and these notes (excluding the appendix); the
textbook contains extra detail which is not assessed, but
worth learning anyway.

If you spot any errors or have suggested edits, the notes are written
in LaTeX and are available on GitHub at
\url{http://github.com/dorchard/co519-logic}. Please fork and submit a
pull request with any suggested changes.

\section{Natural deduction for propositional logic}

Truth tables are a handy way to give meaning to logical operators and
formulas, in terms of truth or falsehood. But they are difficult to
use for reasoning about anything but small formulas since the number
of rows is $2^n$ where $n$ is the number of contingent
formulas. For example, the truth table for
$P \vee (Q \wedge R) \rightarrow (P \vee Q) \wedge (P \vee R)$ has
$2^3 = 8$ rows as there are three contingent formulas: $P$, $Q$, and
$R$ whose truth or falsehood determines the truth or falsehood of the
overall formula. Calculating even this modest truth table takes some
significant calculation to enumerate all possibilities.

Instead, logicians have constructed formal languages (calculi) for
building complex chains of reasoning (proofs) in a more
compact form. In this course, we use the \emph{natural deduction} calculus
due to 20$^{th}$ century logicians such as Gerhard Gentzen
(who formulated natural deduction in 1934) and Dag Pravitz who
promoted this style in the 1960s (and converted
various other results of Gentzen into the natural deduction style).

Natural deduction provides a system of \emph{inference rules} which
explain how to construct and deconstruct formulas to build
a logical argument. These rules represent the derivation of one formula
(the \emph{conclusion}) from several other formulas that are assumed or known to be true
(the \emph{premises}) via the notation:
%
\begin{equation*}
  \dfrac{\textit{premise}_1 \ldots \quad \ldots \textit{premise}_n}
        {\textit{conclusion}}
    \; {(\textit{label})}
\end{equation*}
%
Each logical operator (like conjunction $\wedge$ and
disjunction $\vee$), will have one or more rules for \emph{introducing}
that operator (deriving a conclusion using that operator) and one or
more rules for \emph{eliminating} that operator (deriving a conclusion from a formula
using that operator). These rules can be stacked together to form a
logical argument: \emph{a proof}, which looks like the following:

\begin{equation*}
\dfrac{\dfrac{P_1 \quad P_2}{P_3} \quad P_4}{P_5}
\end{equation*}
%
This isn't an actual concrete proof yet, it's just an example of how
a natural deduction proof looks when you stack together
its inference rules. Informally, we might have something like the
following:
%
\begin{equation*}
\dfrac{\dfrac{\textit{I forgot my coat} \quad \textit{It's
      raining}}{\textit{I get wet}} \quad \textit{My hairdryer broke}}
       {\textit{My hair remains wet}}
\end{equation*}
%
But this isn't always the most helpful format to derive the
proofs. Instead, we'll use a special ``box''-like notation called
Fitch-style which you can find in the recommend reading textbook
\emph{Logic in Computer Science} by Huth and Ryan.
We will still apply the rules of natural deduction, but the Fitch-style
gives us a nice way to layout the proof as we are deriving it.

We will step through the natural deduction rules for the core logical
operators: $\wedge$ (conjunction/and), $\vee$ (disjunction/or),
$\rightarrow$ (implication/if-then), $\neg$ (negation), as well as
truth $\top$ and falsehood/falsity, which is often written in
propositional logic as $\bot$ (pronounced
``bottom'').\footnote{\emph{Bottom} $\bot$ is often used in maths and
  computer science to represent undefined values or behaviour. In
  logic, if have arrived at falsity $\bot$ during a proof then we are
  in a situation where anything could be true as we've arrived at a
  logically inconsistent situation. This is sometimes quite useful for
  doing proofs-by-contradiction, as we will see in
  Section~\ref{sec:negation}.} Note that, in the literature, logical operators
  are sometimes alternatively called \emph{logical connectives}.

\subsection{Properties of formulas}

We will consider three properties that a logical formula may have:

\begin{itemize}[leftmargin=1.5em]
  \item \emph{valid}: a formula which is always true (also
  called a \emph{tautology}). In this part of the course, we will
  mostly prove the validity of formulas. For example, $P \wedge Q
  \rightarrow Q \wedge P$ is true no matter the
  truth/falsehood of $P$ and $Q$. This will be the main property we
  consider in \emph{Part A} (these notes).
%
  \item \emph{satisfiable}: a formula which is true for
  \emph{some} assignments of truth/falsehood to the atoms/variables it contains, \eg{},
  $a \wedge b$ is satisfiable, and the \emph{satisfying assignment} is that
  $a \mapsto \mathsf{T}$ and $b \mapsto \mathsf{T}$. A valid formula
  is trivially satisfiable (true for all assignments).
%
  \item \emph{unsatisfiable}: a formula which is always false (all
  rows in the truth table are false) regardless of assignments to the
  variables/atom, \eg{} $P \wedge \neg (P \vee Q)$. We can
  prove a formula is unsatisfiable by proving its negation is valid.
\end{itemize}

\subsubsection{Entailment and sequents}
\label{sec:entailment}

Suppose we have a set of formulas $P_1, \ldots, P_n$ from which we
want to prove $Q$ by applying the rules of a particular logic
(propositional logic here). The formulas $P_1, \ldots, P_n$ are the
premises and $Q$ is our goal conclusion. This is often written using the
following notation called a \emph{sequent}:
%
\begin{equation*}
P_1, \ldots, P_n \vdash Q
\end{equation*}
%
The turnstile symbol $\vdash$ is read as \emph{entails} and the
premises to the left are sometimes call the \emph{context} of assumed
formulas. This is a compact representation of a formula $Q$ along with
any assumptions used to deduce it.

We say this sequent is \emph{valid} if there is a proof that can be
constructed by deriving the conclusion from the premises. For example,
$P \wedge Q \vdash Q \wedge P$ is valid, meaning from $P \wedge Q$
there is a proof of $Q \wedge P$. If we wish to explain that
a sequent is invalid we can write $P \not\vdash Q$ meaning, it
is not valid that $P$ entails $Q$.

When there are no premises, we often drop the $\vdash$, \ie{},
writing something like ``\emph{$(P \wedge Q) \rightarrow P$ is valid}''
instead of ``\emph{$\vdash (P \wedge Q) \rightarrow P$ is
valid}''.

\subsection{Conjunction (``and'')}

Recall the truth table for conjunction:
%
\begin{center}
\begin{tabular}{cc|c}
  $P$ & $Q$ & $P \wedge Q$ \\ \hline
  F & F & F \\
  F & T & F \\
  T & F & F \\
\rowcolor{yellow}  T & T & T
\end{tabular}
\end{center}
%
From this, we see that in order to conclude the truth of $P \wedge Q$
we need the truth of $P$ and the truth of $Q$. This justifies the
following natural deduction rule for \emph{introducing} conjunction:
%
\begin{align*}
  \dfrac{P \qquad Q}
        {P \wedge Q} \; {\wedge_i}
\end{align*}
%
The label is subscripted with `i' for introduction.
That is, given two premises; $P$ is true and $Q$ is true, then
$P \wedge Q$ is true. There is just one introduction rule,
corresponding to the fact that there is only one ``true'' row
for $P \wedge Q$ in the truth table (highlighted in yellow above).

Note that the $P$ and $Q$ are place-holders here for \emph{any
  propositional formula} so we could instantiate the rule, \eg{},
to something like this if we needed it:
%
\begin{equation*}
\dfrac{(P \vee R \rightarrow Q) \quad S}
      {(P \vee R \rightarrow Q) \wedge S} \; {\wedge_i}
\end{equation*}
%
What about elimination?
Reading the highlighted line in the truth table from right-to-left shows how to
\emph{eliminate} a conjunction, \ie{}, what smaller formulas
can we conclude are true if we know that $P \wedge Q$ is true? We get
two rules:
%
\begin{align*}
  \dfrac{P \wedge Q}
        {P} \; {\wedge_{e1}}
  \qquad & \qquad
      \dfrac{P \wedge Q}
        {Q} \; {\wedge_{e2}}
\end{align*}
%
The labels have a subscript `e' for elimination; this convention will
continue.  (\emph{Aside}: If the syntax of the inference rules let us
have multiple conclusions then we could collapse the two eliminations
into one rule, but natural deduction instead has single
conclusions. There are different proof systems which allow
multiple conclusions (like the \emph{sequent calculus}) but we won't cover that here).

Let's write a proof with just these rules by stacking them together.

\begin{example}
\label{exm:assoc-conj}
  For any formula $P$, $Q$, $R$ then $P \wedge (Q \wedge
  R) \vdash (P \wedge Q) \wedge R$ is valid, \ie{} given $P \wedge (Q \wedge
  R)$ we can prove $(P \wedge Q) \wedge R$.
%
\newcommand{\conge}[1]{\wedge_{e#1}}
  \begin{align*}
    \dfrac{
    \dfrac{\dfrac{P \wedge (Q \wedge R)}
    {P}\conge{1}
    \dfrac{\dfrac{P \wedge (Q \wedge R)}
    {Q \wedge R} \conge{2}}{Q} \conge{1}}
    {P \wedge Q} {\wedge_i}
    \dfrac{\dfrac{P \wedge (Q \wedge R)}
    {Q \wedge R} \conge{2}}{R} \conge{2}}
    {(P \wedge Q) \wedge R} {\wedge_i}
  \end{align*}
The root of the tree is our goal $(P \wedge Q) \wedge R$, which is
built from the premises on the line above. These chains of reasoning
go up to the ``leaves'' of the tree, which is the assumed formula
$P \wedge (Q \wedge R)$. At each step (each line) we've applied
either conjunction introduction or one of the conjunction elimination
rules (as can be seen from the labels on the right).
\end{example}
\noindent
The following exercise is to prove the converse of the above property.
%
\begin{restatable}{exc}{assoc}
  \label{exm:assoc}
  Prove that $(P \wedge Q) \wedge R \vdash P \wedge (Q \wedge R)$ is
  valid by instantiating and stacking together inference rules.
\end{restatable}
%
This proof, and the one above, together imply that conjunction
$\wedge$ is \emph{associative}, \ie{},
$P \wedge (Q \wedge R) = (P \wedge Q) \wedge R$.

\paragraph{Fitch-style proof}
So far we have constructed proofs by stacking natural
deduction inference rules on top of each other. This leads us towards a
\emph{bottom-up} proof strategy starting with the goal and working up
towards the premises. In this course we are going to mostly use a
\emph{top-down} approach called ``Fitch-style''. This style begins
with assumptions, numbers each line of a proof, and uses indentation
and boxes to represent sub-proofs and the scope of their assumptions.

The above proof is rewritten in the following way in Fitch notation:
%
  \begin{logicproof}{2}
    P \wedge (Q \wedge R) & premise \\
    P                     & $\wedge_{e1}$ 1 \\
    Q \wedge R            & $\wedge_{e2}$ 1 \\
    Q                     & $\wedge_{e1}$ 3 \\
    R                     & $\wedge_{e2}$ 3 \\
    P \wedge Q            & $\wedge_i$ 2, 4 \\
    (P \wedge Q) \wedge R & $\wedge_i$ 6, 5 $\quad \Box$
  \end{logicproof}
%
  The proof follows in a number of linear steps. On the left we number
  each line of the argument. On the right, we explain which rule was
  applied to which formula, \eg{}, on the second line we have
  applied conjunction elimination $\wedge_{e1}$ to line $1$ to get the
  formula $P$. Or for example, on line 6, conjunction introduction is
  applied to lines 2 and 4 to get $P \wedge Q$.  We finish on line
  7 with our goal, which is marked with $\Box$ which a way of
  saying the proof is finished and we've reached our goal
  (the symbol means $Q.E.D$ which is an abbreviation of
  \emph{quod erat demonstrandum}, Latin for ``what was to be
  demonstrated'').

We haven't used any sub-proofs yet (which have a box drawn around
them); these appear in the next subsection on implication.

 \textbf{Order of numbers in labels} $\;$
Note that the order of the line numbers in labels tells us the order
of the premises to a natural deduction rule and so the order is
important. For example, line 6 above applies ($\wedge_i$ 2, 4) to
introduce $P \wedge Q$, but if it was actually ($\wedge_i$ 4, 2) we
would be introducing $Q \wedge P$ which is not our intended goal.

\begin{restatable}{exc}{andReproof}
Rewrite your proof to Exercise~\ref{exm:assoc} using
Fitch style.
\end{restatable}

\begin{remark}(\textbf{important})
Depending on what is being proved, a top-down approach (starting
from the premises) or bottom-up approach (starting
from the goal/conclusion) can be easier. In practice, if you are
stuck it can help to start \emph{at both ends} and work towards the
middle. You can do this by putting the goal near a bottom of a piece
of paper, giving enough space to meet in the middle.

It doesn't matter if things get messy-- the primary goal is to reach a
proof. You can rewrite it afterwards to be more clear; you should
do so in your class work and assessments.
\end{remark}

\subsection{Implication}

Recall the truth table for implication:
%
\begin{center}
\begin{tabular}{cc|c}
  $P$ & $Q$ & $P \rightarrow Q$ \\ \hline
  \rowcolor{yellow} F & F & T \\
  \rowcolor{yellow} F & T & T \\
  T & F & F \\
  \rowcolor{yellow} T & T & T
\end{tabular}
\end{center}
%
Implication $P \rightarrow Q$ is interesting because if
$\neg P$ (if $P$ is false) then $Q$ can be true or false, \ie{},
$Q$ can be anything if $P$ is false (the top two lines).

\begin{restatable}{exc}{implProperty}
Recall that $P \rightarrow Q = \neg P \vee Q$. Show this is true
by comparing the truth tables for each side of this equation.
\end{restatable}

As with conjunction, we'll consider the two style of
rule: elimination and introduction.
The elimination rule for implication in natural deduction is:
\begin{align*}
\dfrac{P \rightarrow Q \qquad P}{Q} {\rightarrow_e}
\end{align*}
%
This rule is also known as \emph{modus ponens}.\footnote{\emph{modus
ponens} is short for the Latin phrase \emph{modus ponendo ponens}
which means ``the way that affirms by affirming''.} It says that if
we know $P \rightarrow Q$ and we know $P$ then we know $Q$. You can
verify the soundness of this rule by looking at the truth
table: indeed $Q$ is true when both $P \rightarrow Q$ and $P$ are
true.

There are various other natural deduction rules one might construct by
looking at the truth table-- but this one can be used to derive the
others. The particular set of natural deduction rules we look at was
carefully honed by logicians to provide a kind of ``minimal'' calculus
for proofs.

Introduction of an implication $P \rightarrow Q$ follows from a
\emph{subproof} (which is drawn in a box) which starts with an
assumption of $P$ and ends with $Q$ as a conclusion after any number
of steps. The rule is written as follows:
%
\begin{align*}
\setlength{\arraycolsep}{0em}
\dfrac{\fbox{$\begin{array}{c} P \\ \vdots \\ Q\end{array}$}}
      {P \rightarrow Q} {\rightarrow_i}
\end{align*}
%
Thus subproofs in both tree- and Fitch-style proofs
are of the form:
%
\begin{equation*}
\setlength{\arraycolsep}{0em}
\fbox{$\begin{array}{c} \textit{assumption} \\ \vdots \\[0.5em]
         \textit{conclusion} \end{array}$}
\end{equation*}
%
\begin{remark} (\textbf{important})
%
  When we start a subproof box, the first formula is always an
  assumption. When the box is closed, the assumption does not go away
  but becomes the premise of the implication when applying the
  $\rightarrow_i$ rule.

  \emph{This is an important point}: when proving a theorem we
  have to be careful not to introduce additional assumptions which are
  not part of the theorem. For example, let's say we are proving a
  theorem expressed by a formula $Q$ but in doing so we assume $P$
  but $P$ is not one of $Q$'s assumptions.  Then
  instead we will have proved $P \rightarrow Q$ rather than $Q$.  This
  is something to keep in mind when writing complex
  proofs. The proof system of natural deduction allows us to keep
  track of our assumptions and their eventual inclusion in the final
  result.

  Aside: mechanised proof assistants (software systems in which we
  can write machine-checked proofs, such as \emph{Coq},
  \emph{Isabelle}, \emph{Agda}) have a similar basis to
  natural deduction and give us confidence and precision in writing proofs.
\end{remark}
%
There is an alternate presentation of natural deduction called
\emph{sequent-style natural deduction}, which is described in
Appendix~\ref{app:sequent}, where the inference rules are expressed in
terms of sequents $P_1, \ldots, P_n \vdash Q$. This won't be assessed
on the course, but is worth looking at if you want to read more widely
on logic. Another proof calculus (also due to Gentzen) is the
\emph{sequent calculus} which won't be described here, but there is
plenty of information online.

 \begin{example}
   The following simple formula about conjunction and implication is
   valid: $\vdash (P \wedge Q) \rightarrow (Q \wedge P)$. Here is its proof
   in Fitch-style:
%
\begin{logicproof}{2}
\begin{subproof}
P \wedge Q & assumption    \\
P          & $\wedge_{e1}$ 1 \\
Q          & $\wedge_{e2}$ 1 \\
Q \wedge P & $\wedge_i$ 3, 2
\end{subproof}
P \wedge Q \rightarrow Q \wedge P & $\rightarrow_i$ 1-4 $\qquad \Box$
\end{logicproof}
In the last line, we apply implication introduction and we
label it with the range of the lines of the subproof used (in this
case 1-4).
\end{example}

\begin{remark}
In Example~\ref{exm:assoc-conj} we proved
that given $P \wedge (Q \wedge R)$ then $(P \wedge Q) \wedge
R$. We can turn this into an implication
$P \wedge (Q \wedge R) \rightarrow (P \wedge Q) \wedge R$
simply by using implication introduction on the original
proof.
\end{remark}

\begin{example}
The following is valid:
\begin{align*}
(P \rightarrow (Q \rightarrow R))
\rightarrow
((P \rightarrow Q)
\rightarrow
(P \rightarrow R))
\end{align*}
%
Here is its proof:
%
\begin{logicproof}{5}
 \begin{subproof}
  P \rightarrow (Q \rightarrow R) & ass. \\ % 1
  \begin{subproof}
  P \rightarrow Q   & ass.  \\ % 2
  \begin{subproof}
   P               & ass. \\ % 3
   Q \rightarrow R & $\rightarrow_{e}$ 1, 3 \\ % 4
   Q               & $\rightarrow_{e}$ 2, 3 \\ % 5
   R               & $\rightarrow_{e}$ 4, 5 % 6
  \end{subproof}
   P \rightarrow R & $\rightarrow_{i}$ 3-6 % 7
  \end{subproof}
  (P \rightarrow Q) \rightarrow (P \rightarrow R) & $\rightarrow_{i}$
  2-7 % 8
  \end{subproof}
(P \rightarrow (Q \rightarrow R))
\rightarrow ((P \rightarrow Q) \rightarrow (P \rightarrow R))
& $\rightarrow_i$ 1-8 $\quad \Box$
\end{logicproof}
%
Here we have an example of multiple nesting of subproofs.
(Tip: I proved this by working top-down and bottom-up at the same
time, which was made easier by typing the proof).
\end{example}

Occasionally it is useful to ``copy'' a
formula from earlier in a proof. For example, 
the following proof of $\vdash P \rightarrow P$ copies
a formula from one line of the proof to the other in order
to introduce a trivial implication:
%
\begin{logicproof}{2}
\begin{subproof}
P  & assumption \\
P  & copy 1
\end{subproof}
P \rightarrow P & $\rightarrow_i$ 1-2 $\quad \Box$
\end{logicproof}

\begin{restatable}{exc}{kcombinator}
Prove $P \rightarrow (Q \rightarrow P)$ is valid.
\end{restatable}

\subsubsection{Bi-implication  (``if and only if`'')}

Propositional logic often includes the bi-implication operator
$\leftrightarrow$ also read as ``\emph{if and only if}'' and sometimes
written as \emph{iff} (double f). A bi-implication $P \leftrightarrow Q$ is
equivalent to the conjunction of two implications, pointing in
opposite directions:
%
\begin{align*}
P \leftrightarrow Q \; = \; (P \rightarrow Q) \wedge (Q \rightarrow P)
\end{align*}
%
Therefore to construct or deconstruct a logical
bi-implication one can consider it as ``implemented'' by
conjunction and implication, reducing the number of
introduction/elimination rules that need to be
remembered. Nonetheless, thinking about what these would be
is a nice exercise.

\begin{restatable}{exc}{biimplRules} (optional)
Try to derive your own elimination and introduction rules for
bi-implication. There is usually one introduction and two eliminations.
\end{restatable}



\subsection{Disjunction (``or'')}

Recall the truth table for disjunction (which has the three rows
in which $P \vee Q$ is true):
%
\begin{center}
\begin{tabular}{cc|c}
  $P$ & $Q$ & $P \vee Q$ \\ \hline
  F & F & F \\
\rowcolor{yellow} F & T & T \\
\rowcolor{yellow}  T & F & T \\
\rowcolor{yellow}  T & T & T
\end{tabular}
\end{center}
%
The fact that we can conclude $P \vee Q$ from either $P$
or from $Q$ separately justifies the following two introduction
rules for disjunction in natural deduction:
%
\begin{align*}
  \dfrac{P}
  {P \vee Q} \; {\vee_{i1}}
  \qquad
    \dfrac{Q}
  {P \vee Q} \; {\vee_{i2}}
\end{align*}
%

\begin{example}
Prove $(P \wedge Q) \rightarrow (P \vee Q)$ is valid.

  \begin{logicproof}{2}
    \begin{subproof}
    P \wedge Q & assumption \\
    P          & $\wedge_{e1}$ 1 \\
    P \vee Q   & $\vee_{i1}$ 2
  \end{subproof}
  P \wedge Q \rightarrow P \vee Q & $\rightarrow_i$ 1-3 $\quad \Box$
  \end{logicproof}
  %
  This could have been written equivalently as a natural deduction tree:
  %
  \begin{align*}
   \dfrac{
    \fbox{$\begin{array}{c}
           \dfrac{P \wedge Q}
                 {\dfrac{P}{P \vee Q} {\vee_{i1}}} {\wedge_{e1}}
          \end{array}$}
    }{(P \wedge Q) \rightarrow (P \vee Q)} {\rightarrow_{i}}
  \end{align*}
  %
  This will be the last tree-based proof we see; from now on we'll
  just keep using the Fitch style.
\end{example}
%
What about disjunction elimination? Given the knowledge that $P \vee Q$ is true
then what can be conclude? Either $P$ is true, or $Q$ is true, or both
are true. Therefore, we don't know exactly what true formulas we can derive
from the truth of $P \vee Q$, we just know a selection of
possibilities.


The natural deduction way of eliminating disjunction is to have two
subproofs as premises which are contingent on the assumption of
either $P$ or $Q$:

\begin{align*}
\setlength{\arraycolsep}{0em}
\dfrac{\begin{array}{c} \\ \\[0.7em] P \vee Q\end{array} \quad
\fbox{$\begin{array}{c} P \\ \vdots \\ R\end{array}$}
\quad
\fbox{$\begin{array}{c} Q \\ \vdots \\ R\end{array}$}}{R}
\;
{\vee_e}
\end{align*}

\begin{example}
For any propositions $P, Q, R$ then $(P \wedge Q) \vee (P \wedge
R) \rightarrow P$ is valid.
%
  \begin{logicproof}{3}
    \begin{subproof}
      (P \wedge Q) \vee (P \wedge R) & assumption \\
      \begin{subproof}
        P \wedge Q  & assumption \\
        P           & $\wedge_{e1}$ 2
      \end{subproof}
      \begin{subproof}
        P \wedge R & assumption \\
        P          & $\wedge_{e1}$ 4
      \end{subproof}
        P          & $\vee_{e}$ 1, 2-3, 4-5
    \end{subproof}
    (P \wedge Q) \vee (P \wedge R) \rightarrow P & $\rightarrow_{i}$,
    1-6 $\quad \Box$
  \end{logicproof}
  You can see that the application of disjunction elimination $\vee_e$
  involves three things: a disjunctive formula (line 1) and two
  subproofs (lines 2-3 and lines 4-5) which respectively assume the two subformulas of
  disjunction and conclude with the same formula ($P$), which forms the
  conclusion of the subproof on line 6.
\end{example}

\begin{restatable}{exc}{orassoc}
  Prove $P \vee Q \rightarrow Q \vee P$ is valid.
\end{restatable}

\begin{remark}
  From looking at the truth table for disjunction, one might wonder
  why disjunction elimination does not look like:
  %%
\begin{align*}
\setlength{\arraycolsep}{0em}
\dfrac{\begin{array}{c} \\ \\[0.7em] P \vee Q\end{array} \quad
\fbox{$\begin{array}{c} P \\ \vdots \\ R\end{array}$}
\quad
\fbox{$\begin{array}{c} Q \\ \vdots \\ R\end{array}$}
  \quad
  \fbox{$\begin{array}{c} P \wedge Q \\ \vdots \\ R\end{array}$}}{R}
{\vee_e}
\end{align*}
  %%
  This would match more closely the idea of reading the truth-table
  ``backwards'' from right-to-left on true values of $P \vee Q$. The
  reason we don't have this is that natural deduction strives for
  minimality and the third subproof with assumption $P \wedge Q$ is
  redundant since if we have $P \wedge Q$ true we can apply
  either the subproof $\fbox{$P \ldots R$}$ or the subproof
  $\fbox{$Q \ldots R$}$ by first applying $\wedge_{e1}$ or
  $\wedge_{e2}$ to the assumption $P \wedge Q$ to get $P$ or $Q$ respectively.
%and the above rule can be derived from the disjunction
%  elimination shown here with just two subproofs:
%  \begin{align*}
%\setlength{\arraycolsep}{0em}
%\dfrac{\begin{array}{c} \\ \\[0.7em] P \vee Q\end{array} \quad
%\Delta = \fbox{$\begin{array}{c} P \\ \vdots \\ R\end{array}$}
%\quad
%\fbox{$\begin{array}{c} Q \\ \vdots \\ R\end{array}$}
%  \quad
%  \fbox{$\begin{array}{c} \dfrac{\dfrac{P \wedge Q}{P}
%           {\wedge_{e1}}}{\Delta} \\ \vdots \\ R \end{array}$}}{R}
%{\vee_e}
%\end{align*}
%  That is, if we name the subproof of $\fbox{$P \ldots R$}$ as $\Delta$ and
%  then we can reuse this along with conjunction elimination to get the
%  proof for $\fbox{$P \wedge Q \ldots R$}$. In fact, there is another way to
%  derive this, where we reuse the subproof of $\fbox{$Q \ldots R$}$ and use
%  $\wedge_{e2}$ to build a proof of $\fbox{$P \wedge Q \ldots R$}$.
  \end{remark}

\subsection{Negation}
\label{sec:negation}

Negation introduction and elimination are given by:
%
\begin{align*}
\setlength{\arraycolsep}{0em}
\dfrac{
\fbox{$\begin{array}{c} P \\ \vdots \\ \bot\end{array}$}}
      {\neg P} \; {\neg_i}
\qquad\quad
\dfrac{P \qquad \neg P}{\bot} {\neg_e}
\end{align*}
%
Introduction says that given a proof that assumes $P$ but ends in
falsehood $\bot$ then we know $\neg P$, this is similar to the notion of
\emph{proof by contradiction}, which is derived from this (see below).

Elimination states that given a proof of $P$ and a simultaneous proof of
$\neg P$ then we conclude falsehood $\bot$, \ie{}, we have a
logical inconsistency on our hands and so end up proving false:
$P$ and $\neg P$ cannot both be true at the same time.

\begin{example}
For all $P, Q$ then $P \rightarrow Q \vdash \neg Q \rightarrow
\neg P$.
%
\begin{logicproof}{3}
    P \rightarrow Q  & assumption \\
    \begin{subproof}
      \neg Q        & assumption \\
      \begin{subproof}
        P           & assumption \\
        Q           & $\rightarrow_{e}$ 1, 3 \\
        \bot        & $\neg_{e}$ 2, 4
      \end{subproof}
      \neg P       & $\neg_i$ 3-5
     \end{subproof}
    \neg Q \rightarrow \neg P & $\rightarrow_i$ 2-6 $\qquad \Box$
\end{logicproof}
\end{example}

\begin{remark}
  This example is often given as a derived inference rule called
\emph{modus tollens}\footnote{\emph{modus
tollens} is short for the Latin phrase \emph{modus tollendo tollens}
which means ``the way that denies by denying''.}
that is similar to modus ponens (implication elimination):
%
\begin{align*}
\dfrac{P \rightarrow Q \quad \neg Q}
      {\neg P} \; {\emph{mt}}
\end{align*}
%
If an inference rule can be derived from others we say it is
\emph{admissible}. The system of rules we take as the basis for
natural deduction reasoning contains no admissible rules.
\end{remark}

\begin{remark}
  If we want to prove a formula $P$ is unsatisfiable then we can
  instead prove that $\neg P$ is valid (always true), hence proving
  that $P$ is unsatisfiable (always false).
\end{remark}

\begin{restatable}{exc}{disproveEx}
Prove that $P \wedge \neg (P \vee Q)$ is unsatisfiable.
\end{restatable}

\begin{remark}
  Some formulae are not valid, \eg{}, $P \rightarrow \neg P$, which
  can be seen from drawing its true table. However, this formula is
  \emph{satisfiable}, if $P$ is false then $P \rightarrow \neg P$ is
  true. Natural deduction does not help us to prove
  satisfiability. Part B will look at algorithmic approaches to
  deciding satisfiability.
\end{remark}

\subsubsection{Double negation}

A special rule holds called \emph{double-negation elimination} which
allows us to remove double negations on a proposition:
%
\begin{align*}
\dfrac{\neg \neg P}{P} \;\; {\neg\neg_e}
\end{align*}
%
\begin{example}
The principle of \emph{proof by contradiction} is represented by
following the derived inference rule:
%
\begin{equation*}
\setlength{\arraycolsep}{0em}
\dfrac{
\fbox{$\begin{array}{c} \neg P \\ \vdots \\ \bot\end{array}$}}
      {P} \; {\textsc{PBC}}
\end{equation*}
%
That is, if we assume $\neg P$ and conclude $\bot$, then we have $P$.
To show how to derive this, let the subproof in the above rule be
called $\Delta$, then we construct the following proof:
%
\begin{logicproof}{2}
\begin{subproof}
\neg P & ass. \\
\Delta & $\vdots$ \\
\bot &
\end{subproof}
\neg \neg P & $\neg_i$ 1-3 \\
P           & $\neg\neg_e$ 4
\end{logicproof}
Of course, $\Delta$ might be much longer than 3 lines, but we use
the numbering in the above proof for clarity.
\end{example}

\subsection{Truth and falsity}

If we have $\bot$ (false), then we can derive any formula:
%
\begin{align*}
\dfrac{\bot}{P} \; {\bot_e}
\end{align*}
%
There is no $\bot$ introduction as such, though $\neg_{e}$
provides a kind of $\bot$ introduction (from conflicting formula).
Dually, we can always introduce truth from no premises, but there
is no elimination:
%
\begin{align*}
\dfrac{\qquad}{\top} \; {\top_i}
\end{align*}
%

\subsection{A further derived rule: Law of Excluded Middle}

An interesting rule that we can derive in the propositional logic
is called the \emph{Law of Excluded Middle} or LEM for short. It says
that for any formula $P$ we have the following valid rule:
%
\begin{equation*}
\dfrac{\qquad\qquad}{P \vee \neg P} \textsc{lem}
\end{equation*}
%
\ie{}, whatever $P$ is, then either $P$ is true or $\neg P$ is true. Here
is its derivation:
%
\begin{logicproof}{2}
  \begin{subproof}
    \neg (P \vee \neg P) & ass. \\
    \begin{subproof}
      P  & ass. \\
      P \vee \neg P & $\vee_{i1}$ 2 \\
      \bot          & $\neg_e$ 3, 1
    \end{subproof}
    \neg P          & $\neg_i$ 2-4 \\
    P \vee \neg P   & $\vee_{i2}$ 5 \\
    \bot            & $\neg_e$ 6, 1
  \end{subproof}
\neg \neg (P \vee \neg P) & $\neg_i$ 1-7 \\
P \vee \neg P & $\neg\neg_e$ 8 $\;\;\Box$
\end{logicproof}
%
This rule can be useful in particular proofs.

\begin{restatable}{exc}{lemp}
Using LEM, prove that $\,P \rightarrow Q \vdash \neg P \vee Q\,$ is valid.
\end{restatable}

\paragraph{Aside: constructive vs non-constructive logic}

In this course, we study a particular kind of propositional logic
called \emph{classical} or \emph{non-constructive}
logic. Another variant is known as \emph{intuitionistic} or
\emph{constructive} logic which has a slightly different
set of inference rules: $\neg\neg_e$ is not included. By removing
double-negation elimination we can no longer derive
proof-by-contradiction or LEM.

The central principle of constructive logic is to reason about
\emph{proof} rather than \emph{truth} (as in classical logic). In
constructive logic, a
formula $P$ represents the proof of formula $P$: a mathematical object
witnessing the truth of $P$ which we can separately analyse. The
inference rules of natural deduction are now about preserving proof
rather than truth, \eg{}, conjunction elimination says given a proof
of $P \wedge Q$ then we can prove $P$.

In constructive logic, $\neg\neg_e$ is rejected since it would mean we
can get a proof of $P$ from a proof of the negation of the negation of
$P$, but this proof is not a proof of $P$. This is particularly
troublesome when using $\neg\neg_e$ to prove LEM. If LEM was allowed
in constructive logic, then for any formula $P$ we can construct
either a proof of $P$ or a proof of $\neg P$. But what is that proof
and where has it come from? Out of thin air! (LEM has no
premises). The essence of constructive logic is to disallow such
things so that we always know we have a concrete proof for our
formulas, constructed from proofs of its subformulas or premises. Section 1.2.5
of the Huth and Ryan course textbook gives some more detail and shows
an example mathematical proof about rational numbers in classical logic
which cannot be proved constructively.

Constructive logics are useful because they correspond to type systems
in functional programming: a result known as the \emph{Curry-Howard
  correspondence}. Unfortunately, we will not have time to study that here.

\newpage

\section{Collected rules of natural deduction}

\vspace{2em}

\setlength{\tabcolsep}{1.54em}
\renewcommand{\arraystretch}{1}
\begin{tabular}{r||c|c}
 & \textit{Introduction} & \textit{Elimination} \\[0.5em] \hline \hline
  $\wedge$
& \rule{0cm}{0.75cm} $\dfrac{P \qquad Q}
         {P \wedge Q} \; {\wedge_i}$
& $\dfrac{P \wedge Q}
        {P} \; {\wedge_{e1}}
  \qquad
      \dfrac{P \wedge Q}
  {Q} \; {\wedge_{e2}}$ \\[1.25em] \hline
  %%%%%%%%%%%
  $\vee$
& $\begin{array}{c}\dfrac{P}
  {P \vee Q} \; {\vee_{i1}}
  \qquad
    \dfrac{Q}
  {P \vee Q} \; {\vee_{i2}}\\[3.25em]\end{array}$
& \rule{0cm}{2.15cm} $\setlength{\arraycolsep}{0em}
\dfrac{\begin{array}{c} \\ \\[0.7em] P \vee Q\end{array} \quad
\fbox{$\begin{array}{c} P \\ \vdots \\ R\end{array}$}
\quad
\fbox{$\begin{array}{c} Q \\ \vdots \\ R\end{array}$}}{R}
\;
{\vee_e}$ \\[1em] \hline
  %%%%%%%%%%%
  $\rightarrow$
 & \rule{0cm}{2.15cm} $\setlength{\arraycolsep}{0em}\dfrac{\fbox{$\begin{array}{c} P \\ \vdots \\ Q\end{array}$}}
      {P \rightarrow Q} {\rightarrow_i}$
 & $\begin{array}{c}\dfrac{P \rightarrow Q \qquad P}{Q}
      {\rightarrow_e} \\[3em]\end{array}$
  \\[1em] \hline
  $\neg$
 & \rule{0cm}{2.15cm}
   $\setlength{\arraycolsep}{0em}\dfrac{\fbox{$\begin{array}{c} P \\ \vdots \\ \bot\end{array}$}}
  {\neg P} \; {\neg_i}$
 & $\begin{array}{c}\dfrac{P \qquad \neg P}{\bot} {\neg_e} \\[3em] \end{array}$ \\[0.5em] \hline
  $\top$
& \rule{0cm}{0.75cm}$\dfrac{\qquad}{\top} \; {\top_i}$ \\[1.25em] \hline
  $\bot$
 & & \rule{0cm}{0.75cm}$\dfrac{\bot}{P} \; {\bot_e}$ \\[1.25em] \hline
   $\neg\neg$
 & \rule{0cm}{0.75cm}{\textcolor{gray}{(derivable: $\dfrac{P}{\neg \neg P} \;\; {\neg\neg_i}$)}}  & \rule{0cm}{0.75cm}$\dfrac{\neg \neg P}{P} \;\; {\neg\neg_e}$
\end{tabular}

\vspace{3em}

\noindent
These are all the rules we have and need for propositional proofs. You
should aim to know all of the above rules by the end of the
course/exam.

We derived other useful inference rules from these rules, like modus
tollens, proof-by-contradiction, and law-of-excluded-middle. They
are useful to know but the table above gives the essential rules
for propositional proofs.

\newpage

\section{Exercises}

  This section collects together the exercises given so far. They may
  not all be covered in lectures, so they provide useful
  additional examples to practise on.

  \assoc*
  \andReproof*
  \implProperty*
  \kcombinator*
  \biimplRules*
  \orassoc*
  \disproveEx*
  \lemp*

\appendix

\section{Sequent-style natural deduction}
\label{app:sequent}

Recall from Section~\ref{sec:entailment} that 
a sequent is a compact representation of a formula $P$ along with
any assumptions used to deduce it, written in the form:
%
\begin{equation*}
P_1, \ldots, P_n \vdash P
\end{equation*}
%
The turnstile symbol $\vdash$ is read as \emph{entails} and the premises to the left are
called the \emph{context} of assumed formulas. The right-hand side is
the \emph{conclusion}.  For example, the following
judgment captures the idea of conjunction introduction:
%
\begin{equation*}
P, Q \vdash P \wedge Q
\end{equation*}
%
An alternate formulation of natural deduction gives the usual
introduction and elimination rules in sequent form, making explicit
the assumption context of the formula. This sequent-style of natural
deduction is not assessed in CO519, but is included here for
completeness and to help with any wider reading.

A key rule that was implicit in the previous formulation of natural
deduction is the use of an assumption as a formula. This is usually
called the \emph{axiom} rule:
%
\begin{align*}
\dfrac{\qquad}{\Gamma, P \vdash P} \; (\textit{axiom})
\end{align*}
%
This says that given some context with an assumption $P$ and any other
assumptions, represented by the Greek symbol $\Gamma$ (uppercase
gamma)\footnote{Gamma $\Gamma$ is the third letter of the Greek
  alphabet, corresponding to Latin $C$, hence $\Gamma$ for
  ``Context''. Logicians like Greek as it gives them lots more symbols
  to use to represent things tersely. These are conventions which take
  some getting used to.} then we can conclude $P$.  This is similar to
the idea of copying in Fitch-style proofs.

The order of assumptions on the left of $\vdash$ is not important (we
can freely move assumptions around).  The sequent style captures that
there may be other assumptions $\Gamma$ in scope.  A meta rule says
that we can add arbitrary redundant assumptions into our context
(called \emph{weakening}):
%
\begin{equation*}
  \dfrac{\Gamma \vdash A}
        {\Gamma, \Gamma' \vdash A} \; (\textit{weaken})
  \end{equation*}
  %
This is useful when we have two subproofs that we want to
make have the same set of assumptions (see below). The rest
of the rules are the introduction and elimination rules
for operators.

\paragraph{Conjunction}
%
\begin{align*}
\dfrac{\Gamma \vdash P \quad \Gamma \vdash Q}{\Gamma \vdash P \wedge
  Q} {\wedge_i}
\quad
\dfrac{\Gamma \vdash P \wedge Q}{\Gamma \vdash P} {\wedge_{e1}}
\quad
\dfrac{\Gamma \vdash P \wedge Q}{\Gamma \vdash Q} {\wedge_{e2}}
\end{align*}
%
These rules are very similar to the previously shown natural
deduction rules, but they now carry a context of assumptions
$\Gamma$. If the context doesn't match between
two premises, weakening (above) can be applied so that they match.

\paragraph{Disjunction}

\begin{align*}
\dfrac{\Gamma \vdash P}
      {\Gamma \vdash P \vee Q}  {\vee_{i1}}
\quad
\dfrac{\Gamma \vdash Q}
      {\Gamma \vdash P \vee Q} {\vee_{i2}}
\quad
\dfrac{\Gamma \vdash P \vee Q
  \quad \Gamma, P \vdash R
  \quad \Gamma, Q \vdash R}
      {\Gamma \vdash R} {\vee_{e}}
\end{align*}
%
The disjunction elimination rule is much less unruly than the previous
formulation, but has the same meaning. Note, we are now extending the
context of assumptions with $P$ and $Q$ in the last two premises.

\paragraph{Implication}

\begin{align*}
\dfrac{\Gamma \vdash A \rightarrow B \quad \Gamma \vdash A}
      {\Gamma \vdash B} {\rightarrow_e}
\quad
\dfrac{\Gamma, A \vdash B}{\Gamma \vdash A \rightarrow B}
 {\rightarrow_i}
\end{align*}
As an example, here is the proof of $P \wedge Q \rightarrow P \vee Q$
in this style:
%
\newcommand{\pAB}{\dfrac{}{P \wedge Q \vdash P \wedge Q} {\textit{axiom}}}
\begin{align*}
  \dfrac{
  \dfrac{\dfrac{\pAB}{P \wedge Q \vdash P} {\wedge_{e1}}}
  {P \wedge Q \vdash P \vee Q} {\vee_{i1}}}
  {\vdash (P \wedge Q) \rightarrow (P \vee Q)} {\rightarrow_{i}}
\end{align*}

\paragraph{Negation,  falsity, and truth}

\begin{align*}
  \dfrac{\Gamma, P \vdash \bot}{\Gamma \vdash \neg P} {\neg_i}
  \quad
  \dfrac{\Gamma \vdash P \; \Gamma \vdash \neg P}{\Gamma \vdash \bot} {\neg_e}
  \quad
  \dfrac{\Gamma \vdash \neg}{\Gamma \vdash P} {\bot_e}
  \quad
  \dfrac{}{\Gamma \vdash \top} {\top_i}
\end{align*}
%

\end{document}


\part{Modelling systems using logic}
\setcounter{section}{0}

\newcommand{\atomr}{\textsf{r}}
\newcommand{\atomrp}{\textsf{r'}}
\newcommand{\atomg}{\textsf{g}}
\newcommand{\atomgp}{\textsf{g'}}

In this part we are going to briefly cover using propositional
formula to model systems. This is a common approach in hardware design
where a complete or partial model of a circuit or processor is defined in logic,
against which specifications of particular properties are checked.
The starting point is to work out a good way to represent/model a
system as a logical formula. In this part of the course, we will
consider systems modelled as simple state machines, with states and
transitions between the states describing how a system can
change/evolve. We will then convert this model into a propositional
formula.


To verify a system based on its model we then need a specification of
either good behaviour that we want to make sure follows from our model
or of bad behaviour which we want to ensure does not follow from the
model.  We formulate such a specification as a propositional formula
\emph{spec}, and then prove that the following is valid:
%
\begin{equation*}
\emph{model} \rightarrow \emph{spec}
\end{equation*}
%
The use of implication means that if the model holds then
the specification must hold. Alternatively, and equivalently,
we could state this as judgment $\emph{model} \vdash \emph{spec}$,
\ie{}, the specified behaviour follows from the model.

\section{State-transition models as propositions}

\paragraph{States}
Our running example will be a very simple traffic light comprising
a red light and a green light, with two possible states:
%
\setlength{\tabcolsep}{0.4em}
\begin{center}
  \begin{tabular}{m{1.1cm} m{1.5cm} m{1.1cm}}
    State 1 & & State 2 \\
{\scalebox{0.2}{\includegraphics{images/red-on.pdf}}}
  &
  &
{\scalebox{0.2}{\includegraphics{images/green-on.pdf}}}
\end{tabular}
\end{center}
%
Either the red light is on (left) or the green light is on (right),
but never both at the same time, and there is always at least one
light on. We will use two atoms (propositional
variables that can be true or false) $\atomr{}$ and $\atomg$ to represent the state of
each light separately, where:
%
\begin{itemize}
  \item \atomr{} means the red light is on; $\neg\atomr{}$ therefore
  means the red light is off;
  \item \atomg{} means the green light is on; $\neg\atomg{}$ means the
  green light is off.
\end{itemize}
%
The two valid states of the system can then be modelled as two propositions:
%
\begin{align*}
  \text{State 1} & \qquad\qquad \text{State 2} \\
  \atomr{} \wedge \neg\atomg{} &
  \qquad\qquad
  \neg\atomr{} \wedge \atomg{}
\end{align*}
%
For $n$-propositional atoms there are $2^n$ possible states that can
be modelled. Thus in our model, there are four possible states, two of
which we want to treat as valid (the above two).

\begin{exc}
Define a propositions for each of the two invalid states in the
traffic light.
\end{exc}

\noindent
\textbf{State transitions} \hspace{0.5em} So far we have modelled the states as propositions, but we also want to
model the behaviour of the system in terms of the possible
transitions between states. We can represent this
with a simple diagram:
%
\setlength{\tabcolsep}{0.4em}
\begin{center}
  \begin{tabular}{m{1.1cm} m{1.5cm} m{1.1cm}}
       State 1 & & State 2 \\[-0.5em]
{\scalebox{0.2}{\includegraphics{images/red-on.pdf}}}
  &
    \xymatrix{
    \ar[rr]^{\textit{\normalsize{go}}} & & \\
     & & \ar[ll]^{\textit{\normalsize{stop}}}
    }
  &
{\scalebox{0.2}{\includegraphics{images/green-on.pdf}}}
\end{tabular}
\end{center}
%
\ie{}, when just the red light is on it is possible to
transition to a state with just the green light on, and back again.

To model state transitions we will introduce two additional atoms
that model the future state of the lights in the system:
%
%
\begin{itemize}
\item \atomrp{} for the red light being on in the \emph{next time step};
\item \atomgp{} for the green light being on in the \emph{next time step}.
\end{itemize}
%
We can now express the above two transitions as implications:
%
\begin{align*}
  (\textit{go}) : \qquad & \atomr{} \wedge \neg\atomg{} \; \rightarrow \;
                           \neg\atomr{}' \wedge \atomg{}' \\
  (\textit{stop}): \qquad & \neg\atomr{} \wedge \atomg{} \; \rightarrow \;
                            \atomr{}' \wedge \neg\atomg{}'
\end{align*}
%
Said another way, (\textit{go}) defines that if State 1 is true now
we can move to State 2 in the future (in the ``next'' time step of the
system), and (\textit{stop}) conversely defines that if State 2 is
true now we can move to State 1 in the future.

We can now describe the full transition behaviour of the system as
the conjunction of the above two formula:
%
\begin{equation*}
\textit{model} =
(\atomr{} \wedge \neg\atomg{} \; \rightarrow \; \neg\atomr{}' \wedge
\atomg{}')
\, \wedge \,
(\neg\atomr{} \wedge \atomg{} \; \rightarrow \; \atomr{}' \wedge \neg\atomg{}')
\end{equation*}
%
This provides our model of the system. You might be wondering why we
don't add more to this, \eg{}, ruling out invalid states. We will come
back to this point in Section~\ref{subsec:whenisamodelgood}.

% TODO: guiding principles, why not and this with the valid states?
% want to make a small formula and this follows
% need to come back to this?

\paragraph{A general approach}

The general approach to modelling a system in this way is to:
%
\begin{itemize}
  \item Decide what to represent about the state space of the system
  and introduce propositional atoms for these components.
  \item Model future states using ``next step'' atoms (usually written with
  an apostrophe, and called the ``primed'' atoms, \eg{}, \atomrp{} is
  read as ``r prime'').
 \item Write a propositional formula \emph{model} using these
  variables which takes the conjunction of state transitions expressed
  as implications.
\end{itemize}

\begin{exc}
Write down a model for a more realistic traffic light, \ie{}, that can
be described by the following states and transitions:
%

\begin{center}
\scalebox{0.28}{\includegraphics{images/full-light.pdf}}
\end{center}
\end{exc}

\section{Defining specifications as propositions}
\label{sec:spec}

Consider the following property which we might want to check
for our system:
%
\begin{quote}
\emph{If we are in a valid state and change state, then our new state
  is also valid.}
\end{quote}
%
We can abstract the notion of a valid state with the following
meta-level operation (you can think of this as a syntax function mapping from the
two propositions to a proposition):
%
\begin{equation*}
\textsf{valid-state}(r, g) = (r \wedge \neg g) \vee (\neg r \wedge g)
\end{equation*}
%
From this, we can then capture our specification as the proposition:
%
\begin{equation*}
  \textit{specification} = \textsf{valid-state}(\atomr, \atomg)
  \rightarrow \textsf{valid-state}(\atomrp, \atomgp)
\end{equation*}
%
\ie{}, a valid state now implies a valid state in the next time step.
To verify that our system (based on its model) satisfies this
property, we then need to prove that the following is true:
%
\begin{align*}
 & \textit{model} \rightarrow \textit{specification} \\
  \equiv \; &
((\atomr{} \wedge \neg\atomg{} \; \rightarrow \; \neg\atomr{}' \wedge
\atomg{}')
\wedge
(\neg\atomr{} \wedge \atomg{} \; \rightarrow \; \atomr{}' \wedge
  \neg\atomg{}'))
  \rightarrow
   (\textsf{valid-state}(\atomr, \atomg)
           \rightarrow \textsf{valid-state}(\atomrp, \atomgp)) \\
  \equiv \; &
((\atomr{} \wedge \neg\atomg{} \; \rightarrow \; \neg\atomr{}' \wedge
\atomg{}')
\wedge
(\neg\atomr{} \wedge \atomg{} \; \rightarrow \; \atomr{}' \wedge
  \neg\atomg{}'))
              \rightarrow
              (((\atomr \wedge \neg \atomg) \vee (\neg \atomr \wedge \atomg))
              \rightarrow ((\atomrp \wedge \neg \atomgp) \vee (\neg \atomrp \wedge \atomgp)))
\end{align*}
%
This is quite a big proposition so we might not want to prove it by
hand. Instead, in the next part of the course we are going to use an
algorithmic technique for proving that this holds (Part C,
specifically \textbf{Example 5} in the notes). We will later see that
this does indeed hold, and thus the system (as described by the model)
satisfies our specification.


\section{When is a model a good model?}
\label{subsec:whenisamodelgood}

\begin{quote}
\emph{All models are wrong but some are useful} (Box, 1978)
\end{quote}

\noindent
Indeed, a model is necessarily ``wrong'' in the sense that a model
abstracts some of the details of a real system; we are not
capturing every aspect of the system, such as: how are the transitions
triggered? or, what happens if a car crashes into the traffic light? Some
models, though eliding details, are useful in the sense that we
can detect when the system does not behave according to our
specification or we can verify that it always behaves according to our
specification. Just enough detail is needed in the model to capture what
we want to prove.

Our running example has the model:
\begin{equation*}
\mathit{model} =
(\atomr{} \wedge \neg\atomg{} \; \rightarrow \; \neg\atomr{}' \wedge
\atomg{}')
\wedge
(\neg\atomr{} \wedge \atomg{} \; \rightarrow \; \atomr{}' \wedge \neg\atomg{}')
\end{equation*}
%
We might add to this model the explicit exclusion of invalid
states. Let's define the condition of the invalid states via the meta-level function:
%
\begin{equation*}
\textsf{invalid-state}(r, g) = (\neg r \wedge \neg g) \vee (r \wedge g)
\end{equation*}
%
Then we could define an alternate model as:
%
\begin{equation*}
\mathit{model2} = \mathit{model} \wedge \neg \textsf{invalid-state}(\atomr, \atomg)
\end{equation*}
%
Thus, the new model adds to the old model that the current state is not
invalid.

In the case of the proving our specification from
Section~\ref{sec:spec} (that valid states transition to valid states)
we do not need this additional detail in the model. But we might want
to have this more restrictive model if we want to prove, for example,
that we can never be in an invalid state, regardless of what state we
started from. This can be captured by the new specification:
%
\begin{equation*}
\mathit{specification2} = \textsf{invalid-state}(\atomr, \atomg)
\vee \textsf{invalid-state}(\atomr', \atomg')
\end{equation*}
%
and proving that the following proposition is valid:
$$
\mathit{model2} \rightarrow
\neg \mathit{specification2}
$$
Note we are using a negative property
here: we show that the new model (\textit{model2}) implies that the bad behaviour is not possible.
For the old model (\textit{model}), a similar proposition capturing the same idea is valid:
$$
\mathit{model} \rightarrow \neg \mathit{specification2}
$$
However, the following proposition is also valid!!!
$$
\mathit{model} \rightarrow \mathit{specification2}
$$
Why? If the right-hand side of the implication is true (we have an
invalid state) the left-hand side is still true (the premise of each
transition implication is false, therefore the implications are
trivially true); the original model never excludes invalid states on
their own, only as the result of a transition from a valid
state. Thus, the original model was not a good model for checking the
$\mathit{specification2}$ property, but it was sufficient for
$\mathit{specification}$.

Creating a rich enough model is up to you, and requires some care and
thought about the domain and what properties are of interest.


\part{Satisfiability for propositional logic}
\setcounter{section}{0}

\renewcommand{\highlight}[1]{%
  \colorbox{yellow!50}{$\displaystyle#1$}}
\newcommand{\highlightG}[1]{%
  \colorbox{green!30}{$\displaystyle#1$}}
\newcommand{\highlightR}[1]{%
  \colorbox{red!20}{$\displaystyle#1$}}

\newcommand{\rel}[1]{\mathsf{#1}}

First-order logic (also called \emph{predicate logic}) extends
propositional logic with \emph{quantification}: existential quantification
$\exists$ (``\emph{there exists}'') and universal quantification
$\forall$ (``\emph{for all}''). A quantification $\forall x$
binds a variable $x$ which range over the elements of some underlying
\emph{universe} which is external to the logic, e.g., quantifying
over all people or objects.  First-order logic also
allows the use of relations, predicates (unary relations, also called
\emph{classifiers}) and functions, operating
over elements of the universe, which can
be defined externally and are
domain-specific for whatever purpose the logic is being used.

Consider the following sentence:
%
\begin{equation*}
  \textit{Not all birds can fly}
\end{equation*}
%
We can capture this in first-order logic using quantification and
unary predicates. Let our universe be ``animals''
over which we informally define two predicates:
%
\begin{align*}
  \rel{B}(x)\ & \stackrel{\text{def}}{=}\ \textit{$x$ is a bird} \\
  \rel{F}(x)\ & \stackrel{\text{def}}{=}\ \textit{$x$ can fly}
\end{align*}
%
As with propositional logic, we are studying the process and framework
of the logic rather than physical reality; it is up to us how we
assign the semantics of $\rel{B}$ and $\rel{F}$ above, but the
semantics of quantification and logical operators is fixed by the
definition of first-order logic.

We can then express the above sentence in first-order logic as:
%
\begin{equation}
  \neg (\forall x . \rel{B}(x) \rightarrow \rel{F}(x))
  \label{eq:nonflying1}
\end{equation}
%
We can read this exactly as \emph{it is not true that for all $x$, if
  $x$ is a bird then $x$ can fly}. Another way to write this is
that there are some birds which cannot fly:
%
\begin{equation}
  \exists x . \rel{B}(x) \wedge \neg \rel{F}(x)
    \label{eq:nonflying2}
\end{equation}
%
If we have a universe and semantics for $\rel{B}$ and $\rel{F}$ that includes, for
example, penguins, then both \eqref{eq:nonflying1} and
\eqref{eq:nonflying2} will be true.
We can prove that \eqref{eq:nonflying1} and
\eqref{eq:nonflying2} are equivalent in first-order logic via two
proofs, one for:
$$\neg (\forall x . \rel{B}(x) \rightarrow \rel{F}(x)) \; \vdash \;
\exists x . \rel{B}(x) \wedge \neg \rel{F}(x)$$
and one for:
$$
\exists x . \rel{B}(x) \wedge \neg \rel{F}(x) \; \vdash \;
\neg (\forall x . \rel{B}(x) \rightarrow \rel{F}(x))
$$
We will do this later once we have explained more about the
meta theory of the logic.

\section{Key concepts (meta theory) of first-order-logic}

\subsection{Names and binding}

In propositional logic, variables range over propositions, \ie{},
their ``type'' is a proposition. For example, $x \wedge y$ has two
propositional variables $x$ and $y$ which could be replaced with true
or false, or with any other formula. In predicate logic, universal and
existential quantifiers provide \emph{variable bindings} which
introduce variables ranging over objects in some fixed universe rather than
over propositions. For example, the formula $\forall x . P$ binds
a variable $x$ in the \emph{scope} of $P$. That is, $x$
is available within $P$, but not outside of it. A variable which does
not have a binding in scope is called \emph{free}.

For example, the formula below has free variables $x$ and $y$
and bound variables $u$ and $v$:
%
\begin{equation*}
\rel{P}(x) \, \vee  \, \forall u . \, (\, \rel{Q}(y) \, \wedge \,
\rel{R}(u) \, \rightarrow \exists v . \, (\, \rel{P}(v) \, \wedge \, \rel{Q}(x) \,))
\end{equation*}
%
The following repeats the formula and
highlights the binders in yellow, the bound variables
in green, and the free variables in red:
%
\begin{equation*}
\highlightR{\rel{P}(x)} \vee \highlight{\forall u} . (\highlightR{\rel{Q}(y)} \wedge \highlightG{\rel{R}(u)}
\rightarrow \highlight{\exists v} .
(\highlightG{\rel{P}(v)} \wedge \highlightR{\rel{Q}(x)}))
\end{equation*}
%
%
In the following formula, there are two syntactic occurrences of a
variable called $x$, but semantically these are different variables:
%
\begin{equation*}
\rel{Q}(x) \wedge (\forall x . \rel{P}(x))
\end{equation*}
%
The $x$ on the left (used with a predicate $\mathsf{Q}$) is free,
whilst the $x$ used with the predicate $\rel{P}$ is bound by the
universal quantifier. Thus, these are semantically two different
variables.

\paragraph{Alpha renaming}
The above formula is
semantically equivalent to the following formula obtained by
consistently renaming bound variables:
%
\begin{equation*}
\rel{Q}(x) \wedge (\forall y . \rel{P}(y))
\end{equation*}
%
Renaming variables is a meta-level operation we can apply to any
formula: we can rename a bound variable as long as we do not rename it
to clash with any other free or bound variable names, and as long as
we rename the variable consistently. This principle is more generally
known as $\alpha$-renaming (alpha renaming) and equality up-to
renaming (equality that accounts for renaming) is known as
$\alpha$-equality. For example, writing $\alpha$-equality as
$=_{\alpha}$ the following equality and inequality hold:
%
\begin{equation*}
\exists x . \rel{P}(x) \rightarrow \rel{P}(y)
  \;\;\; =_{\alpha} \;\;\;
\exists z . \rel{P}(z) \rightarrow \rel{P}(y)
 \;\;\; \neq_{\alpha} \;\;\; \exists y . \rel{P}(y) \rightarrow \rel{P}(y)
\end{equation*}
%
The middle formula can be obtained from the left by renaming $x$ to a
fresh variable $z$. However, if we rename $x$ to $y$ (on the right)
we conflate the bound variable with the previously free variable to the
right of the implication; we accidentally capture the
free occurence of $y$ via the binding. The right-hand formula has
a different meaning to the other two.

\subsection{Substitution}

Recall in Part C, we used the function $\textit{replace}$ in the DPLL
algorithm where $\textit{replace}(x, Q, P)$ rewrites formula $P$ such
that any occurrences of variable $x$ are replaced with formula
$Q$. This is more generally called \emph{substitution}.

From now on we will use a more compact syntax for substitution
written $$P[t/x]$$ which means: \emph{replace variable $x$
with the term $t$ in formula $P$} (akin to $\textit{replace}(x, t,
P)$). This term could be another variable or a
concrete element of our universe.

Note that in predicate logic we have to be careful
about free and bound variables. Thus, $P[t/x]$ means replace any
\emph{free} occurrences of $x$ in $P$ with object $t$. (One way to remember
this notation is to observe that the letters used in the general
form above are in alphabetical order: $P$ then $t$ then $x$ to
give $P[t/x]$ for replacing $x$ with $t$ in $P$).  This is a common
notation also used in the course textbook.

We must be careful to replace only the free
occurrences of variables, that is, those variables which are not in
the scope of a variable binding of the same name. For example, in the
following we have a free $x$ and a bound $x$, so substitution only
affects the free $x$:
%
\begin{equation*}
(\rel{P}(x) \wedge \forall x . \rel{P}(x))[t/x]
= \rel{P}(t) \wedge \forall x . \rel{P}(x)
\end{equation*}
%
In general, it is best practice to give each bound variable a different name
to all other free and bound names in a formula in order to avoid
confusion.

\subsection{The meaning of quantification}
\label{subsec:quantifier-meaning}

We can define the meaning of universal and existential quantification
in terms of the propositional logic connectives.

\paragraph{Universal quantification}

Universal quantification essentially generalises conjunction.  That
is, if the objects in the universe over which we are quantifying are
$a_0, a_1, \ldots, a_n \in \mathcal{U}$ then universal quantification
of $x$ over a formula $P$ is equivalent to taking the repeated
conjunction of $P$, substituting each object for $x$, \ie{}
%
\begin{equation}
\forall x . P = P[a_0/x] \wedge P[a_1/x] \wedge
\ldots \wedge P[a_{n}/x]
\label{eq:forall-meaning}
\end{equation}
%
Thus, $\forall x . P$ means that we want $P$ to be true for
all the objects in the universe being used.
Note that there may be an infinite number of such objects.

\paragraph{Existential quantification}

Whilst universal quantification generalises conjunction,
existential quantification generalises disjunction.
If existential quantification binds a variable ranging
over objects $a_0, a_1, \ldots, a_n \in \mathcal{U}$ then:
%
\begin{equation}
\exists x . P = P[a_0/x] \vee P[a_1/x] \vee
\ldots \vee P[a_{n}/x]
\label{eq:exists-meaning}
\end{equation}
%
Thus, existential quantification is equivalent to the repeated
disjunction of the formula $P$ with each object in the universe
replacing $x$.



\subsection{Defining models/universes}

First-order logic can be instantiated for particular concrete tasks by
defining a universe $\mathcal{U}$ (a set of elements) and any
relations, functions, and predicates over this universe.

For example, we could define
$\mathcal{U} = \{\mathsf{cat}, \mathsf{dog}, \mathsf{ant},
\mathsf{chair}\}$ meaning that when we write quantified formulas like
$\forall x . P$ (for some formula $P$) then $x$ refers to any of the things in
$\mathcal{U}$ (i.e., $x \in \mathcal{U}$). We could then concretely
define some functions and predicates. For example, let's define a function
$\mathsf{legs}$ which maps from $\mathcal{U}$ to $\mathbb{N}$ (i.e.,
$\mathsf{legs} : \mathcal{U} \rightarrow \mathbb{N}$) as:
%
\begin{align*}
  \mathsf{legs}(\mathsf{cat}) = 4 \qquad \mathsf{legs}(\mathsf{dog}) = 4 \qquad
  \mathsf{legs}(\mathsf{ant}) = 6 \qquad \mathsf{legs}(\mathsf{chair}) = 4
\end{align*}
%
We can define predicates by listing all their true instances. For
example, $\mathsf{mammal}$ classifies some members
of $\mathcal{U}$, defined via a proposition that lists all the true
instances as a conjunction:
%
\begin{align*}
  \mathsf{mammal}(\mathsf{cat})\ \wedge\
  \mathsf{mammal}(\mathsf{dog})
\end{align*}
%
Let's consider some true formulas in this instantiation of first-order
logic:
%
\begin{align*}
  \begin{array}{ll}
  \vdash \forall x . \mathsf{mammal}(x) \rightarrow (\mathsf{legs}(x) = 4) &
\quad (\textit{every mammal has four legs}) \\
  \vdash \exists x . \mathsf{legs}(x) < 4 &
\quad (\textit{there is something with less than four legs})
  \end{array}
\end{align*}
%
We have also employed two relations over $\mathbb{N}$ here:\footnote{Strictly speaking, we are
  therefore using first-order logic where the universe contains
  our set $\{\mathsf{cat}, \mathsf{dog}, \mathsf{ant},
  \mathsf{chair}\}$ and $\mathbb{N}$, i.e.,
  $\mathcal{U} = \{\mathsf{cat}, \mathsf{dog}, \mathsf{ant},
\mathsf{chair}\} \cup \mathbb{N}$, and our function $\mathsf{legs}$ is
partial, defined only for a part of the universe.}
equality $=$ and less-than $<$.

A false proposition in this instantiation is:
%
\begin{align*}
  \not\vdash \forall u . (\mathsf{legs}(u) = 4) \rightarrow \mathsf{mammal}(u)
& \quad (\textit{everything with four legs is a mammal})
\end{align*}
%
This is false because $u$ could be $\mathsf{chair}$ (making the
premise of the implication true) but
$\mathsf{mammal}(\mathsf{chair})$ is false.

\section{Equational reasoning}
\label{sec:fo-eqn-reasoning}

As in propositional logic, there are equations (logical
equivalences) between particular first-order formulas. These can be used to
rearrange and simplify formulas. This sections shows
these equations, some of which are proved
in the next section via natural deduction.

Two key equations show that universal and existential
quantification are \emph{dual}:
%
\begin{align}
  \label{eq:quantifier-dual-first}
  \forall x . \neg P \equiv \neg \exists x . P
  \qquad
   \neg \forall x . P \equiv \exists x . \neg P
\end{align}
%
The order of repeated quantifications is irrelevant as shown by
the following equalities:
%
\begin{align}
  \forall x . \forall y. P \equiv \forall y . \forall x . P
  \qquad
  \exists x . \exists y . P \equiv \exists y . \exists x . P
\end{align}
%
Note however that these equalities are only for quantifications that
are of the same kind; $\forall x . \exists y . P$ is not equivalent to
$\exists y . \forall x . P$.
The rest of the equations capture interaction between quantification
and the other propositional connectives:
%
\begin{align}
  (\exists x . P) \vee (\exists x . Q) \equiv \exists x . (P \vee Q) \\
  (\forall x . P) \wedge (\forall x . Q) \equiv \forall x . (P \wedge
                                                                       Q) \\
  P \wedge (\exists x . Q) \equiv \exists x . (P \wedge Q) & \;\; \textit{when $x$ is  not free in $P$} \\
  P \vee (\forall x . Q) \equiv \forall x . (P \vee Q) & \;\;
 \textit{when $x$ is  not free in $P$}
\end{align}
\vspace{-2em}
\begin{example}
  We can now go back to the example from the introduction: that
  \emph{not all birds can fly}. We formulated this sentence as both
  $\neg (\forall x . \rel{B}(x) \rightarrow \rel{F}(x))$ and
  $\exists x . \rel{B}(x) \wedge \neg \rel{F}(x)$.
  We can show these two statements are equivalent by algebraic
  reasoning:
%
\begin{align*}
\begin{array}{rll}
    & \neg (\forall x . \rel{B}(x) \rightarrow \rel{F}(x)) & \{\textit{by \eqref{eq:quantifier-dual-first}}\} \\[0.4em]
  \equiv \;\; &  \exists x . \, \neg (\rel{B}(x) \rightarrow
                \rel{F}(x)) & \{\textit{$P \rightarrow Q \equiv \neg P
                              \vee Q$}\} \\[0.4em]
  \equiv \;\; & \exists x . \,\neg (\neg \rel{B}(x) \vee \rel{F}(x))
& \{\textit{De Morgan's}\} \\[0.4em]
  \equiv \;\; & \exists x . \,\neg \neg \rel{B}(x) \wedge \neg
                \rel{F}(x)
& \{\textit{Double negation elim.}\} \\[0.4em]
  \equiv \;\; & \exists x . \,\rel{B}(x) \wedge \neg \rel{F}(x) & \Box
\end{array}
\end{align*}
  Note that the actual universe and the definition of $\rel{B}$ and
  $\rel{F}$ is irrelevant to this proof; we did not rely on their
  definition but just the general properties of first-order logic.
\end{example}

\vspace{-1em}

\begin{restatable}{exc}{eqnProofF}
  Prove via equational reasoning that:
  %
  \vspace{-0.25em}
\begin{equation*}
  \forall x . \mathsf{mammal}(x) \rightarrow (\mathsf{legs}(x) = 4) \;
  \equiv \; \neg \exists x . \mathsf{mammal}(x) \wedge
  \mathsf{legs}(x)
  \neq 4
\end{equation*}
\end{restatable}
\vspace{-1em}


\part{First-order logic and its natural deduction
  proofs}
\setcounter{section}{0}

\documentclass{article}

\usepackage{amsmath}
\usepackage{amsthm}
\usepackage{amssymb}
\usepackage[table,x11names]{xcolor}
\usepackage{logicproof}
\usepackage{url}
\usepackage{fancyhdr}
\usepackage{enumitem}

% For restatables
\usepackage{thmtools}
\usepackage{thm-restate}
\usepackage[hidelinks]{hyperref}
\usepackage{cleveref}

\theoremstyle{definition}
\newtheorem{definition}{Definition}
\newtheorem{lemma}{Lemma}
\newtheorem{example}{Example}
\newtheorem*{remark}{Remark}
\declaretheorem[name=Exercise,numberwithin=section]{exc}

\newcommand{\highlight}[1]{%
  \colorbox{yellow!50}{$\displaystyle#1$}}
\newcommand{\highlightG}[1]{%
  \colorbox{green!30}{$\displaystyle#1$}}
\newcommand{\highlightR}[1]{%
  \colorbox{red!20}{$\displaystyle#1$}}

\newcommand{\rel}[1]{\mathsf{#1}}

% Writing
\newcommand{\ie}{\emph{i.e.}}
\newcommand{\eg}{\emph{e.g.}}

\title{\vspace{-3em}CO519 - Theory of Computing - Logic \\
  {\large{Part D : First-order logic and its natural deduction
      proofs}}}
\author{Dominic Orchard \\
  {\small{School of Computing, University of Kent}}}

\date{Last updated on \today}

\begin{document}
\maketitle

\noindent
If you spot any errors or have suggested edits, the notes are written
in LaTeX and are available on GitHub at
\url{http://github.com/dorchard/co519-logic}. Please fork and submit a
pull request with any suggested changes.

\section{Natural deduction for first-order logic}

First-order logic (also known as \emph{predicate logic}) extends
propositional logic with quantification: existential quantification
$\exists$ (``\emph{there exists....}'') and universal quantification
$\forall$ (``\emph{for all...}'').  First-order logic also adds
relations, predicates (unary relations or \emph{classifiers}), and
functions. The relations, predicates, and functions are
domain-specific for whatever purpose the logic is being used and may
be defined externally. Usually some underlying ``universe'' is fixed
over which quantified variables range and on which predicates,
relations, and functions are defined.

Consider the following sentence:
%
\begin{equation*}
  \textit{Not all birds can fly}
\end{equation*}
%
We can capture this in first-order logic using quantification and
unary predicates. Let's define abstractly two predicates:
%
\begin{align*}
  \rel{B}(x) & : \textit{$x$ is a bird} \\
  \rel{F}(x) & : \textit{$x$ can fly}
\end{align*}
%
Our universe here might be ``animals'' or just general objects. As
with propositional logic, we are studying the process and framework of
logic rather than any connections of certain logical statements to
physical reality; it is up to us how we assign the semantics of
$\rel{B}$ and $\rel{F}$ above, but the semantics of quantification and
logical operators is fixed by the definition of first-order logic.

We can then express the above sentence in predicate logic as:
%
\begin{equation}
  \neg (\forall x . \rel{B}(x) \rightarrow \rel{F}(x))
  \label{eq:nonflying1}
\end{equation}
%
We can read this exactly as \emph{it is not true that for all $x$, if
  $x$ is a bird then $x$ can fly}. Another way to write this is 
that there are some birds which cannot fly:
%
\begin{equation}
  \exists x . \rel{B}(x) \wedge \neg \rel{F}(x)
    \label{eq:nonflying2}
\end{equation}
%
If we have a semantics for $\rel{B}$ and $\rel{F}$ that includes, for
example, penguins then both \eqref{eq:nonflying1} and
\eqref{eq:nonflying2} will be true.

We can prove that these two statements (\eqref{eq:nonflying1} and
\eqref{eq:nonflying2}) are equivalent in predicate logic via two
proofs, one for:
$$\neg (\forall x . \rel{B}(x) \rightarrow \rel{F}(x)) \vdash
\exists x . \rel{B}(x) \wedge \neg \rel{F}(x)$$
and one for:
$$
\exists x . \rel{B}(x) \wedge \neg \rel{F}(x) \vdash
\neg (\forall x . \rel{B}(x) \rightarrow \rel{F}(x))
$$
We will do this later once we have explained more about the
meta-theory of the logic, as well as the introduction and elimination
rules for quantification.

\subsection{Names and binding}

In propositional logic, variables range over propositions, \ie{},
their ``type'' is a proposition. For example, $x \wedge y$ has two
propositional variables $x$ and $y$ which could be replaced with true
or false, or with any other formula. In predicate logic, universal and
existential quantifiers provide \emph{variable bindings}, which
introduce variables ranging over objects in some universe rather than
over propositions. For example, the formula $\forall x . P$ binds
a variable $x$ in the \emph{scope} of $P$. That is, $x$
is available within $P$, but not outside of it. A variable which does
not have a binding in scope is called \emph{free}.

For example, the following formula has free variables $x$ and $y$
and bound variables $u$ and $v$:
%
\begin{equation*}
\rel{P}(x) \, \vee  \, \forall u . \, (\, \rel{Q}(y) \, \wedge \,
\rel{R}(u) \, \rightarrow \exists v . \, (\, \rel{P}(v) \, \wedge \, \rel{Q}(x) \,))
\end{equation*}
%
The following repeats the formula and 
highlights the binders in yellow, the bound variables
in green, and the free variables in red:
%
\begin{equation*}
\highlightR{\rel{P}(x)} \vee \highlight{\forall u} . (\highlightR{\rel{Q}(y)} \wedge \highlightG{\rel{R}(u)}
\rightarrow \highlight{\exists v} .
(\highlightG{\rel{P}(v)} \wedge \highlightR{\rel{Q}(x)}))
\end{equation*}
%
%
In the following formula, there are two syntactic occurrences of a
variable called $x$, but semantically these are different variables:
%
\begin{equation*}
\rel{Q}(x) \wedge (\forall x . \rel{P}(x))
\end{equation*}
%
The $x$ on the left (used with a predicate $\mathsf{Q}$) is free,
whilst the $x$ used with the predicate $\rel{P}$ is bound by the
universal quantifier. Thus, these are semantically two different
variables.

\paragraph{Alpha renaming}
The above formula is
semantically equivalent to the following formula obtained by
consistently renaming bound variables:
%
\begin{equation*}
\rel{Q}(x) \wedge (\forall y . \rel{P}(y))
\end{equation*}
%
Renaming variables is a meta-level operation we can apply to any
formula: we can rename a bound variable as long as we do not rename it
to clash with any other free or bound variable names, and as long as
we rename the variable consistently. This principle is more generally
known as $\alpha$-renaming (alpha renaming) and equality up-to
renaming (equality that accounts for renaming) is known as
$\alpha$-equality. For example, writing $\alpha$-equality as
$=_{\alpha}$ the following equality and inequality hold:
%
\begin{equation*}
\exists x . \rel{P}(x) \rightarrow \rel{P}(y)
  \;\;\; =_{\alpha} \;\;\; 
\exists z . \rel{P}(z) \rightarrow \rel{P}(y)
 \;\;\; \neq_{\alpha} \;\;\; \exists y . \rel{P}(y) \rightarrow \rel{P}(y)
\end{equation*}
%
The middle formula can be obtained from the left formula by
renaming $x$ to a fresh variable $z$. However, in the right-hand
formula, we have renamed $x$ to $y$ which conflates the bound
variable with the free variable $y$ on the right of the implication.
The formula on the right has a different meaning to the left two.

\subsection{Substitution}

Recall in Part C, we used the function $\textit{replace}$ in the DPLL
algorithm where $\textit{replace}(x, Q, P)$ rewrites formula $P$ such
that any occurrences of variable $x$ are replaced with formula
$Q$. This is more generally called \emph{substitution}.

From now on we will use a more compact syntax for substitution
written $$P[t/x]$$ which means: \emph{replace variable $x$
with the variable $t$ in formula $P$} (akin to $\textit{replace}(x, t,
P)$). We will only replace variables with other variables.
Note however that in predicate logic we have to careful
about free and bound variables. Thus, $P[t/x]$ means replace any
\emph{free} occurrences of $x$ in $P$ with object $t$. (One way to remember
this notation is to observe that the letters used in the general
form above are in alphabetical order: $P$ then $t$ then $x$ to
give $P[t/x]$ for replacing $x$ with $t$ in $P$).  This is a common
notation which is also used in the course textbook.

We must be careful to replace only the free
occurrences of variables, that is, those variables which are not in
the scope of a variable binding of the same name. For example, in the
following we have a free $x$ and a bound $x$, so substitution only
affects the free $x$ as such:
%
\begin{equation*}
(\rel{P}(x) \wedge \forall x . \rel{P}(x))[t/x] 
= \rel{P}(t) \wedge \forall x . \rel{P}(x)
\end{equation*}
%
In general, it is best practise to give bound variables a different name
to all other free and bound names in a formula, in order to avoid confusion.

\subsection{Natural deduction rules}

As with propositional logic, we will have elimination and introduction
rules for the new logical operators in first-order logic.

\subsubsection{Universal quantification (\emph{for all})}

Universal quantification essentially generalises conjunction.  That
is, if the objects in the universe over which we are quantifying are
$a_0, a_1, \ldots a_n \in \mathcal{U}$ then universal quantification
of $x$ over a formula $P$ is equivalent to taking the repeated
conjunction of $P$ substituting each object for $x$, \ie{}
%
\begin{equation*}
\forall x . P = P[a_0/x] \wedge P[a_1/x] \wedge
\ldots \wedge P[a_{n}/x]
\end{equation*}
%
(Note that there may be an infinite number of such objects). 
This perspective helps us to understand elimination and introduction
for universal quantification as a generalisation of elimination and
introduction for conjunction:
%
\begin{equation*}
\dfrac{P \wedge Q}
        {P} \; {\wedge_{e1}}
\;\;
\dfrac{P \wedge Q}
  {Q} \; {\wedge_{e2}}
\;\;
\dfrac{P \qquad Q}
         {P \wedge Q} \; {\wedge_i}
\end{equation*}
%
The elimination rule for universal quantification is then as follows:
%
\begin{equation*}
  \dfrac{\forall x . P}
  {P [t/x]} \; {\forall_e} \quad \textit{where $t$ is free when
    replacing $x$ in $P$}
\end{equation*}
%
The intuition is that we can eliminate a $\forall$ by replacing the
bound variable $x$ with an arbitrary variable $t$ which represents \emph{any} of
things ranged over by the $\forall$. This is similar to conjunction
elimination where we eliminate to either of the left or right-hand sides of
the conjunction.

The side condition requires that the variable $t$ is a free variable
 when it is substituted for $x$ in $P$. We'll see an example of
why this is needed. \footnote{The textbook (Huth and Ryan) uses the
  phrase ``\emph{$t$ is free for $x$ in $P$}'' which is a little
  confusing, but means the same as the above side condition for
  $\forall$ elimination: that the variable $t$ remains a free variable
  even once it replaces $x$ inside of $P$.}

Consider the following statement about integers, that for
every integer there exists a bigger integer:
%
\begin{equation*}
  \forall x . \exists y . (x < y)
\end{equation*}
In a natural deduction proof, we can eliminate the $\forall$ with
$t = x_0$ (some fresh variable) to get $\exists y . (x_0 < y)$. We
know nothing about $x_0$, it is just a variable representing any
object (integer) in the set of things ranged over by the $\forall$.
If we instead performed the elimination with the substitution where
$t = y$ (violating the side condition) then we would get
$\exists y . (y < y)$ which is no longer true: it says that there
exists a number which is greater than itself. The problem is that by
performing the substitution $(\exists y. (x < y))[y/x]$ we have
``captured'' the binding of $y$ with the $y$ we are substituting in,
which then changed the meaning of the formula. This is why we have the
side condition.

Next we will see introduction. This uses boxes like we used for
subproofs previously, but the boxes are no longer subproofs, but
instead mark out the scope of a variable. The rule is as follows:
%
\begin{equation*}
\setlength{\arraycolsep}{0.2em}
\dfrac{\fbox{$\begin{array}{lc} \textcolor{blue}{x_0} \;\; & \\ & \vdots \\ & P[x_0/x] \end{array}$}}
{\forall x . P}
\; {\forall_i}
\end{equation*}
%
This says that there is a \emph{scope} (not subproof) that has a
variable $x_0$ (marked in blue in the top corner) which appears in a 
proof concluding with $P[x_0/x]$. If this variable $x_0$ is only used
in this scope, then we can leave the scope and conclude $\forall x
. P$. Thus, the proof inside is of $P$ but where $x$ is replaced by
$x_0$. Here's an example to make this idea more clear:
%
\begin{example}
Prove $\forall x . (\rel{P}(x) \rightarrow \rel{Q}(x)), \forall y
. \rel{P}(y) \vdash \forall z . \rel{Q}(z)$ is valid.

That is, for all $x$ such that $\rel{P}(x)$ implies $\rel{Q}(x)$ and
for all $y$ that $\rel{P}(y)$ then $\rel{Q}(z)$ holds for all $z$. Note
how I have used different names for the bound variables for the sake
of clarity, but the formula would have the exact same meaning if each
bound variable was called $x$ here.

The proof proceeds:
  \begin{logicproof}{2}
  \forall x . (\rel{P}(x) \rightarrow \rel{Q}(x)) & premise \\
  \forall y . \rel{P}(y)                          & premise \\
  \begin{subproof}
    \hspace{-1em}\textcolor{blue}{x_0}
    \;\; \rel{P}(x_0) \rightarrow \rel{Q}(x_0) & $\forall e$ 1 \\
    \;\; \rel{P}(x_0)                          & $\forall e$ 2 \\
    \;\; \rel{Q}(x_0)                          & $\rightarrow_e$ 3, 4
  \end{subproof}
  \forall z . \rel{Q}(z)                       & $\forall i$ 3-5
  \end{logicproof}
\end{example}
%
\noindent
We start with the two premises. The proof then proceeds
with a scope box on lines 3-5: \emph{but remember this is not
a subproof}, it merely serves to delimit the scope of 
the fresh variable $x_0$ which starts on line 3 when
the ``for all'' on line 1 is eliminated.

When we eliminate $\forall y . \rel{P}(y)$ on line 4 we use the same
$x_0$. Since the quantification is universal we can pick the same
object $x_0$ to eliminate line 4 that we ``picked'' for the
elimination on line 3. Line 5 uses modus ponens to get $\rel{Q}(x_0)$
which gives us the formula on which we apply $\forall
i$.

Philosophically we are applying universal quantification introduction
only on $\rel{Q}(x_0)$, but we state the lines in which $x_0$ is
in scope. Once we close this scope box, we can't do anything with
$x_0$. The proof has shown that if we pick an arbitrary object $x_0$
then we get $Q(x_0)$ and so we can conclude that for all
objects $z$ we have $\rel{Q}(z)$ (expressed as $\forall z
. \rel{Q}(z)$).

Existential quantification is more restrictive.

\subsubsection{Existential quantification (\emph{there exists})}

Whilst universal quantification generalises conjunction,
existential quantification generalises disjunction. 
If existential quantification binds variables ranging
over the objects $a_0, a_1, \ldots a_n$ then:
%
\begin{equation}
\exists x . P = P[a_0/x] \vee P[a_1/x] \vee
\ldots \vee P[a_{n}/x]
\label{eq:exists-meaning}
\end{equation}
%
That is, existential quantification is equivalent to the disjunction
of $P$ for every object in the universe, replacing $x$.

Recall the introduction and elimination rules for disjunction:
%
\begin{align*}
\begin{array}{c}\dfrac{P}
  {P \vee Q} \; {\vee_{i1}}
  \qquad
    \dfrac{Q}
  {P \vee Q} \; {\vee_{i2}}\\[3.25em]\end{array}
\qquad \setlength{\arraycolsep}{0em}
\dfrac{\begin{array}{c} \\ \\[0.7em] P \vee Q\end{array} \quad
\fbox{$\begin{array}{c} P \\ \vdots \\ R\end{array}$}
\quad
\fbox{$\begin{array}{c} Q \\ \vdots \\ R\end{array}$}}{R}
\;
{\vee_e}
\end{align*}
%
Disjunction introduction generalises to the following existential
introduction rule:
%
\begin{align*}
\dfrac{P[t/x]}{\exists x . P} \; \exists_i \;\;
\textit{$t$ is free for $x$ in $P$}
\end{align*}
%
That is, we can introduce $\exists x . P$ if we have
$P$ where some other variable $t$ replaces $x$. This
is a bottom-up reading of the rule, where the substitution
is applied to the premise. Note how this rule is essentially
the converse of $\forall elimination$ and thus we can
rather directly prove the following:
%\begin{align*}
%\dfrac{P}{\exists x . P[x/t]} \;\; \exists_i
%\end{align*}

\begin{example}
  Prove $\forall x. \rel{P}(x) \vdash \exists x . \rel{P}(x)$ is valid.

  \begin{logicproof}{2}
    \forall x . \rel{P}(x) & premise \\
    \rel{P}(t)             & $\forall_e$ \\
    \exists x . \rel{P}(x) & $\exists_i$ $\qquad \Box$
   \end{logicproof}
 \end{example}
%
Existential elimination resembles disjunction elimination
but now combines the notion of a variable scope box with a subproof
box:
%
\begin{equation*}
\setlength{\arraycolsep}{0.2em}
\dfrac{\begin{array}{l} \\[2em] \exists x . P\end{array} \quad 
\fbox{$\begin{array}{lc} \textcolor{blue}{x_0} \;\; & P[x_0/x]
 \\ &  \vdots \\ & Q \end{array}$}}{Q}
\end{equation*}
%
The intuition is that the subproof here represents a case for every
single part of the disjunction in~\eqref{eq:exists-meaning} by using
an arbitrary variable $x_0$ that we know nothing about to represent
each atom in the universe. Similarly to disjunction elimination, if we
can then conclude $Q$ (but without using $x_0$ since the scope box
ends here) then we can conclude $Q$ overall.


\begin{restatable}{exc}{duality}
  $\neg \forall x . \rel{P} (x) \vdash \exists x . \neg \rel{P} (x)$
\end{restatable}

\section{Collected rules of natural deduction for
first-order logic}

\vspace{2em}

\setlength{\tabcolsep}{1.54em}
\renewcommand{\arraystretch}{1}
\begin{tabular}{r||c|c}
 & \textit{Introduction} & \textit{Elimination} \\[0.5em] \hline \hline
  $\forall$
& \rule{0cm}{2.25cm} $\setlength{\arraycolsep}{0.2em}
\dfrac{\fbox{$\begin{array}{lc} \textcolor{blue}{x_0} \;\; & \\ & \vdots \\ & P[x_0/x] \end{array}$}}
{\forall x . P}
\; {\forall_i}$
& $\dfrac{\forall x . P}
  {P [t/x]} \; {\forall e} \;$ {\small{$\begin{array}{c}\textit{where $t$ is
                                  free when} \\
\textit{replacing $x$ in $P$}\end{array}$}} \\[1.25em] \hline
$\exists$ 
&
\rule{0cm}{0.75cm}
$\dfrac{P[t/x]}{\exists x . P} \;\exists_i$ {\small{$\begin{array}{c}\textit{where $t$ is
                                  free when} \\
\textit{replacing $x$ in $P$}\end{array}$}}
&
\rule{0cm}{2.25cm}
$\setlength{\arraycolsep}{0.2em}
\dfrac{\begin{array}{l} \\[2em] \exists x . P\end{array} \quad 
\fbox{$\begin{array}{lc} \textcolor{blue}{x_0} \;\; & P[x_0/x]
 \\ &  \vdots \\ & Q \end{array}$}}{Q}$
\end{tabular}

\section{Exercises}

\duality*

\end{document}


\newpage
\thispagestyle{empty}

\section*{Aide memoire}

\noindent
Write down any terms, symbols, concepts, processes, etc. that
you don't yet understand or would like to grasp better.
You can then cross these out as you make progress:
it might be 5 minutes after you first wrote the
item down here, or it might be 5 minutes before the exam.
This will help you to keep a record of the things you
need to work on. \\[0.5em]

\begin{center}
\hspace{-2em}\begin{tabular}{l|l}
  \textbf{Things I don't understand yet} &
   \textbf{Things I need to get better at} \\ \hline
  & \\
  & \\
                                         & \\
                                         & \\
                                         & \\
                                         & \\
                                         & \\
                                         & \\
                                         & \\
                                         & \\
                                         & \\
                                         & \\
                                         & \\
                                         & \\
                                         & \\
                                         & \\
                                         & \\
                                         & \\
                                         & \\
                                         & \\
                                         & \\
                                         & \\
                                         & \\
                                         & \\
                                         & \\
                                         & \\
                                         & \\
                                         & \\
                                         & \\
                                         & \\
                                         & \\
                                         & \\
                                         & \\
                                         & \\
                                         & \\
                                         & \\
                                         & \\
             \end{tabular}
\end{center}


\end{document}