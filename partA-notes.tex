Truth tables are a handy way to give meaning to logical operators and
formulas, in terms of truth or falsehood. But they are difficult to
use for reasoning about anything but small formulas since the number
of rows is $2^n$ where $n$ is the number of contingent
formulas. For example, the truth table for
$P \vee (Q \wedge R) \rightarrow (P \vee Q) \wedge (P \vee R)$ has
$2^3 = 8$ rows as there are three contingent formulas: $P$, $Q$, and
$R$ whose truth or falsehood determines the truth or falsehood of the
overall formula. Enumerating all possibilities takes time even for
this modest truth table, and is error prone (humans are
bad at repetitive detailed tasks).

Instead, logicians have constructed formal languages (calculi) for
building complex chains of reasoning (proofs) in a more
compact form. In this course, we use the \emph{natural deduction} calculus
due to 20$^{th}$ century logicians such as Gerhard Gentzen
(who formulated natural deduction in 1934) and Dag Pravitz (who
promoted this style in the 1960s and converted
various other results of Gentzen into the natural deduction style).

\section{Natural deduction for propositional logic}

Natural deduction provides a system of \emph{inference rules} which
explain how to construct and deconstruct formulas to build
a logical argument. These rules represent the derivation of a formula
(the \emph{conclusion}) from several other formulas that are assumed or known to be true
(the \emph{premises}) via the notation:
%
\begin{equation*}
  \dfrac{\textit{premise}_1 \ldots \quad \ldots \textit{premise}_n}
        {\textit{conclusion}}
    \; {(\textit{label})}
\end{equation*}
%
Each logical operator (like conjunction $\wedge$ (``and'') and
disjunction $\vee$ (``or'')), will have one or more rules for \emph{introducing}
that operator (deriving a conclusion formula which uses that operator) and one or
more rules for \emph{eliminating} that operator (deriving a conclusion
from a premise formula that uses that operator). These rules can be stacked together to form a
logical argument: \emph{a proof}, which looks like the following:

\begin{equation*}
\dfrac{\dfrac{P_1 \quad P_2}{P_3} \quad P_4}{P_5}
\end{equation*}
%
This isn't an actual concrete proof yet, it's just an example of how
a natural deduction proof looks when you stack together
its inference rules. Informally, we might have something like the
following:
%
\begin{equation*}
\dfrac{\dfrac{\textit{I forgot my coat} \quad \textit{It's
      raining}}{\textit{I get wet}} \quad \textit{My hairdryer broke}}
       {\textit{My hair remains wet}}
\end{equation*}
%
But this isn't always the most helpful format to derive the
proofs. Instead, we'll use a special ``box''-like notation called
Fitch-style which you can find in the recommend textbook
\emph{Logic in Computer Science} by Huth and Ryan.
We will still apply the rules of natural deduction, but the Fitch-style
gives us a nice way to layout the proof as we are deriving it.

We will step through the natural deduction rules for the core logical
operators: $\wedge$ (conjunction/and), $\vee$ (disjunction/or),
$\rightarrow$ (implication/if-then), $\neg$ (negation), as well as
truth $\top$ and falsehood/falsity which is often written in
propositional logic as $\bot$ (pronounced
``bottom'').\footnote{\emph{Bottom} $\bot$ is often used in maths and
  computer science to represent undefined values or behaviour. In
  logic, if have arrived at falsity $\bot$ during a proof then we are
  in a situation where anything could be true as we've arrived at a
  logically inconsistent situation. This is sometimes quite useful for
  doing proofs-by-contradiction, as we will see in
  Section~\ref{sec:negation}.} Note that, in the literature, logical operators
  are sometimes alternatively called \emph{logical connectives}.

\subsection{Properties of formulas}

We will consider three properties that a logical formula may have:

\begin{itemize}[leftmargin=1.5em]
  \item \emph{valid}: a formula which is always true (also
  called a \emph{tautology}). We will
  mostly prove the validity of formulas in \emph{Part A}. For example, $P \wedge Q
  \rightarrow Q \wedge P$ is true no matter the
  truth/falsehood of $P$ and $Q$.
%
  \item \emph{satisfiable}: a formula which is true for
  \emph{some} assignments of truth/falsehood to the atoms/variables it contains, \eg{},
  $a \wedge b$ is satisfiable, and the \emph{satisfying assignment} is that
  $a \mapsto \top$ and $b \mapsto \top$ (both must be true). A valid formula
  is trivially satisfiable (true for any and all assignments)

  Satisfiability, and an algorithm for finding a satisfying assignment,
  will be covered in \emph{Part C}.
  %

  \item \emph{unsatisfiable}: a formula which is always false (all
  rows in the truth table are false) regardless of assignments to the
  variables/atom, \eg{} $P \wedge \neg (P \vee Q)$. We can
  prove a formula is unsatisfiable by proving its negation is valid.
\end{itemize}

\subsubsection{Entailment and sequents}
\label{sec:entailment}

Suppose we have a set of formulas $P_1, \ldots, P_n$ from which we
want to prove $Q$ by applying the rules of a particular logic
(propositional logic here). The formulas $P_1, \ldots, P_n$ are the
premises and $Q$ is our goal conclusion. This is often written using the
following notation called a \emph{sequent}:
%
\begin{equation*}
P_1, \ldots, P_n \vdash Q
\end{equation*}
%
The turnstile symbol $\vdash$ is read as \emph{entails} and the
premises to the left are sometimes call the \emph{context} of assumed
formulas. This is a compact representation of a formula $Q$ along with
any assumptions used to deduce it.

We say a sequent is \emph{valid} if there is a proof that derives the conclusion from the premises. For example,
$P \wedge Q \vdash Q \wedge P$ is valid, meaning from $P \wedge Q$
there is a proof of $Q \wedge P$. %If we wish to explain that
%a sequent is invalid we can write $P \not\vdash Q$ meaning, it
%is not valid that $P$ entails $Q$.

When there are no premises, we often drop the $\vdash$, \ie{},
writing something like ``\emph{$(P \wedge Q) \rightarrow P$ is valid}''
instead of ``\emph{$\vdash (P \wedge Q) \rightarrow P$ is
valid}''.

\subsection{Conjunction (``and'')}

Recall the truth table for conjunction:
%
\begin{center}
\begin{tabular}{cc|c}
  $P$ & $Q$ & $P \wedge Q$ \\ \hline
  F & F & F \\
  F & T & F \\
  T & F & F \\
\rowcolor{yellow}  T & T & T
\end{tabular}
\end{center}
%
From this, we see that in order to conclude the truth of $P \wedge Q$
we need the truth of $P$ and the truth of $Q$. This justifies the
following natural deduction rule for \emph{introducing} conjunction:
%
\begin{align*}
  \dfrac{P \qquad Q}
        {P \wedge Q} \; {\wedge_i}
\end{align*}
%
The label is subscripted with `i' for introduction.
That is, given two premises that $P$ is true and $Q$ is true, then
$P \wedge Q$ is true. There is just one introduction rule,
corresponding to the fact that there is only one ``true'' row
for $P \wedge Q$ in the truth table (highlighted above).

The letters $P$ and $Q$ are place-holders here for \emph{any
  propositional formula} so we could instantiate the rule
to something like the following if we needed it:
%
\begin{equation*}
\dfrac{(P \vee R \rightarrow Q) \quad S}
      {(P \vee R \rightarrow Q) \wedge S} \; {\wedge_i}
\end{equation*}
%
What about elimination?
Reading the highlighted line in the truth table from right-to-left shows how to
\emph{eliminate} a conjunction, \ie{}, what smaller formulas
can we conclude are true if we know that $P \wedge Q$ is true? We get
two rules:
%
\begin{align*}
  \dfrac{P \wedge Q}
        {P} \; {\wedge_{e1}}
  \qquad & \qquad
      \dfrac{P \wedge Q}
        {Q} \; {\wedge_{e2}}
\end{align*}
%
The labels have a subscript `e' for elimination; this convention will
continue.  (\emph{Aside}: If the syntax of the inference rules let us
have multiple conclusions then we could collapse the two eliminations
into one rule, but natural deduction instead has single
conclusions. There are different proof systems which allow
multiple conclusions (like the \emph{sequent calculus}) but we won't cover that here).

Let's write a proof with just these rules by stacking them together
into a \emph{proof tree}.

\begin{example}
\label{exm:assoc-conj}
  For any formula $P$, $Q$, $R$ then $P \wedge (Q \wedge
  R) \vdash (P \wedge Q) \wedge R$ is valid, \ie{} given $P \wedge (Q \wedge
  R)$ we can prove $(P \wedge Q) \wedge R$.
%
\newcommand{\conge}[1]{\wedge_{e#1}}
  \begin{align*}
    \dfrac{
    \dfrac{\dfrac{P \wedge (Q \wedge R)}
    {P}\conge{1}
    \dfrac{\dfrac{P \wedge (Q \wedge R)}
    {Q \wedge R} \conge{2}}{Q} \conge{1}}
    {P \wedge Q} {\wedge_i}
    \dfrac{\dfrac{P \wedge (Q \wedge R)}
    {Q \wedge R} \conge{2}}{R} \conge{2}}
    {(P \wedge Q) \wedge R} {\wedge_i}
  \end{align*}
The root of the tree is our goal $(P \wedge Q) \wedge R$, which is
built from the premises on the line above. These chains of reasoning
go up to the ``leaves'' of the tree, which is the assumed formula
$P \wedge (Q \wedge R)$. At each step (each line) we've applied
either conjunction introduction or one of the conjunction elimination
rules (as can be seen from the labels on the right).
\end{example}
\noindent
The following exercise is to prove the converse of the above property.
%
\begin{restatable}{exc}{assoc}
  \label{exm:assoc}
  Prove that $(P \wedge Q) \wedge R \vdash P \wedge (Q \wedge R)$ is
  valid by instantiating and stacking together inference rules.
\end{restatable}
%
This proof, and the one above, together imply that conjunction
$\wedge$ is \emph{associative}, \ie{},
$P \wedge (Q \wedge R) = (P \wedge Q) \wedge R$. We come back
to such equations on formulas in Section~\ref{sec:algebraic}.

\paragraph{Fitch-style proof}
So far we have constructed proofs by stacking natural
deduction inference rules on top of each other. This leads us towards a
\emph{bottom-up} proof strategy starting with the goal and working up
towards the premises. In this course, we use a different
approach called ``Fitch-style''. This style begins
with assumptions, numbers each line of a proof, and uses indentation
and boxes to represent sub-proofs and the scope of their assumptions.
%You can work both bottom-up and top-down in this approach.

The previous proof tree is rewritten in the following way in Fitch notation:
%
  \begin{logicproof}{2}
    P \wedge (Q \wedge R) & premise \\
    P                     & $\wedge_{e1}$ 1 \\
    Q \wedge R            & $\wedge_{e2}$ 1 \\
    Q                     & $\wedge_{e1}$ 3 \\
    R                     & $\wedge_{e2}$ 3 \\
    P \wedge Q            & $\wedge_i$ 2, 4 \\
    (P \wedge Q) \wedge R & $\wedge_i$ 6, 5 $\quad \Box$
  \end{logicproof}
%
  The proof follows in a number of linear steps. On the left, we number
  each line of the proof. On the right, we explain which rule was
  applied to which formulas as premises, \eg{}, on the second line we have
  applied conjunction elimination $\wedge_{e1}$ taking line $1$ as the
  premise, concluding with the
  formula $P$. Or for example, on line 6, conjunction introduction is
  applied to lines 2 and 4 (i.e., they are taken as the premises of
  the rule) to get $P \wedge Q$.  We finish on line
  7 with our goal, which is marked with $\Box$ which is a way of
  saying the proof is finished and we've reached our goal
  (the symbol means $Q.E.D$ which is an abbreviation of
  \emph{quod erat demonstrandum}, Latin for ``what was to be
  demonstrated'').

We haven't used any sub-proofs yet (which have a box drawn around
them); these appear in the next subsection on implication.

 \textbf{Order of numbers in labels} $\;$
Note that the order of the line numbers in labels tells us the order
of the premises to a natural deduction rule and so the order is
important. For example, line 6 above applies ($\wedge_i$ 2, 4) to
introduce $P \wedge Q$, but if it was actually ($\wedge_i$ 4, 2) we
would be introducing $Q \wedge P$ which is not our intended goal.

\begin{restatable}{exc}{andReproof}
Rewrite your proof to Exercise~\ref{exm:assoc} using
Fitch style.
\end{restatable}

\begin{remark}(\textbf{important})
Depending on what is being proved, a top-down approach (starting
from the premises) or bottom-up approach (starting
from the goal/conclusion) can be easier. In practice, if you are
stuck it can help to start \emph{at both ends} and work towards the
middle. You can do this by putting the goal near a bottom of a piece
of paper, giving enough space to meet in the middle. You can sort out
the numbering afterwards.

It doesn't matter if things get messy---the primary goal is to reach a
proof. You can rewrite it afterwards to be more clear; you should
do so in your class work and assessments.
\end{remark}

\subsection{Implication}

Recall the truth table for implication:
%
\begin{center}
\begin{tabular}{cc|c}
  $P$ & $Q$ & $P \rightarrow Q$ \\ \hline
  \rowcolor{yellow} F & F & T \\
  \rowcolor{yellow} F & T & T \\
  T & F & F \\
  \rowcolor{yellow} T & T & T
\end{tabular}
\end{center}
%
Implication $P \rightarrow Q$ is interesting because if
$\neg P$ (if $P$ is false) then $Q$ can be true or false, \ie{},
$Q$ can be anything if $P$ is false (the top two lines).

\begin{restatable}{exc}{implProperty}
Recall that $P \rightarrow Q = \neg P \vee Q$. Show this is true
by comparing the truth tables for each side of this equation.
\end{restatable}

As with conjunction, we'll consider the two styles of
rule: elimination and introduction.
The elimination rule for implication in natural deduction is:
\begin{align*}
\dfrac{P \rightarrow Q \qquad P}{Q} {\rightarrow_e}
\end{align*}
%
This rule is also known as \emph{modus ponens}.\footnote{\emph{modus
ponens} is short for the Latin phrase \emph{modus ponendo ponens}
which means ``the way that affirms by affirming''.} It says that if
we know $P \rightarrow Q$ and we know $P$ then we know $Q$. You can
verify the soundness of this rule by looking at the truth
table: indeed $Q$ is true when both $P \rightarrow Q$ and $P$ are
true.

There are various other natural deduction rules one might construct by
looking at the truth table-- but this one can be used to derive the
others. The particular set of natural deduction rules we look at was
carefully honed by logicians to provide a kind of ``minimal'' calculus
for proofs.

Introduction of an implication $P \rightarrow Q$ follows from a
\emph{subproof} (which is drawn in a box) which starts with an
assumption of $P$ and ends with $Q$ as a conclusion after any number
of steps. The rule is written as follows:
%
\begin{align*}
\setlength{\arraycolsep}{0em}
\dfrac{\fbox{$\begin{array}{c} P \\ \vdots \\ Q\end{array}$}}
      {P \rightarrow Q} {\rightarrow_i}
\end{align*}
%
Thus subproofs in both tree- and Fitch-style proofs
are of the form:
%
\begin{equation*}
\setlength{\arraycolsep}{0em}
\fbox{$\begin{array}{c} \textit{assumption} \\ \vdots \\[0.5em]
         \textit{conclusion} \end{array}$}
\end{equation*}
%
\begin{remark} (\textbf{important})
%
  When we start a subproof box, the first formula is always an
  assumption, which we are free to choose. When the box is closed, the assumption does not go away
  but becomes the premise of the implication when applying the
  $\rightarrow_i$ rule.

  \emph{This is an important point}: when proving a theorem we
  have to be careful not to introduce additional assumptions which are
  not part of the theorem. For example, let's say we are proving a
  theorem expressed by a formula $Q$ but in doing so we assume $P$
  but $P$ is not one of $Q$'s assumptions.  Then
  instead we will have proved $P \rightarrow Q$ rather than $Q$.  This
  is something to keep in mind when writing complex
  proofs by hand in an informal way. The proof system of natural deduction allows us to keep
  track of our assumptions and their eventual inclusion in the final
  result.

  Aside: mechanised proof assistants (software systems in which we
  can write machine-checked proofs, such as \emph{Coq},
  \emph{Isabelle}, \emph{Agda}) have a similar basis to
  natural deduction and give us confidence and precision in writing proofs.
\end{remark}
%
There is an alternate presentation of natural deduction called
\emph{sequent-style natural deduction} (described in the
appendix for this part) where the inference rules are expressed in
terms of sequents $P_1, \ldots, P_n \vdash Q$. This won't be assessed
on the course, but is worth looking at if you want to read more widely
on logic. Another proof calculus (also due to Gentzen) is the
\emph{sequent calculus} which you can read about elsewhere.

 \begin{example}
   The following simple formula about conjunction and implication is
   valid: $\vdash (P \wedge Q) \rightarrow (Q \wedge P)$. Here is its proof
   in Fitch-style:
%
\begin{logicproof}{2}
\begin{subproof}
P \wedge Q & assumption    \\
P          & $\wedge_{e1}$ 1 \\
Q          & $\wedge_{e2}$ 1 \\
Q \wedge P & $\wedge_i$ 3, 2
\end{subproof}
P \wedge Q \rightarrow Q \wedge P & $\rightarrow_i$ 1-4 $\qquad \Box$
\end{logicproof}
In the last line, we apply implication introduction and we
label it with the range of the lines of the subproof used (in this
case 1-4).
\end{example}

\begin{remark}
In Example~\ref{exm:assoc-conj} we proved
that given $P \wedge (Q \wedge R)$ then $(P \wedge Q) \wedge
R$. We can turn this into an implication
$P \wedge (Q \wedge R) \rightarrow (P \wedge Q) \wedge R$
simply by using implication introduction on the original
proof. Indeed, we have a general meta-theorem of propositional
logic, that if $P \vdash Q$ then $\vdash P \rightarrow Q$, and
vice versa.
\end{remark}

\begin{example}
The following is valid:
\begin{align*}
(P \rightarrow (Q \rightarrow R))
\rightarrow
((P \rightarrow Q)
\rightarrow
(P \rightarrow R))
\end{align*}
%
Here is its proof:
%
\begin{logicproof}{5}
 \begin{subproof}
  P \rightarrow (Q \rightarrow R) & ass. \\ % 1
  \begin{subproof}
  P \rightarrow Q   & ass.  \\ % 2
  \begin{subproof}
   P               & ass. \\ % 3
   Q \rightarrow R & $\rightarrow_{e}$ 1, 3 \\ % 4
   Q               & $\rightarrow_{e}$ 2, 3 \\ % 5
   R               & $\rightarrow_{e}$ 4, 5 % 6
  \end{subproof}
   P \rightarrow R & $\rightarrow_{i}$ 3-6 % 7
  \end{subproof}
  (P \rightarrow Q) \rightarrow (P \rightarrow R) & $\rightarrow_{i}$
  2-7 % 8
  \end{subproof}
(P \rightarrow (Q \rightarrow R))
\rightarrow ((P \rightarrow Q) \rightarrow (P \rightarrow R))
& $\rightarrow_i$ 1-8 $\quad \Box$
\end{logicproof}
%
Here we have an example of multiple nesting of subproofs.
(Tip: I proved this by working top-down and bottom-up at the same
time).
\end{example}

Occasionally it is useful to ``copy'' a
formula from earlier in a proof. For example,
the following proof of $\vdash P \rightarrow P$ copies
a formula from one line of the proof to the other in order
to introduce a trivial implication:
%
\begin{logicproof}{2}
\begin{subproof}
P  & assumption \\
P  & copy 1
\end{subproof}
P \rightarrow P & $\rightarrow_i$ 1-2 $\quad \Box$
\end{logicproof}

\begin{restatable}{exc}{kcombinator}
Prove $P \rightarrow (Q \rightarrow P)$ is valid.
\end{restatable}

\subsubsection{Bi-implication  (``if and only if`'')}
\label{sec:bi-implication}

Propositional logic often includes the bi-implication operator
$\leftrightarrow$ also read as ``\emph{if and only if}'' and sometimes
written as \emph{iff} (double f). A bi-implication $P \leftrightarrow Q$ is
equivalent to the conjunction of two implications, pointing in
opposite directions:
%
\begin{align*}
P \leftrightarrow Q \; \stackrel{\text{def}}{=} \; (P \rightarrow Q) \wedge (Q \rightarrow P)
\end{align*}
%
This means that $P$ and $Q$ have exactly the same truth table, or we
say they are \emph{equivalent} (see Section~\ref{sec:algebraic} later).

Therefore to construct or deconstruct a logical
bi-implication one can consider it as ``implemented'' by
conjunction and implication, reducing the number of
introduction/elimination rules that need to be
remembered. Nonetheless, thinking about what these
introduction/elimination rules would be is a nice exercise.

\begin{restatable}{exc}{biimplRules} (optional)
Try to derive your own elimination and introduction rules for
bi-implication. There is usually one introduction and two eliminations.
\end{restatable}



\subsection{Disjunction (``or'')}

Recall the truth table for disjunction, which has three rows
where $P \vee Q$ is true:
%
\begin{center}
\begin{tabular}{cc|c}
  $P$ & $Q$ & $P \vee Q$ \\ \hline
  F & F & F \\
\rowcolor{yellow} F & T & T \\
\rowcolor{yellow}  T & F & T \\
\rowcolor{yellow}  T & T & T
\end{tabular}
\end{center}
%
The fact that we can conclude $P \vee Q$ from either $P$
or from $Q$ separately justifies the following two introduction
rules for disjunction in natural deduction:
%
\begin{align*}
  \dfrac{P}
  {P \vee Q} \; {\vee_{i1}}
  \qquad
    \dfrac{Q}
  {P \vee Q} \; {\vee_{i2}}
\end{align*}
%

\begin{example}
Prove $(P \wedge Q) \rightarrow (P \vee Q)$ is valid.

  \begin{logicproof}{2}
    \begin{subproof}
    P \wedge Q & assumption \\
    P          & $\wedge_{e1}$ 1 \\
    P \vee Q   & $\vee_{i1}$ 2
  \end{subproof}
  P \wedge Q \rightarrow P \vee Q & $\rightarrow_i$ 1-3 $\quad \Box$
  \end{logicproof}
  %
  This could have been written equivalently as a natural deduction
  proof tree:
  %
  \begin{align*}
   \dfrac{
    \fbox{$\begin{array}{c}
           \dfrac{P \wedge Q}
                 {\dfrac{P}{P \vee Q} {\vee_{i1}}} {\wedge_{e1}}
          \end{array}$}
    }{(P \wedge Q) \rightarrow (P \vee Q)} {\rightarrow_{i}}
  \end{align*}
  %
  This will be the last tree-based proof we see; from now on we'll
   keep using Fitch style.
\end{example}
%
What about disjunction elimination? Given the knowledge that $P \vee
Q$ is true, what can we conclude? Either $P$ is true, or $Q$ is true, or both
are true. Therefore, we don't know exactly what true formulas we can derive
from $P \vee Q$, we just know some possibilities.


The natural deduction way of eliminating disjunction is to have two
subproofs as premises which are contingent on the assumption of
either $P$ or $Q$:

\begin{align*}
\setlength{\arraycolsep}{0em}
\dfrac{\begin{array}{c} \\ \\[0.7em] P \vee Q\end{array} \quad
\fbox{$\begin{array}{c} P \\ \vdots \\ R\end{array}$}
\quad
\fbox{$\begin{array}{c} Q \\ \vdots \\ R\end{array}$}}{R}
\;
{\vee_e}
\end{align*}

\begin{example}
For any propositions $P, Q, R$ then $(P \wedge Q) \vee (P \wedge
R) \rightarrow P$ is valid.
%
  \begin{logicproof}{3}
    \begin{subproof}
      (P \wedge Q) \vee (P \wedge R) & assumption \\
      \begin{subproof}
        P \wedge Q  & assumption \\
        P           & $\wedge_{e1}$ 2
      \end{subproof}
      \begin{subproof}
        P \wedge R & assumption \\
        P          & $\wedge_{e1}$ 4
      \end{subproof}
        P          & $\vee_{e}$ 1, 2-3, 4-5
    \end{subproof}
    (P \wedge Q) \vee (P \wedge R) \rightarrow P & $\rightarrow_{i}$,
    1-6 $\quad \Box$
  \end{logicproof}
  You can see that the application of disjunction elimination $\vee_e$
  involves three things: a disjunctive formula (line 1) and two
  subproofs (lines 2-3 and lines 4-5) which respectively assume the two subformulas of
  disjunction and conclude with the same formula ($P$), which forms the
  conclusion of the subproof on line 6.
\end{example}

An important point to remember is that \textbf{you cannot just
  eliminate a disjunction $P \vee Q$ into one side, e.g. to $P$}. This
would be \emph{unsound} as we can see from the truth table and it is
not what disjunction elimination provides us. If $P \vee Q$ is true it
might be because $Q$ is true, and not because of $P$, so we cannot
just conclude $P$. Consider the true statement $(\textit{This module has
  module code CO519}) \vee (\textit{It sunny every day in
England})$--- we cannot from this conclude that it is true that
\textit{It is sunny every day in England}.

\begin{restatable}{exc}{orassoc}
  Prove $P \vee Q \rightarrow Q \vee P$ is valid.
\end{restatable}

\begin{remark}
  From looking at the truth table for disjunction, one might wonder
  why disjunction elimination does not look like:
  %%
\begin{align*}
\setlength{\arraycolsep}{0em}
\dfrac{\begin{array}{c} \\ \\[0.7em] P \vee Q\end{array} \quad
\fbox{$\begin{array}{c} P \\ \vdots \\ R\end{array}$}
\quad
\fbox{$\begin{array}{c} Q \\ \vdots \\ R\end{array}$}
  \quad
  \fbox{$\begin{array}{c} P \wedge Q \\ \vdots \\ R\end{array}$}}{R}
{\vee_e}
\end{align*}
  %%
  This would match more closely the idea of reading the truth-table
  ``backwards'' from right-to-left on true values of $P \vee Q$. The
  reason we don't have this is that natural deduction strives for
  minimality and the third subproof with assumption $P \wedge Q$ is
  redundant since if we have $P \wedge Q$ true we can apply
  either the subproof $\fbox{$P \ldots R$}$ or the subproof
  $\fbox{$Q \ldots R$}$ by first applying $\wedge_{e1}$ or
  $\wedge_{e2}$ to the assumption $P \wedge Q$ to get $P$ or $Q$ respectively.
%and the above rule can be derived from the disjunction
%  elimination shown here with just two subproofs:
%  \begin{align*}
%\setlength{\arraycolsep}{0em}
%\dfrac{\begin{array}{c} \\ \\[0.7em] P \vee Q\end{array} \quad
%\Delta = \fbox{$\begin{array}{c} P \\ \vdots \\ R\end{array}$}
%\quad
%\fbox{$\begin{array}{c} Q \\ \vdots \\ R\end{array}$}
%  \quad
%  \fbox{$\begin{array}{c} \dfrac{\dfrac{P \wedge Q}{P}
%           {\wedge_{e1}}}{\Delta} \\ \vdots \\ R \end{array}$}}{R}
%{\vee_e}
%\end{align*}
%  That is, if we name the subproof of $\fbox{$P \ldots R$}$ as $\Delta$ and
%  then we can reuse this along with conjunction elimination to get the
%  proof for $\fbox{$P \wedge Q \ldots R$}$. In fact, there is another way to
%  derive this, where we reuse the subproof of $\fbox{$Q \ldots R$}$ and use
%  $\wedge_{e2}$ to build a proof of $\fbox{$P \wedge Q \ldots R$}$.
  \end{remark}

\subsection{Negation}
\label{sec:negation}

Negation introduction and elimination are given by:
%
\begin{align*}
\setlength{\arraycolsep}{0em}
\dfrac{
\fbox{$\begin{array}{c} P \\ \vdots \\ \bot\end{array}$}}
      {\neg P} \; {\neg_i}
\qquad\quad
\dfrac{P \qquad \neg P}{\bot} {\neg_e}
\end{align*}
%
Introduction says that given a subproof that assumes $P$ but ends in
falsehood $\bot$ then we know $\neg P$. This is similar to the notion of
\emph{proof by contradiction}, which is derived from negation introduction (see below).

Elimination states that given a proof of $P$ and a simultaneous proof of
$\neg P$ then we conclude falsehood $\bot$, \ie{}, we have a
logical inconsistency on our hands and so end up proving false:
$P$ and $\neg P$ cannot both be true at the same time.

\begin{example}
For all $P, Q$ then $P \rightarrow Q \vdash \neg Q \rightarrow
\neg P$.
%
\begin{logicproof}{3}
    P \rightarrow Q  & premise \\
    \begin{subproof}
      \neg Q        & assumption \\
      \begin{subproof}
        P           & assumption \\
        Q           & $\rightarrow_{e}$ 1, 3 \\
        \bot        & $\neg_{e}$ 2, 4
      \end{subproof}
      \neg P       & $\neg_i$ 3-5
     \end{subproof}
    \neg Q \rightarrow \neg P & $\rightarrow_i$ 2-6 $\qquad \Box$
\end{logicproof}
\end{example}

\begin{remark}
  This example is often given as a derived inference rule called
\emph{modus tollens}\footnote{\emph{modus
tollens} is short for the Latin phrase \emph{modus tollendo tollens}
which means ``the way that denies by denying''.}
that is similar to modus ponens (implication elimination):
%
\begin{align*}
\dfrac{P \rightarrow Q \quad \neg Q}
      {\neg P} \; {\emph{mt}}
\end{align*}
%
If an inference rule can be derived from others we say it is
\emph{admissible}. The system of rules we take as the basis for
natural deduction reasoning contains no admissible rules.
\end{remark}

\begin{remark}
  If we want to prove a formula $P$ is unsatisfiable then we can
  instead prove that $\neg P$ is valid (always true), hence proving
  that $P$ is unsatisfiable (always false).
\end{remark}

\begin{restatable}{exc}{disproveEx}
Prove that $P \wedge \neg (P \vee Q)$ is unsatisfiable.
\end{restatable}

\begin{remark}
  Some formulae are not valid, \eg{}, $P \rightarrow \neg P$, which
  can be seen from drawing its truth table. However, this formula is
  \emph{satisfiable}, if $P$ is false then $P \rightarrow \neg P$ is
  true. Natural deduction does not help us to prove
  satisfiability. Part C will look at algorithmic approaches to
  deciding satisfiability.
\end{remark}

\subsubsection{Double negation}

A special rule called \emph{double-negation elimination}
removes double negations on a proposition:
%
\begin{align*}
\dfrac{\neg \neg P}{P} \;\; {\neg\neg_e}
\end{align*}
%
\begin{example}
The principle of \emph{proof by contradiction} is represented by
following the derived inference rule:
%
\begin{equation*}
\setlength{\arraycolsep}{0em}
\dfrac{
\fbox{$\begin{array}{c} \neg P \\ \vdots \\ \bot\end{array}$}}
      {P} \; {\textsc{PBC}}
\end{equation*}
%
That is, if we assume $\neg P$ and conclude $\bot$ (i.e. we get a
contradiction), then we have $P$.
To show how to derive this, let the subproof in the above rule be
called $\Delta$, then we construct the following proof:
%
\begin{logicproof}{2}
\begin{subproof}
\neg P & ass. \\
\Delta & $\vdots$ \\
\bot &
\end{subproof}
\neg \neg P & $\neg_i$ 1-3 \\
P           & $\neg\neg_e$ 4
\end{logicproof}
Of course, $\Delta$ might be longer than 3 lines, but we use
the numbering above proof for clarity.
\end{example}

\subsection{Truth and falsity}

If we have $\bot$ (false), then we can derive any formula:
%
\begin{align*}
\dfrac{\bot}{P} \; {\bot_e}
\end{align*}
%
The embodies the principle that if we have logical inconsistency
then anything goes.

There is no $\bot$ introduction as such, though $\neg_{e}$
provides a kind of $\bot$ introduction (from conflicting formula).
Dually, we can always introduce truth from no premises, but there
is no elimination:
%
\begin{align*}
\dfrac{\qquad}{\top} \; {\top_i}
\end{align*}
%

\subsection{A further derived rule: Law of Excluded Middle}

An interesting rule that we can derive in the propositional logic
is called the \emph{Law of Excluded Middle} or LEM for short. It says
that for any formula $P$ we have the following valid rule:
%
\begin{equation*}
\dfrac{\qquad\qquad}{P \vee \neg P} \textsc{lem}
\end{equation*}
%
\ie{}, for any formula $P$, either $P$ is true or $\neg P$ is true. Here
is its derivation:
%
\begin{logicproof}{2}
  \begin{subproof}
    \neg (P \vee \neg P) & ass. \\
    \begin{subproof}
      P  & ass. \\
      P \vee \neg P & $\vee_{i1}$ 2 \\
      \bot          & $\neg_e$ 3, 1
    \end{subproof}
    \neg P          & $\neg_i$ 2-4 \\
    P \vee \neg P   & $\vee_{i2}$ 5 \\
    \bot            & $\neg_e$ 6, 1
  \end{subproof}
\neg \neg (P \vee \neg P) & $\neg_i$ 1-7 \\
P \vee \neg P & $\neg\neg_e$ 8 $\;\;\Box$
\end{logicproof}
%
This rule can be useful in particular proofs.

\begin{restatable}{exc}{lemp}
Using LEM, prove that $\,P \rightarrow Q \vdash \neg P \vee Q\,$ is valid.
\end{restatable}

\paragraph{Aside: constructive vs non-constructive logic}

In this course, we study a particular kind of propositional logic
called \emph{classical} or \emph{non-constructive}
logic. Another variant is known as \emph{intuitionistic} or
\emph{constructive} logic which has a slightly different
set of proof rules: $\neg\neg_e$ is not included. By removing
double-negation elimination we can no longer derive
proof-by-contradiction or LEM.

The central principle of constructive logic is to reason about
\emph{provability} rather than \emph{truth}, which is what
we reason about in classical logic. In
constructive logic, a
formula $P$ represents the proof of formula $P$: a mathematical object
witnessing the truth of $P$. The
inference rules of natural deduction are now about preserving proof
rather than truth, \eg{}, conjunction elimination says given a proof
of $P \wedge Q$ then we can prove $P$.

In constructive logic, $\neg\neg_e$ is rejected since it would mean we
can get a proof of $P$ from a proof of the negation of the negation of
$P$, but the proof of $\neg\neg P$ is not really a proof of $P$. The
presence of $\neg\neg_e$ is particularly troublesome for provability
when used in the derivation of LEM. If LEM was allowed in constructive
logic, then it would be saying that for any formula $P$ we have
constructed either a proof of $P$ or a proof of $\neg P$. But what is
that proof and where has it come from? It would be out of thin air!
(since LEM has no premises). The essence of constructive logic is to disallow such
things so that we always know we have a concrete proof (sometimes
called a \emph{witness}) for our
formulas, constructed from proofs of its subformulas or
premises. Section 1.2.5 of the Huth and Ryan textbook gives more
detail and shows an example mathematical proof about rational numbers
in classical logic which cannot be proved constructively.

A useful thing about constructive logics is that they correspond to
type systems in functional programming: a result known as the
\emph{Curry-Howard correspondence}. This is a rich source
of ideas in programming languages. Unfortunately, we will not have
time to study that here but it will appear at the end of CO545:
\emph{Functional and Concurrent Programming}.

\section{Algebraic properties of logic}
\label{sec:algebraic}

In Section~\ref{sec:bi-implication}, we saw \emph{bi-implication}
$\leftrightarrow$, a derived logical operator where:
%
\begin{equation*}
P \leftrightarrow Q \; \stackrel{\text{def}}{=} \; (P \rightarrow Q) \wedge (Q \rightarrow P)
\end{equation*}
%
If there is a bi-implication between two formulas then it means their
truth tables are exactly the same and we can see the two formulas as
equivalent. From the natural deduction rules (or from the truth
tables) a number of general equivalences can be derived which give
us algebraic laws about propositional formula
(this was first described by Boole in 1847). We'll use the operator
$=$ to denote equivalent formula, where $P = Q$ can be
read as ``$P$ is equivalent to $Q$'' (or that there is a
bi-implication). We can thus prove equivalences by proving the
bi-implication of the two formula.

The following lists a number of equivalences between general
propositional formula which amounts to algebraic properties of
conjunction and disjunction: \\

\setlength{\tabcolsep}{0.75em}
\begin{tabular}{r|ll}
\textbf{property} & conjunction & disjunction \\ \hline
  \emph{idempotence} & $P \wedge P = P$ & $P \vee P = P$ \\
  \emph{commutativity} & $P \wedge Q = Q \wedge P$ & $P \vee Q = Q
                                                     \vee P$ \\
  \emph{associativity} & $(P \wedge Q) \wedge R = P \wedge (Q \wedge
                         R)$ & $(P \vee Q) \vee R = P \vee (Q \vee R)$
  \\
  \emph{unitality} & $P \wedge \top = P$ & $P \vee \bot = P$ \\
  \emph{annihilation} & $P \wedge \bot = \bot$ & $P \vee \top = \top$
\end{tabular} \\[0.75em]
%
We will often refer to these as ``algebraic laws'' or ``axioms''
though we can derive them (prove them) via natural deduction.

(Aside: you might like to think about other notions in mathematics
that have axioms of a similar form (or a subset of these axioms),
e.g. integer addition is commutative, associative, has $0$ as its unit, but is
not idempotent and does not have an annihilator).

\begin{restatable}{exc}{unitality}
Prove the algebraic property of unitality for conjunction, i.e.,
that $P \wedge \top = P$.
\end{restatable}

\noindent
``Distributivity'' and ``absorption'' laws give us a relationship
between $\wedge$ and $\vee$:
%
\begin{align*}
\textit{distributivity} \qquad & (P \vee Q) \wedge R \, = \, (P \wedge R) \vee (Q \wedge R) \\
 & (P \wedge Q) \vee R \, = \, (P \vee R) \wedge (Q
                                               \vee R) \\
  \textit{absorption} \qquad & (P \wedge Q) \vee P \, = \, P \\
   & (P \vee Q) \wedge P \, = \, P
\end{align*}
%
De Morgan's two laws give us a useful interaction between negation and
conjunction and disjunction:
%
\begin{align*}
\textit{De Morgan's} \qquad & \neg (P \wedge Q) \, = \, \neg P \vee \neg Q \\
& \neg (P \vee Q) \, = \, \neg P \wedge \neg Q
\end{align*}
%
``Complementation'' laws give us interaction between a formula and its
negation:
%
\begin{align*}
  \textit{complementation} \qquad  & P \wedge \neg P \, = \, \bot \\
  & P \vee \neg P \, = \, \top
\end{align*}
%
Note that the second complementation law here only hold in classical
(non-constructive) logic (which is what we primarily study here) where we
allow the law of excluded middle, derived from double-negation
elimination. Relatedly, the following law, known as ``involution''
only holds when we have double negation elimination in our logic:
%
\begin{align*}
\textit{involution} \qquad & \neg\neg P \, = \, P
\end{align*}
%
This notion of equivalence, given by the operator $=$, is symmetric,
transitive, reflexive, and a
\emph{congruence} with respect to all the operators of
logic. This means that if we have an equivalence between two formula
we can get an equivalence between larger formulas by
``plugging'' the first equivalence into a template for a formula.
For example, we get the following congruence property for conjunction:
%
\begin{equation*}
P = Q \;\; \Rightarrow \;\; P \wedge R = Q \wedge R
\end{equation*}
%
We can think of this as plugging the equivalence $P = Q$ into a
formula template $- \wedge R$. A similar congruence property holds for
the template $R \wedge -$ and for all the other operators, e.g.,
disjunction (with templates $R \vee -$ and $- \vee R$), and
implication, and negation.

The congruence property of $=$ is useful because it means we can use
equivalences to ``rewrite'' parts of a propositional formula.
We can then given \emph{equational proofs} that
some formula $P$ is equivalent to another formula $Q$
by applying  algebraic laws one at a time, possibly
to subparts of formulas.

\begin{example}
  Prove that $P \vee (Q \wedge \neg P) = Q \vee P$.

  We can proceed in the following steps where the subformula
  to which I have applied an equality is underlined and the rule
  which I am applying is written on the right:
  %
  \begin{align*}
  & \underline{P \vee (Q \wedge \neg P)} & \{\textit{distributity}\} \\
= \; & (P \vee Q) \wedge \underline{(P \vee \neg P)}  & \{\textit{complementation}\}  \\
= \; & \underline{(P \vee Q) \wedge \top} &  \{\textit{unitality of conjunction}\} \\
= \; & \underline{P \vee Q} & \{\textit{commutativity of disjunction}\} \\
= \; & Q \vee P & \qquad \qed
\end{align*}
\end{example}

\begin{restatable}{exc}{equational}
Prove that $P \wedge \neg(P \wedge Q) = P \wedge \neg Q$.
\end{restatable}

\section{Exercises}

  This section collects together the exercises given so far. They may
  not all be covered in lectures, so they provide useful
  additional examples to practise on.

  \assoc*
  \andReproof*
  \implProperty*
  \kcombinator*
  \biimplRules*
  \orassoc*
  \disproveEx*
  \lemp*
  \unitality*
  \equational*

\section{Collected rules of natural deduction}

\vspace{-1em}
\setlength{\tabcolsep}{1.54em}
\renewcommand{\arraystretch}{1}
\begin{longtable}{r||c|c}
 & \textit{Introduction} & \textit{Elimination} \\[0.5em] \hline \hline
  $\wedge$
& \rule{0cm}{0.75cm} $\dfrac{P \qquad Q}
         {P \wedge Q} \; {\wedge_i}$
& $\dfrac{P \wedge Q}
        {P} \; {\wedge_{e1}}
  \qquad
      \dfrac{P \wedge Q}
  {Q} \; {\wedge_{e2}}$ \\[-0.4em] & & \\ \hline
  %%%%%%%%%%%
  $\vee$
& $\begin{array}{c}\dfrac{P}
  {P \vee Q} \; {\vee_{i1}}
  \qquad
    \dfrac{Q}
  {P \vee Q} \; {\vee_{i2}}\\[3.25em]\end{array}$
& \rule{0cm}{2.15cm} $\setlength{\arraycolsep}{0em}
\dfrac{\begin{array}{c} \\ \\[0.7em] P \vee Q\end{array} \quad
\fbox{$\begin{array}{c} P \\ \vdots \\ R\end{array}$}
\quad
\fbox{$\begin{array}{c} Q \\ \vdots \\ R\end{array}$}}{R}
\;
{\vee_e}$ \\[-0.4em] & & \\ \hline
  %%%%%%%%%%%
  $\rightarrow$
 & \rule{0cm}{2.15cm} $\setlength{\arraycolsep}{0em}\dfrac{\fbox{$\begin{array}{c} P \\ \vdots \\ Q\end{array}$}}
      {P \rightarrow Q} {\rightarrow_i}$
 & $\begin{array}{c}\dfrac{P \rightarrow Q \qquad P}{Q}
      {\rightarrow_e} \\[3em]\end{array}$
  \\[-0.4em] & & \\ \hline
  $\neg$
 & \rule{0cm}{2.15cm}
   $\setlength{\arraycolsep}{0em}\dfrac{\fbox{$\begin{array}{c} P \\ \vdots \\ \bot\end{array}$}}
  {\neg P} \; {\neg_i}$
 & $\begin{array}{c}\dfrac{P \qquad \neg P}{\bot} {\neg_e}
      \\[3em] \end{array}$ \\[-0.4em] & & \\ \hline
  $\top$
& \rule{0cm}{0.75cm}$\dfrac{\qquad}{\top} \; {\top_i}$ \\[-0.4em] & &
  \\ \hline
  $\bot$
 & & \rule{0cm}{0.75cm}$\dfrac{\bot}{P} \; {\bot_e}$ \\[-0.4em] & & \\
  \hline
   $\neg\neg$
 & \rule{0cm}{0.75cm}{\textcolor{gray}{(derivable: $\dfrac{P}{\neg \neg P} \;\; {\neg\neg_i}$)}}  & \rule{0cm}{0.75cm}$\dfrac{\neg \neg P}{P} \;\; {\neg\neg_e}$
\end{longtable}

\vspace{0em}

\noindent
These are all the rules we have and need for propositional proofs. You
should aim to know all of the above rules by the end of the
course/exam.

We derived other useful inference rules from these rules, like modus
tollens, proof-by-contradiction, and law-of-excluded-middle. They
are useful to know but the table above gives the essential rules
for propositional proofs.



\section*{Appendix: Sequent-style natural deduction {\small{(not examined)}}}
\label{app:sequent}

Recall from Section~\ref{sec:entailment} that
a sequent is a compact representation of a formula $P$ along with
any assumptions used to deduce it, written in the form:
$P_1, \ldots, P_n \vdash P$.
The turnstile symbol $\vdash$ is read as \emph{entails} and the premises to the left are
called the \emph{context} of assumed formulas (or \emph{assumptions}). The right-hand side is
the \emph{conclusion}.  For example, the
judgment $P, Q \vdash P \wedge Q$ captures the idea of conjunction introduction.

An alternate formulation of natural deduction gives the usual
introduction and elimination rules in sequent form, making explicit
the assumption context of the formula. This sequent-style of natural
deduction is not assessed in CO519, but is included here for
completeness and to help with any wider reading.

A key rule that was implicit in the previous formulation of natural
deduction is the use of an assumption as a formula. This is usually
called the \emph{axiom} rule:
%
\begin{align*}
\dfrac{\qquad}{\Gamma, P \vdash P} \; (\textit{axiom})
\end{align*}
%
This says that given some context with an assumption $P$ and any other
assumptions, represented by the Greek symbol $\Gamma$ (uppercase
gamma)\footnote{Gamma $\Gamma$ is the third letter of the Greek
  alphabet, corresponding to Latin $C$, hence $\Gamma$ for
  ``Context''. Logicians like Greek as it gives them lots more symbols
  to use to represent things tersely. These are conventions which take
  some getting used to.} then we can conclude $P$.  This is similar to
the idea of copying in Fitch-style proofs.

The order of assumptions on the left of $\vdash$ is not important.
The sequent style captures that
there may be other assumptions $\Gamma$ in scope.  A meta rule says
that we can add arbitrary redundant assumptions into our context
(called \emph{weakening}):
%
\begin{equation*}
  \dfrac{\Gamma \vdash A}
        {\Gamma, \Gamma' \vdash A} \; (\textit{weaken})
  \end{equation*}
  %
This is useful when we have two subproofs that we want to
make have the same set of assumptions (see below). The rest
of the rules are introduction and elimination rules.

\paragraph{Conjunction}

\begin{align*}
\dfrac{\Gamma \vdash P \quad\; \Gamma \vdash Q}{\Gamma \vdash P \wedge
  Q} {\wedge_i}
\quad\;
\dfrac{\Gamma \vdash P \wedge Q}{\Gamma \vdash P} {\wedge_{e1}}
\quad\;
\dfrac{\Gamma \vdash P \wedge Q}{\Gamma \vdash Q} {\wedge_{e2}}
\end{align*}
%
These rules are very similar to the previously shown natural
deduction rules, but they now carry a context of assumptions
$\Gamma$. If the context doesn't match between
two premises, weakening (above) can be applied so that they match.

\paragraph{Disjunction}

\begin{align*}
\dfrac{\Gamma \vdash P}
      {\Gamma \vdash P \vee Q}  {\vee_{i1}}
\quad\;
\dfrac{\Gamma \vdash Q}
      {\Gamma \vdash P \vee Q} {\vee_{i2}}
\quad\;
\dfrac{\Gamma \vdash P \vee Q
  \quad\; \Gamma, P \vdash R
  \quad\; \Gamma, Q \vdash R}
      {\Gamma \vdash R} {\vee_{e}}
\end{align*}
%
Disjunction elimination is less unruly than the previous
formulation, but has the same meaning. Note, we extend the
context of assumptions with $P$ and $Q$ in the last two premises.

\paragraph{Implication}

\begin{align*}
\dfrac{\Gamma \vdash A \rightarrow B \quad \Gamma \vdash A}
      {\Gamma \vdash B} {\rightarrow_e}
\quad\;
\dfrac{\Gamma, A \vdash B}{\Gamma \vdash A \rightarrow B}
 {\rightarrow_i}
\end{align*}
As an example, here is the proof of $P \wedge Q \rightarrow P \vee Q$
in this style:
%
\newcommand{\pAB}{\dfrac{}{P \wedge Q \vdash P \wedge Q} {\textit{axiom}}}
\begin{align*}
  \dfrac{
  \dfrac{\dfrac{\pAB}{P \wedge Q \vdash P} {\wedge_{e1}}}
  {P \wedge Q \vdash P \vee Q} {\vee_{i1}}}
  {\vdash (P \wedge Q) \rightarrow (P \vee Q)} {\rightarrow_{i}}
\end{align*}

\paragraph{Negation,  falsity, and truth}

\begin{align*}
  \dfrac{\Gamma, P \vdash \bot}{\Gamma \vdash \neg P} {\neg_i}
  \qquad
  \dfrac{\Gamma \vdash P \quad \Gamma \vdash \neg P}{\Gamma \vdash \bot} {\neg_e}
  \qquad
  \dfrac{\Gamma \vdash \bot}{\Gamma \vdash P} {\bot_e}
  \qquad
  \dfrac{}{\Gamma \vdash \top} {\top_i}
\end{align*}
