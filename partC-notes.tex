% \newcommand{\subst}[3]{#3[#1/#2]}
\newcommand{\subst}[3]{\textit{replace}(#2, #1, #3)}

\newcommand{\highlight}[1]{%
 \underline{\colorbox{yellow!50}{$\displaystyle#1$}}}

\section{Satisfiable formula}

This section covers \emph{satisfiability} of propositional
formula. Recall from Part A that a formula is \emph{satisfiable} if it
is true for a particular ``assignment'' of either true or false to
variables in the formula. For example, $a \wedge b$ is satisfiable
since we can make it true by setting $a$ to be true and $b$ to be
true. We say this is the \emph{satisfying assignment}, and will write
this as:
%
\begin{equation*}
\{a = \top, \; b = \top\}
\end{equation*}
%
A valid formula is trivially satisfiable since it is true no matter
how we assign truth or falsehood to the variables.

The general problem of finding a satisfying assignment for a formula
is known in computer science as the \emph{Boolean satisfiability
  problem} or simply SAT for short. There are various
algorithmic approaches to solving the SAT problem, \ie{}, for
calculating a satisfying assignment for a propositional formula.  SAT
is used in many areas of formal verification, such as AI, planning,
circuit design, and theorem provers, and is applied to problems with
thousands of variables.

One approach is to exhaustively list all possible assignments of
true/false to variables in a formula by constructing its truth
tables. Thus for a formula with $n$ variables we need to calculate
$2^n$ rows. This is completely infeasible for problems with hundreds
or even thousands of variables. Instead, we'll look at the \emph{DPLL
  algorithm} which utilises properties of propositional
logic to be more efficient for many problems.

We can use SAT solving algorithms to prove validity of a formula $P$
by applying SAT to $\neg P$. If the algorithm shows us that $\neg P$
is unsatisfiable, that implies $\neg P$ is false for any assignment of
its variables and thus $P$ is true for any assignment of its
variables, \ie{}, \emph{valid}. This is very useful for modelling
problems where we have a complex model of some system in logic, and a
complex specification, and we want to prove the validity of:
%%
\begin{equation}
\textit{model} \rightarrow \textit{specification}
\label{eq:model-property}
\end{equation}
(see Part B of the course on modelling in logic). Instead of proving
this valid using natural deduction, we can instead use an algorithm
to show unsatisfiability of
$\neg (\textit{model} \rightarrow \textit{specification})$. If however
we find a satisfying assignment, then we have a counterexample to the
original property, that is, a set of variable assignments which makes
equation~\eqref{eq:model-property} false and thus shows us
a configuration which leads to a bug in our system.

\section{The DPLL algorithm}

DPLL stands for \emph{Davis-Putnam-Logemann-Loveland} (who proposed
this algorithm)\footnote{Actually the history is more complicated as
  in the history of most science: in 1960 Davis and Putnam did some
  work on automatically showing validity of formulae following a
  different technique, and David, Logemann, and Loveland built on some
  of the ideas to create their SAT in algorithm 1962.}. We'll go over
the technique since its a nice algorithm and makes some clever use of
the properties of propositional logic to simplify the exploration of
the state space. Despite the fact it is more than 50 years old, DPLL
still forms the basis of many SAT solvers, though there has been
some progress since.

\subsection{CNF}

The input to the DPLL algorithm is a formula in \emph{conjunctive
  normal form} (CNF for short).  A formula in Conjunction Normal
Form is a conjunction of disjunctions of literals. A literal is
either a variable or the negation of a variable. Thus, CNF formula are
of the form:
%
\begin{equation*}
  (\textit{x} \vee \neg \textit{y} \vee ...) \wedge
  (\textit{z} \vee \ldots) \wedge \ldots \wedge
  (\neg \textit{w} \vee \textit{x} \vee \textit{y})
\end{equation*}
%
We refer to a disjunction of literals as a \emph{clause}.
For example, the following highlights the middle clause:
%
\begin{equation}
(x \vee \neg w) \wedge {\highlight{(y \vee z)}} \wedge (\neg z \vee \neg x)
\end{equation}
%
A CNF formula consists of a set of clauses all of which have to be
true, since they are combined using conjunction. Within each clause,
just one of literals has to be true since they are combined via
disjunction.

Any formula can be converted into CNF by applying algebraic properties
of logic (see Part A, Section~\ref{sec:algebraic}) to rewrite a formula. Informally, this can be done by
doing the following:
%
\begin{itemize}[leftmargin=1em]
\item Replace implication $\rightarrow$ with disjunction $\vee$
and negation $\neg$, via: $$P \rightarrow Q = \neg P \vee Q$$
\item Use De Morgan's law to push negation inside of
disjunction and conjunction:
\begin{align*}
\neg (P \vee Q) & = \neg P \wedge \neg Q \\
\neg (P \wedge Q) & = \neg P \vee \neg Q
\end{align*}
\item Push disjunction inside and pull conjunction out (using
  distributive laws):
%
\begin{align*}
(P \wedge Q) \vee R & = (P \vee R) \wedge (Q \vee R) \\
P \vee (Q \wedge R) & = (P \vee Q) \wedge (P \vee R)
\end{align*}
\item Eliminate double negation $\neg\neg P = P$
%
\end{itemize}
%
By repeatedly applying these equations as rewrites (orienting the
equalities from left to the right), we end up with a formula
in CNF which is ready for DPLL.

\subsection{DPLL, step-by-step}

DPLL can be summarised in pseudo code as follows, which
has four main steps which I remember using the acronym \textbf{TUPS}:
%
\begin{align*}
\textsf{DPLL}(\top) = &\textit{satisfiable} \\
\textsf{DPLL}(\bot) = & \textit{unsatisfiable} \\
\textsf{DPLL}(P) = & \\
    1. & \; \textit{\textbf{T}autology elimination} \\
    2. & \; \textit{\textbf{U}nit propagation} \\
    3. & \; \textit{\textbf{P}ure literal elimination} \\
    4. & \; \textit{\textbf{S}plit on a variable: choose a variable $v$} \\
       &  \quad \textsf{DPLL}(P') \{v = \top\}  \\
       & \quad \textsf{DPLL}(P') \{v = \bot\}
\end{align*}
%
Note that this is a recursive algorithm which branches
into two executions at the last step.
The four steps gradually reduce the size of the input CNF formula and
create an assignment for its variables (setting variables to be true
or false). %We can think of the input to DPLL as being a list of
%formula which are all joined together by conjunction. If this list is
%empty then we are done and the formula is satisfied.

The first step is a simplification step using properties
of propositional logic and the structure of CNF
formula. The next two steps provide simplification and can give us
some satisfying assignments.  The last step splits the problem into
two by picking a variable and repeating the DPLL procedure with that
variable assigned to be true in one branch (written above as $\{v = \top\}$)
and repeating DPLL with that variable assigned to false ($\{v = \bot\}$).

\paragraph{Step 1. Tautology elimination}

Consider a formula in CNF where one clause (highlighted in yellow
bellow) contains both $x$ and $\neg x$, e.g., a formula of
the form:
%
\begin{equation*}
P \wedge (\highlight{\ldots x \vee \ldots \vee \neg x}) \wedge Q
\end{equation*}
%
Such a formula can be simplified by completely removing the
highlighted clause. This is because a disjunction of a formula with
its negation is a tautology (always true) by \emph{complementation}:
 $P \vee \neg P = \top$.
Furthermore, since disjunction of anything with truth is equivalent to truth:
$P \vee \top = \top$ and for conjunction $P \wedge \top = P$
(\emph{unitality} properties) we can
completely filter out clauses that have a tautology, \ie{}
%
\begin{equation*}
P \wedge (\highlight{\ldots x \vee \ldots \vee \neg x}) \wedge Q
\;\;\; \longrightarrow \;\;\;
P \wedge (\top) \wedge Q
\;\;\; \longrightarrow \;\;\;
P \wedge Q
\end{equation*}
%
Note that this doesn't tell us whether to assign true or false to
$x$ yet. So far we just know that this clause didn't depend on the
truth or falsehood of $x$ because it contained this tautology.

\paragraph{Step 2: Unit propagation}

In CNF terminology, a ``unit'' is a clause which contains just one
literal (a variable or negation of a variable), \eg{},
%
\begin{equation*}
P \wedge (\highlight{x}) \wedge Q
\quad
\textit{or}
\quad
P \wedge (\highlight{\neg x}) \wedge Q
\end{equation*}
%
where the ``unit'' clauses are highlighted.

If we see a formula in the left form, then we know that $x$ must be
true if we want the overall formula to be true, since we are taking
the conjunction of $x$ with all the other formula. If we have a
formula with the right form, then we know that $x$ must be false so
that $\neg x$ is true to make the overall formula true. Thus
in the left case we get the assignment $x = \top$ and in the right
case we get $x = \bot$. We then \emph{propagate} this assignment to
the rest of the clauses $P$ and $Q$, replacing any occurrences of $x$
with its assignment:
%
\begin{align*}
  \begin{array}{lrcll}
& P \wedge (\highlight{x}) \wedge Q
& \longrightarrow &
\subst{\top}{x}{P} \wedge \subst{\top}{x}{Q} & \qquad
\{x = \top\} \\
\textit{or} \qquad & P \wedge
(\highlight{\neg x}) \wedge Q
& \longrightarrow &
\subst{\bot}{x}{P} \wedge \subst{\bot}{x}{Q} & \qquad
\{x = \bot\}
\end{array}
\end{align*}
%
The assignment that is output by this step is written on the
right-hand side. We have also removed the unit clause since it has
been made true and is therefore redundant now. Thus, unit propagation
gives us both a simplification and an assignment when we have unit
clauses in our formula.

Here we are using a function on the syntax of formulas called
\emph{replace} where $\subst{\top}{x}{P}$ means replace/substitute any
occurrence of $x$ in $P$ with $\top$, and similarly $\subst{\bot}{x}{P}$
replaces any occurrence of $x$ in $P$ with $\bot$.
For example, $\subst{\top}{x}{\neg x \vee \neg y}$ would be
the formula $\neg \top \vee \neg y$.
This idea of
substituting one proposition for another will crop up again in Part D
when we look at first-order logic proofs.

\begin{example}
The following formula has a unit clause highlighted in yellow and underlined:
%
\begin{equation*}
(\neg x \vee \neg y) \wedge \highlight{x} \wedge (y \vee x)
\end{equation*}
%
Since it is ``positive'' (\ie{}, not negated) then unit propagation
emits the assignment $\{x = \top\}$ and then propagates this
assignment by replacement:
%%
\begin{equation*}
\subst{\top}{x}{(\neg x \vee \neg y)} \wedge \subst{\top}{x}{(y \vee x)}
\end{equation*}
%%
Applying the replacement function then gives us:
%
\begin{equation*}
(\neg \top \vee \neg y) \wedge (y \vee \top)
\end{equation*}
%
which simplifies via the following steps
%
\begin{align*}
 & (\neg \top \vee \neg y) \wedge (y \vee \top) \\
= \; & (\neg \top \vee \neg y) \wedge \top \\
= \; & (\neg \top \vee \neg y) \\
= \; & (\bot \vee \neg y) \\
= \; & \neg y
\end{align*}
%
In DPLL, we need only apply simplifications that involve disjunction
or conjunction with $\top$ or $\bot$. It is straightforward to build
this into an implementation so that these steps are implicit and
automatic: essentially the ``replacement'' can remove entire clauses when
we know we are making a literal true, or remove literals when they
are false.

Performing just one step of unit propagation has greatly simplified
our original formula from $(\neg x \vee \neg y) \wedge (x)
\wedge (y \vee x)$ to $\neg y$, along with giving the assignment
$\{x = \top\}$.  Since $\neg y$ is itself a unit, we apply unit
propagation again, yielding the assignment $\{y = \bot\}$ and
the formula $\top$. Thus we have reached a satisfying assignment
$\{x = \top, y = \bot\}$ just by applying unit propagation twice.
\end{example}

\begin{restatable}{exc}{unitFoo}
Convince yourself that $\{x = \top, y = \bot\}$ is a satisfying
assignment for $(\neg x \vee \neg y) \wedge (x)
\wedge (y \vee x)$ by substituting the variables
for their assignment and simplifying.
\end{restatable}

\begin{restatable}{exc}{unitFooTwo}
Perform unit propagation on the formula $(y \vee \neg x)
\wedge (\neg y)$.
\end{restatable}

\paragraph{Step 3: Pure literal elimination}

In DPLL terminology, a pure literal is a literal whose
negation does not appear anywhere else in the entire formula, \eg{}
%
\begin{align*}
P \wedge (\ldots \vee \highlight{x} \vee \ldots) \wedge Q
\end{align*}
%
where its dual $\neg x$ does not appear in $P$ nor is it in $Q$.
For example, $x$ is a pure literal in this formula:
%
\begin{equation*}
(x \vee y \vee \neg z) \wedge (\neg y \vee z \vee x)
\end{equation*}
%
The literal $x$ can appear multiple times, what makes it ``pure'' is
that its negation never appears anywhere. This means we can assign $x$
to be true. We could assign $x$ to be false, but it might be the wrong
decision later, for example if the other literals in the clause turn
out to false as well. It turns out that the most useful approach is to
assign $x$ to true by following the principle of progressing towards
a true formula as quickly as possible.

A pure literal might also be negative, for example:
\begin{align*}
P \wedge (\ldots \vee \highlight{\neg x} \vee \ldots) \wedge Q
\end{align*}
where $x$ does not appear in $P$ and $Q$. In this case, we can assign
$x$ to be false, and propagate this assignment into $P$ and $Q$.

Pure literal elimination therefore has the two dual rules:
%
\begin{align*}
\begin{array}{rlcll}
  & P \wedge (\ldots \vee \highlight{x} \vee \ldots) \wedge Q
  & \longrightarrow & \subst{\top}{x}{P} \wedge \subst{\top}{x}{Q}
& \quad \{x = \top\}
  \\[0.25em]
\textit{or} & P \wedge (\ldots \vee \highlight{\neg x} \vee \ldots) \wedge Q
  & \longrightarrow & \subst{\bot}{x}{P} \wedge \subst{\bot}{x}{Q}
& \quad \{x = \bot\}
\end{array}
\end{align*}
%
\begin{example}
The following formula has pure literal $x$:
%
\begin{equation*}
(\highlight{x} \vee y \vee \neg z) \wedge (\neg y \vee x)
\wedge (y \vee z)
\end{equation*}
%
Pure literal elimination then produces the assignment
$\{x = \top\}$, eliminates the first clause (since it is now true),
and rewrites the rest of the formula as follows:
%
\begin{align*}
  & \subst{\top}{x}{(\neg y \vee x) \wedge (y \vee z)} \\
= \; & (\neg y \vee \top) \wedge (y \vee z) \\
= \; & (y \vee z)
\end{align*}
%
We are left with a clause where both $y$ and $z$ are now pure
literals, so pure literal elimination can be applied to either.

Let's pick $y$ and assign it to $\{y = \top\}$. This gives us $\top$
as the resulting formula.  Thus, we have found that
$({x} \vee y \vee \neg z) \wedge (\neg y \vee x) \wedge (y
\vee z)$
is satisfiable with $\{x = \top, y = \top\}$, and it doesn't matter
whether $z$ is true or false.
\end{example}

\begin{restatable}{exc}{pureLiteral}
Apply pure literal elimination to the formula:
%
\begin{equation*}
(\neg x \vee y \vee \neg z) \wedge (\neg y \vee z) \wedge (\neg x \vee z)
\end{equation*}
%
\end{restatable}

\paragraph{Step 4: Split a variable}

The previous steps have applied the rules of logic, and the shape
of CNF formula, to make simplifications and assignments. Once we've
done all that we can with those steps, the last step falls back to
a ``brute force'' approach. We pick a variable, and ``split it'', that is,
we assign it to be true and apply DPLL on the result and separately
assign it to false and apply DPLL on the result. This results
in us running two DPLL separate procedures from this
point forwards. That is, given a formula $P$, splitting recursively calls the
DPLL algorithm under two new assignments:
%
\begin{align*}
             & \textsf{DPLL}(\subst{\top}{x}{P}) \qquad\qquad\qquad \{x =
               \top\}  \\
\textit{and} \qquad &  \textsf{DPLL}(\subst{\bot}{x}{P}) \qquad\qquad\qquad \{x = \bot\}
\end{align*}
%
These recursive calls will be separate, producing
possibly different assignments in each.

\begin{example}
The following formula has no tautologies, units, or pure literals
(\ie{}, none of the first three steps of DPLL apply):
%
\begin{equation*}
(x \vee \neg y) \wedge (y \vee z) \wedge (\neg z \vee \neg x)
\end{equation*}
%
The splitting step chooses any variable in the formula, let's say
$z$, and splits the DPLL process.
Let's follow the branch with the assignment $\{z = \top\}$,
which is then propagated to the rest of the formula by replacement:
%%
\begin{align*}
\begin{array}{rll}
= & \subst{\top}{z}{(x \vee \neg y) \wedge (y \vee z) \wedge (\neg z \vee \neg x)}
& \qquad \{z = \top\} \\
= & (x \vee \neg y) \wedge (y \vee \top) \wedge (\bot \vee \neg x) \\
= & (x \vee \neg y) \wedge (\neg x) \\
\end{array}
\end{align*}
%%
The other branched DPLL process has assignment $\{z = \bot\}$ which
produces:
%
\begin{align*}
\begin{array}{rll}
= & \subst{\bot}{z}{(x \vee \neg y) \wedge (y \vee z) \wedge (\neg z \vee \neg x)}
& \qquad \{z = \bot\} \\
= & (x \vee \neg y) \wedge (y \vee \bot) \wedge (\top \vee \neg x) \\
= & (x \vee \neg y) \wedge (y)
\end{array}
%
\end{align*}
We then return to step 1 for both branches, and continue with two
separate instances of the DPLL procedure on the above two formulas.
%
\end{example}

\begin{restatable}{exc}{split}
Apply the splitting step to the following formula on variable $y$:
%
\begin{equation*}
(x \vee y \vee \neg z) \wedge (x \vee \neg y \vee z) \wedge (y \vee z)
\end{equation*}
%
Write down the two resulting formula after replacement and doing trivial
simplifications.
\end{restatable}

The following gives an example putting all the steps together.

\begin{example}
  Consider the following formula: $(\neg x \rightarrow y) \wedge x$.
  Is it satisfiable? Before we apply DPLL we first have to
convert it to CNF:
  %
   \begin{align*}
      \begin{array}{lll}
        & (\neg x \rightarrow y) \wedge x & \textit{\{$\rightarrow$ as
                                          $\vee$\}} \\
      = & (\neg \neg x \vee y) \wedge x & \textit{\{Double negation
                                          elimination}\} \\
      = & (x \vee y) \wedge x &
      \end{array}
    \end{align*}
    %
  Now we have the formula in CNF, we can apply DPLL. I'll write
  the steps in a table: \\[1em]
%
  \vspace{2em}
  \begin{tabular}{l|l|l|l}
    Step & Note & Resulting CNF formula & Satisfying assignments \\ \hline
      & start & $(x \vee y) \wedge x$ &  \\
    1 & Tautology elim & $(x \vee y) \wedge x$ &  \\
    2 & Unit propagation $x$ & $\top$ & $\{x = \top\}$
  \end{tabular} \\[-1em]
  %
  Thus the original formula is satisfiable with assignment $x = \top$
  (and $y$ can be anything). We reached this conclusion quickly,
  just by applying tautology elimination and unit propagation.
This tabulated form is handy for small examples; you might like
to use it for class exercises.
\end{example}

\begin{example}
The following is the proposition $\neg (\textit{model}
\rightarrow \textit{specification})$ for the traffic light
example shown in Part B, but in CNF:
%
\begin{align*}
& (\neg r \vee g') \wedge (\neg r \vee \neg r') \wedge (\neg g \vee \neg
  g') \wedge (\neg g \vee r') \\[-0.3em]
\wedge \; & (r \vee g) \wedge (\neg r \vee \neg g) \wedge \highlight{(\neg r' \vee r')} \\[-0.4em]
\wedge  \; & (\neg r' \vee g') \wedge (\neg g' \vee r') \wedge \highlight{(\neg g' \vee g)}
\end{align*}
%
We'll apply DPLL to it here. Due to the large size of the formula,
I'll use a more free-form style rather than the tabulated form
used above.

\textit{Step 1: Tautology elimination} - There are two immediate
tautologies in the above formula which are highlighted.
These are eliminated to give:
%
\begin{align*}
& (\neg r \vee g') \wedge (\neg r \vee \neg r') \wedge (\neg g \vee \neg
  g') \wedge (\neg g \vee r') \\
 \wedge \; & (r \vee g) \wedge (\neg r \vee \neg g) \\
\wedge \; & (\neg r' \vee g') \wedge (\neg g' \vee r')
\end{align*}
%
There are no units or pure literals so we move to step 4.

\textit{Step 4: Splitting a variable} - Choose $r$:
%
\begin{longtable}{c|c}
%%%%
$\{r = \top\}$  & $\{r = \bot\}$ \\[0.0em]
yielding: & yielding: \\[0.0em]
$\begin{array}{ll}
& (\neg \top \vee g') \wedge (\neg \top \vee \neg r')
  \wedge \\
& (\neg g \vee \neg
  g') \wedge (\neg g \vee r') \wedge \\
& (\top \vee g) \wedge (\neg \top \vee \neg g) \wedge \\
& (\neg r' \vee g') \wedge (\neg g' \vee r')
 \end{array}$ &
%%%
$\begin{array}{ll}
& (\neg \bot \vee g') \wedge (\neg \bot \vee \neg r')
  \wedge \\
& (\neg g \vee \neg
  g') \wedge (\neg g \vee r') \wedge \\
& (\bot \vee g) \wedge (\neg \bot \vee \neg g) \wedge \\
& (\neg r' \vee g') \wedge (\neg g' \vee r')
 \end{array}$ \\[0em]
%%%%%%%%
which simplifies to & which simplifies to \\[0em]
$\begin{array}{ll}
& {g'} \wedge (\neg r') \wedge (\neg g \vee \neg
  g') \wedge \\
& (\neg g \vee r') \wedge {(\neg g)} \wedge \\
& (\neg r' \vee g') \wedge (\neg g' \vee r')
 \end{array}$ &
%%%
$\begin{array}{ll}
& (\neg g \vee \neg g') \wedge (\neg g \vee r') \wedge \\
& {(g)} \wedge (\neg r' \vee g') \wedge (\neg g' \vee r')
 \end{array}$  \\ \hline \\[-1em]
  \textit{1. Unit propagation} on the unit $g'$ &
  \textit{1. Unit propagation} on the unit $g$ \\[0em]
$\begin{array}{ll}
& \highlight{g'} \wedge (\neg r') \wedge (\neg g \vee \neg
  g') \wedge \\
& (\neg g \vee r') \wedge {(\neg g)} \wedge \\
& (\neg r' \vee g') \wedge (\neg g' \vee r')
 \end{array}$ &
$\begin{array}{ll}
& (\neg g \vee \neg g') \wedge (\neg g \vee r') \wedge \\
& \highlight{(g)} \wedge (\neg r' \vee g') \wedge (\neg g' \vee r')
 \end{array}$ \\
yields assignment $\{g' = \top\}$ giving
& yields assignment $\{g = \top\}$ giving \\[0em]
  %%%%%%%
$\begin{array}{ll}
& {(\neg r')} \,\wedge\, (\neg g) \,\wedge\, \\
& (\neg g \vee r') \,\wedge\, \\
& {(\neg g)} \,\wedge\, (r')
\end{array}$ &
%%%
$\begin{array}{ll}
& {(\neg g')} \,\wedge\, (r') \,\wedge\, \\
& (\neg r' \vee g') \,\wedge\, (\neg g' \vee r')
 \end{array}$ \\[0em]
  \textit{1. Unit propagation:} on the unit $\neg r'$ &
  \textit{1. Unit propagation:} on the unit $\neg g'$ \\
$\begin{array}{ll}
& \highlight{(\neg r')} \,\wedge\, (\neg g) \,\wedge\, \\
& (\neg g \vee r') \,\wedge\, \\
& {(\neg g)} \,\wedge\, (r')
\end{array}$ &
%%%
$\begin{array}{ll}
& \highlight{(\neg g')} \,\wedge\, (r') \,\wedge\, \\
& (\neg r' \vee g') \,\wedge\, (\neg g' \vee r')
 \end{array}$ \\[1em]
yields assignment $\{r' = \bot \}$ giving &
yields assignment $\{g' = \bot\}$ giving \\[0em]
  %%%%%%%
$\begin{array}{ll}
& (\neg g) \wedge (\neg g) \wedge (\neg g) \wedge \bot
 \end{array}$ &
%%%
$\begin{array}{ll}
& (r') \wedge (\neg r')
 \end{array}$  \\
 & \\[-0.6em]
 & \textit{1. Unit propagation:} on the unit $r'$ \\
simplifies to & yields assignment $\{r' = \top\}$ giving \\
$\bot$ & $\bot$
\end{longtable}

\noindent
Both branches have ended with $\bot$ hence $\neg(\textit{model}
\rightarrow \textit{specification})$ is unsatisfiable and
therefore $\textit{model} \rightarrow \textit{specification}$
is valid! Hurray!
\end{example}

\begin{restatable}{exc}{full}
Apply the full DPLL procedure to the following formula (already
in CNF):
%
\begin{equation*}
(x \vee y \vee z)
\wedge (\neg x \vee y)
\wedge (\neg x \vee \neg y \vee \neg z)
\end{equation*}
\end{restatable}

\section{Exercises from this section}

%
\unitFoo*
\unitFooTwo*
\pureLiteral*
\split*
\full*
