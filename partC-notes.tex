\renewcommand{\highlight}[1]{%
  \colorbox{yellow!50}{$\displaystyle#1$}}
\newcommand{\highlightG}[1]{%
  \colorbox{green!30}{$\displaystyle#1$}}
\newcommand{\highlightR}[1]{%
  \colorbox{red!20}{$\displaystyle#1$}}

\newcommand{\rel}[1]{\mathsf{#1}}

First-order logic (also called \emph{predicate logic}) extends
propositional logic with \emph{quantification}: existential quantification
$\exists$ (``\emph{there exists}'') and universal quantification
$\forall$ (``\emph{for all}''). A quantification $\forall x$
binds a variable $x$ which range over the elements of some underlying
\emph{universe} which is external to the logic, e.g., quantifying
over all people or objects.  First-order logic also
allows the use of relations, predicates (unary relations, also called
\emph{classifiers}) and functions, operating
over elements of the universe, which can
be defined externally and are
domain-specific for whatever purpose the logic is being used.

Consider the following sentence:
%
\begin{equation*}
  \textit{Not all birds can fly}
\end{equation*}
%
We can capture this in first-order logic using quantification and
unary predicates. Let our universe be ``animals''
over which we informally define two predicates:
%
\begin{align*}
  \rel{B}(x)\ & \stackrel{\text{def}}{=}\ \textit{$x$ is a bird} \\
  \rel{F}(x)\ & \stackrel{\text{def}}{=}\ \textit{$x$ can fly}
\end{align*}
%
As with propositional logic, we are studying the process and framework
of the logic rather than physical reality; it is up to us how we
assign the semantics of $\rel{B}$ and $\rel{F}$ above, but the
semantics of quantification and logical operators is fixed by the
definition of first-order logic.

We can then express the above sentence in first-order logic as:
%
\begin{equation}
  \neg (\forall x . \rel{B}(x) \rightarrow \rel{F}(x))
  \label{eq:nonflying1}
\end{equation}
%
We can read this exactly as \emph{it is not true that for all $x$, if
  $x$ is a bird then $x$ can fly}. Another way to write this is
that there are some birds which cannot fly:
%
\begin{equation}
  \exists x . \rel{B}(x) \wedge \neg \rel{F}(x)
    \label{eq:nonflying2}
\end{equation}
%
If we have a universe and semantics for $\rel{B}$ and $\rel{F}$ that includes, for
example, penguins, then both \eqref{eq:nonflying1} and
\eqref{eq:nonflying2} will be true.
We can prove that \eqref{eq:nonflying1} and
\eqref{eq:nonflying2} are equivalent in first-order logic via two
proofs, one for:
$$\neg (\forall x . \rel{B}(x) \rightarrow \rel{F}(x)) \; \vdash \;
\exists x . \rel{B}(x) \wedge \neg \rel{F}(x)$$
and one for:
$$
\exists x . \rel{B}(x) \wedge \neg \rel{F}(x) \; \vdash \;
\neg (\forall x . \rel{B}(x) \rightarrow \rel{F}(x))
$$
We will do this later once we have explained more about the
meta theory of the logic.

\section{Key concepts (meta theory) of first-order-logic}

\subsection{Names and binding}

In propositional logic, variables range over propositions, \ie{},
their ``type'' is a proposition. For example, $x \wedge y$ has two
propositional variables $x$ and $y$ which could be replaced with true
or false, or with any other formula. In predicate logic, universal and
existential quantifiers provide \emph{variable bindings} which
introduce variables ranging over objects in some fixed universe rather than
over propositions. For example, the formula $\forall x . P$ binds
a variable $x$ in the \emph{scope} of $P$. That is, $x$
is available within $P$, but not outside of it. A variable which does
not have a binding in scope is called \emph{free}.

For example, the formula below has free variables $x$ and $y$
and bound variables $u$ and $v$:
%
\begin{equation*}
\rel{P}(x) \, \vee  \, \forall u . \, (\, \rel{Q}(y) \, \wedge \,
\rel{R}(u) \, \rightarrow \exists v . \, (\, \rel{P}(v) \, \wedge \, \rel{Q}(x) \,))
\end{equation*}
%
The following repeats the formula and
highlights the binders in yellow, the bound variables
in green, and the free variables in red:
%
\begin{equation*}
\highlightR{\rel{P}(x)} \vee \highlight{\forall u} . (\highlightR{\rel{Q}(y)} \wedge \highlightG{\rel{R}(u)}
\rightarrow \highlight{\exists v} .
(\highlightG{\rel{P}(v)} \wedge \highlightR{\rel{Q}(x)}))
\end{equation*}
%
%
In the following formula, there are two syntactic occurrences of a
variable called $x$, but semantically these are different variables:
%
\begin{equation*}
\rel{Q}(x) \wedge (\forall x . \rel{P}(x))
\end{equation*}
%
The $x$ on the left (used with a predicate $\mathsf{Q}$) is free,
whilst the $x$ used with the predicate $\rel{P}$ is bound by the
universal quantifier. Thus, these are semantically two different
variables.

\paragraph{Alpha renaming}
The above formula is
semantically equivalent to the following formula obtained by
consistently renaming bound variables:
%
\begin{equation*}
\rel{Q}(x) \wedge (\forall y . \rel{P}(y))
\end{equation*}
%
Renaming variables is a meta-level operation we can apply to any
formula: we can rename a bound variable as long as we do not rename it
to clash with any other free or bound variable names, and as long as
we rename the variable consistently. This principle is more generally
known as $\alpha$-renaming (alpha renaming) and equality up-to
renaming (equality that accounts for renaming) is known as
$\alpha$-equality. For example, writing $\alpha$-equality as
$=_{\alpha}$ the following equality and inequality hold:
%
\begin{equation*}
\exists x . \rel{P}(x) \rightarrow \rel{P}(y)
  \;\;\; =_{\alpha} \;\;\;
\exists z . \rel{P}(z) \rightarrow \rel{P}(y)
 \;\;\; \neq_{\alpha} \;\;\; \exists y . \rel{P}(y) \rightarrow \rel{P}(y)
\end{equation*}
%
The middle formula can be obtained from the left by renaming $x$ to a
fresh variable $z$. However, if we rename $x$ to $y$ (on the right)
we conflate the bound variable with the previously free variable to the
right of the implication; we accidentally capture the
free occurence of $y$ via the binding. The right-hand formula has
a different meaning to the other two.

\subsection{Substitution}

Recall in Part C, we used the function $\textit{replace}$ in the DPLL
algorithm where $\textit{replace}(x, Q, P)$ rewrites formula $P$ such
that any occurrences of variable $x$ are replaced with formula
$Q$. This is more generally called \emph{substitution}.

From now on we will use a more compact syntax for substitution
written $$P[t/x]$$ which means: \emph{replace variable $x$
with the term $t$ in formula $P$} (akin to $\textit{replace}(x, t,
P)$). This term could be another variable or a
concrete element of our universe.

Note that in predicate logic we have to be careful
about free and bound variables. Thus, $P[t/x]$ means replace any
\emph{free} occurrences of $x$ in $P$ with object $t$. (One way to remember
this notation is to observe that the letters used in the general
form above are in alphabetical order: $P$ then $t$ then $x$ to
give $P[t/x]$ for replacing $x$ with $t$ in $P$).  This is a common
notation also used in the course textbook.

We must be careful to replace only the free
occurrences of variables, that is, those variables which are not in
the scope of a variable binding of the same name. For example, in the
following we have a free $x$ and a bound $x$, so substitution only
affects the free $x$:
%
\begin{equation*}
(\rel{P}(x) \wedge \forall x . \rel{P}(x))[t/x]
= \rel{P}(t) \wedge \forall x . \rel{P}(x)
\end{equation*}
%
In general, it is best practice to give each bound variable a different name
to all other free and bound names in a formula in order to avoid
confusion.

\subsection{The meaning of quantification}
\label{subsec:quantifier-meaning}

We can define the meaning of universal and existential quantification
in terms of the propositional logic connectives.

\paragraph{Universal quantification}

Universal quantification essentially generalises conjunction.  That
is, if the objects in the universe over which we are quantifying are
$a_0, a_1, \ldots, a_n \in \mathcal{U}$ then universal quantification
of $x$ over a formula $P$ is equivalent to taking the repeated
conjunction of $P$, substituting each object for $x$, \ie{}
%
\begin{equation}
\forall x . P = P[a_0/x] \wedge P[a_1/x] \wedge
\ldots \wedge P[a_{n}/x]
\label{eq:forall-meaning}
\end{equation}
%
Thus, $\forall x . P$ means that we want $P$ to be true for
all the objects in the universe being used.
Note that there may be an infinite number of such objects.

\paragraph{Existential quantification}

Whilst universal quantification generalises conjunction,
existential quantification generalises disjunction.
If existential quantification binds a variable ranging
over objects $a_0, a_1, \ldots, a_n \in \mathcal{U}$ then:
%
\begin{equation}
\exists x . P = P[a_0/x] \vee P[a_1/x] \vee
\ldots \vee P[a_{n}/x]
\label{eq:exists-meaning}
\end{equation}
%
Thus, existential quantification is equivalent to the repeated
disjunction of the formula $P$ with each object in the universe
replacing $x$.



\subsection{Defining models/universes}

First-order logic can be instantiated for particular concrete tasks by
defining a universe $\mathcal{U}$ (a set of elements) and any
relations, functions, and predicates over this universe.

For example, we could define
$\mathcal{U} = \{\mathsf{cat}, \mathsf{dog}, \mathsf{ant},
\mathsf{chair}\}$ meaning that when we write quantified formulas like
$\forall x . P$ (for some formula $P$) then $x$ refers to any of the things in
$\mathcal{U}$ (i.e., $x \in \mathcal{U}$). We could then concretely
define some functions and predicates. For example, let's define a function
$\mathsf{legs}$ which maps from $\mathcal{U}$ to $\mathbb{N}$ (i.e.,
$\mathsf{legs} : \mathcal{U} \rightarrow \mathbb{N}$) as:
%
\begin{align*}
  \mathsf{legs}(\mathsf{cat}) = 4 \qquad \mathsf{legs}(\mathsf{dog}) = 4 \qquad
  \mathsf{legs}(\mathsf{ant}) = 6 \qquad \mathsf{legs}(\mathsf{chair}) = 4
\end{align*}
%
We can define predicates by listing all their true instances. For
example, $\mathsf{mammal}$ classifies some members
of $\mathcal{U}$, defined via a proposition that lists all the true
instances as a conjunction:
%
\begin{align*}
  \mathsf{mammal}(\mathsf{cat})\ \wedge\
  \mathsf{mammal}(\mathsf{dog})
\end{align*}
%
Let's consider some true formulas in this instantiation of first-order
logic:
%
\begin{align*}
  \begin{array}{ll}
  \vdash \forall x . \mathsf{mammal}(x) \rightarrow (\mathsf{legs}(x) = 4) &
\quad (\textit{every mammal has four legs}) \\
  \vdash \exists x . \mathsf{legs}(x) < 4 &
\quad (\textit{there is something with less than four legs})
  \end{array}
\end{align*}
%
We have also employed two relations over $\mathbb{N}$ here:\footnote{Strictly speaking, we are
  therefore using first-order logic where the universe contains
  our set $\{\mathsf{cat}, \mathsf{dog}, \mathsf{ant},
  \mathsf{chair}\}$ and $\mathbb{N}$, i.e.,
  $\mathcal{U} = \{\mathsf{cat}, \mathsf{dog}, \mathsf{ant},
\mathsf{chair}\} \cup \mathbb{N}$, and our function $\mathsf{legs}$ is
partial, defined only for a part of the universe.}
equality $=$ and less-than $<$.

A false proposition in this instantiation is:
%
\begin{align*}
  \not\vdash \forall u . (\mathsf{legs}(u) = 4) \rightarrow \mathsf{mammal}(u)
& \quad (\textit{everything with four legs is a mammal})
\end{align*}
%
This is false because $u$ could be $\mathsf{chair}$ (making the
premise of the implication true) but
$\mathsf{mammal}(\mathsf{chair})$ is false.

\section{Equational reasoning}
\label{sec:fo-eqn-reasoning}

As in propositional logic, there are equations (logical
equivalences) between particular first-order formulas. These can be used to
rearrange and simplify formulas. This sections shows
these equations, some of which are proved
in the next section via natural deduction.

Two key equations show that universal and existential
quantification are \emph{dual}:
%
\begin{align}
  \label{eq:quantifier-dual-first}
  \forall x . \neg P \equiv \neg \exists x . P
  \qquad
   \neg \forall x . P \equiv \exists x . \neg P
\end{align}
%
The order of repeated quantifications is irrelevant as shown by
the following equalities:
%
\begin{align}
  \forall x . \forall y. P \equiv \forall y . \forall x . P
  \qquad
  \exists x . \exists y . P \equiv \exists y . \exists x . P
\end{align}
%
Note however that these equalities are only for quantifications that
are of the same kind; $\forall x . \exists y . P$ is not equivalent to
$\exists y . \forall x . P$.
The rest of the equations capture interaction between quantification
and the other propositional connectives:
%
\begin{align}
  (\exists x . P) \vee (\exists x . Q) \equiv \exists x . (P \vee Q) \\
  (\forall x . P) \wedge (\forall x . Q) \equiv \forall x . (P \wedge
                                                                       Q) \\
  P \wedge (\exists x . Q) \equiv \exists x . (P \wedge Q) & \;\; \textit{when $x$ is  not free in $P$} \\
  P \vee (\forall x . Q) \equiv \forall x . (P \vee Q) & \;\;
 \textit{when $x$ is  not free in $P$}
\end{align}
\vspace{-2em}
\begin{example}
  We can now go back to the example from the introduction: that
  \emph{not all birds can fly}. We formulated this sentence as both
  $\neg (\forall x . \rel{B}(x) \rightarrow \rel{F}(x))$ and
  $\exists x . \rel{B}(x) \wedge \neg \rel{F}(x)$.
  We can show these two statements are equivalent by algebraic
  reasoning:
%
\begin{align*}
\begin{array}{rll}
    & \neg (\forall x . \rel{B}(x) \rightarrow \rel{F}(x)) & \{\textit{by \eqref{eq:quantifier-dual-first}}\} \\[0.4em]
  \equiv \;\; &  \exists x . \, \neg (\rel{B}(x) \rightarrow
                \rel{F}(x)) & \{\textit{$P \rightarrow Q \equiv \neg P
                              \vee Q$}\} \\[0.4em]
  \equiv \;\; & \exists x . \,\neg (\neg \rel{B}(x) \vee \rel{F}(x))
& \{\textit{De Morgan's}\} \\[0.4em]
  \equiv \;\; & \exists x . \,\neg \neg \rel{B}(x) \wedge \neg
                \rel{F}(x)
& \{\textit{Double negation elim.}\} \\[0.4em]
  \equiv \;\; & \exists x . \,\rel{B}(x) \wedge \neg \rel{F}(x) & \Box
\end{array}
\end{align*}
  Note that the actual universe and the definition of $\rel{B}$ and
  $\rel{F}$ is irrelevant to this proof; we did not rely on their
  definition but just the general properties of first-order logic.
\end{example}

\vspace{-1em}

\begin{restatable}{exc}{eqnProofF}
  Prove via equational reasoning that:
  %
  \vspace{-0.25em}
\begin{equation*}
  \forall x . \mathsf{mammal}(x) \rightarrow (\mathsf{legs}(x) = 4) \;
  \equiv \; \neg \exists x . \mathsf{mammal}(x) \wedge
  \mathsf{legs}(x)
  \neq 4
\end{equation*}
\end{restatable}
\vspace{-1em}
