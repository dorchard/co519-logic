
\part{Natural deduction for first-order logic}
\setcounter{section}{0}

First-order logic quantifiers have natural deduction elimination and introduction rules.

\section{Introducing and eliminating quantifiers}

\subsection{Universal quantification $\forall$ (\emph{for all})}

The notion that universals generalise conjunction
($\forall x . P = P[a_0/x] \wedge P[a_1/x] \wedge
\ldots \wedge P[a_{n}/x]$, \S\ref{subsec:quantifier-meaning})
helps to understand the elimination and introduction rules of universal
quantification.

\paragraph{Elimination} As a reminder, here
are the conjunction elimination rules:
%
\begin{equation*}
\dfrac{P \wedge Q}
        {P} \; {\wedge_{e1}}
\;\;
\dfrac{P \wedge Q}
  {Q} \; {\wedge_{e2}}
\end{equation*}
%
We could apply these two
rules repeatedly to get any formula out of the conjunction
$P[a_0/x] \wedge P[a_1/x] \wedge
\ldots \wedge P[a_{n}/x]$, giving us the following inference for
the $i^{th}$ part of the conjunction:
%
\begin{equation*}
  \dfrac{P[a_0/x] \wedge P[a_1/x] \wedge \ldots \wedge P[a_{n}/x]}
  {P[a_i/x]}
\end{equation*}
%
This is essentially how we do universal quantification elimination.
Universal quantification differs to conjunction in that each one of the things
being ``conjuncted'' together is of the same form: they are all $P$
with some variable $x$ replaced by an element of $\mathcal{U}$.
Therefore, eliminations
always give us some formula $P$ but with $x$ replaced by different objects. This is how
we define universal quantification elimination, with the rule:
%
\begin{equation*}
  \dfrac{\forall x . P}
  {P [t/x]} \; {\forall_e} \quad \textit{where $t$ is free for $x$ in $P$}
\end{equation*}
%
The intuition is that we eliminate $\forall$ by replacing the bound
variable $x$ with $t$ which is any element of $\mathcal{U}$ or a
 free variable which represents
\emph{any} of things ranged over by the $\forall$-quantified variable.
The side condition requires that if $t$ is a variable it is a free variable
 when it is substituted for $x$ in $P$-- that is the meaning of the
 phrase \emph{where $t$ is free for $x$ in $P$}.

 Why is this needed? Consider the following property of integers: for
every integer there exists a bigger integer:
%
\begin{equation*}
  \forall x . \exists y . (x < y)
\end{equation*}
In a natural deduction proof, we can eliminate the $\forall$ with
$t = x_0$ (some fresh variable) to get $\exists y . (x_0 < y)$. We
know nothing about $x_0$, it is just a variable representing any
object (integer) in the set of things ranged over by the $\forall$.
If we performed the elimination with the substitution where
$t = y$ (violating the side condition) then we would get
$\exists y . (y < y)$ which is no longer true: it says that there
exists a number which is greater than itself. The problem is that by
performing the substitution $(\exists y. (x < y))[y/x]$ we have
``captured'' the binding of $y$ with the $y$ we are substituting in,
which then changes the meaning of the formula. The
side condition prevents us substituting a variable which
gets inadvertently bound.

\begin{example}
  Prove $\forall x . (\rel{P}(x) \rightarrow Q) \vdash \rel{P}(t)
  \rightarrow Q$

  \begin{logicproof}{2}
    \forall x . (\rel{P}(x) \rightarrow Q) & premise \\
    \rel{P}(t) \rightarrow Q & $\forall_e$ 1 $\qquad \Box$
  \end{logicproof}
\end{example}
\vspace{-1.25em}
\paragraph{Introduction} Recall conjunction introduction:
%
\begin{equation*}
 \dfrac{P \qquad Q}{P \wedge Q} \; \wedge_i
\end{equation*}
%
Again, in the identity $\forall x . P = P[a_0/x] \wedge P[a_1/x] \wedge
\ldots \wedge P[a_{n}/x]$, every formula
being conjuncted is of the form $P[t/x]$. We could therefore define an
introduction for universal quantification if we had $P[t/x]$ for
all objects $t$ in the universe being used,
but this is usually not possible (the universe
could be infinite, \eg{}, integers).
Instead, it suffices to know that $P[x_0/x]$ is true
for an arbitrary free variable $x_0$ about which we know nothing else and
which is no longer used once we bind it with the universal
quantifier (it is then out of scope).

We denote the scope of such arbitrary variables using boxes like those
used for subproofs, but these boxes are not subproofs. The rule for
introduction of universal quantifiers is:
%
\begin{equation*}
\setlength{\arraycolsep}{0.2em}
\dfrac{\fbox{$\begin{array}{lc} \textcolor{freshVariableColor}{x_0} \;\; & \\[-1em] & \vdots \\ & P[x_0/x] \end{array}$}}
{\forall x . P}
\; {\forall_i}
\qquad
\textit{where $x_0$ is free for $x$ in $P$}
\end{equation*}
%
This says that there is a \emph{scope} (not a subproof) that has a
variable $x_0$ (marked in \freshVariableColorName{} in the top corner) which appears in a
proof concluding with $P[x_0/x]$. We can distinguish subproofs from
scope boxes by the fact that scope boxes declare their variable in the
top-left and they do not start with any assumptions.

If this variable $x_0$ above is only used in this scope, then we can leave
the scope and conclude $\forall x . P$. Thus, the proof inside the
scope is of $P$ but where $x$ is replaced by $x_0$. Note that there is
the same side condition as in $\forall_e$.

%Here's an example to make this idea more clear:
%
\begin{example}
Prove $\forall x . (\rel{P}(x) \rightarrow \rel{Q}(x)), \forall y
. \rel{P}(y) \vdash \forall z . \rel{Q}(z)$ is valid.

That is, for all $x$ such that $\rel{P}(x)$ implies $\rel{Q}(x)$ and
for all $y$ that $\rel{P}(y)$ then $\rel{Q}(z)$ holds for all $z$.
I have used different names for the bound variables for the sake
of clarity, but the formula would have the exact same meaning if each
bound variable was called $x$ here.

The following is a proof:
  \begin{logicproof}{2}
  \forall x . (\rel{P}(x) \rightarrow \rel{Q}(x)) & premise \\
  \forall y . \rel{P}(y)                          & premise \\
  \begin{subproof}
    \hspace{-1em}\textcolor{freshVariableColor}{x_0}
    \;\; \rel{P}(x_0) \rightarrow \rel{Q}(x_0) & $\forall e$ 1 \\
    \;\; \rel{P}(x_0)                          & $\forall e$ 2 \\
    \;\; \rel{Q}(x_0)                          & $\rightarrow_e$ 3, 4
  \end{subproof}
  \forall z . \rel{Q}(z)                       & $\forall i$ 3-5
  $\qquad \Box$
  \end{logicproof}
\end{example}
%
\noindent
We start with the two premises. The proof then proceeds
with a scope box on lines 3-5: \emph{but remember this is not
a subproof}, it merely serves to delimit the scope of
the fresh variable $x_0$ which starts on line 3 when
the $\forall$ on line 1 is eliminated.
Eliminating $\forall y . \rel{P}(y)$ on line 4 uses the same
$x_0$. Since the quantification is universal we can pick the same
object $x_0$ to eliminate line 4 that was given to us by the
elimination on line 3. Line 5 uses modus ponens to get $\rel{Q}(x_0)$
which gives us the formula on which we apply $\forall
i$.

Logically, we are applying universal quantification introduction
only on $\rel{Q}(x_0)$, but we state the lines in which $x_0$ is
in scope. Once we close this scope box, we can't do anything with
$x_0$. The proof has shown that if we pick an arbitrary object $x_0$
then we get $Q(x_0)$ and so we can conclude that for all
objects $z$ we have $\rel{Q}(z)$, expressed as $\forall z
. \rel{Q}(z)$.

\begin{restatable}{exc}{forallAndElim}
  Prove $\forall x . \rel{P}(x) \wedge \rel{Q}(x) \vdash \forall x
  . \rel{P}(x)$ is valid.
\end{restatable}
%
\begin{remark}
Recall that quantification binds more loosely than
any of the other logical operators, so $\forall x . \rel{P}(x)
\wedge \rel{Q}(x)$ is equivalent to $\forall x . (\rel{P}(x) \wedge
\rel{Q}(x))$ rather than $(\forall x . \rel{P} (x)) \rightarrow
\rel{Q}(x)$.
\end{remark}

\subsection{Existential quantification $\exists$ (\emph{there exists})}

Since existentials generalise disjunction:
$\exists x . P = P[a_0/x] \vee P[a_1/x] \vee
\ldots \vee P[a_{n}/x]$ (\S\ref{subsec:quantifier-meaning}),
we can use this to understand the
existential quantification natural deduction rules.

\paragraph{Introduction} Recall the introduction rules for disjunction:
%
\begin{align*}
    \dfrac{P}
  {P \vee Q} \; {\vee_{i1}}
  \qquad
    \dfrac{Q}
  {P \vee Q} \; {\vee_{i2}}
\end{align*}
%
Given a true formula $P$ then the disjunction of $P$ with any other
formula $Q$ is true. Since existential quantification is essentially
disjunction of formulas all of the same form $P[a_i/x]$, then
disjunction introduction generalises to the existential
introduction rule:
%
\begin{align*}
\dfrac{P[t/x]}{\exists x . P} \; \exists_i \;\;\;\;
\textit{$t$ is free for $x$ in $P$}
\end{align*}
%
That is, we can introduce $\exists x . P$ if we have
$P$ where some other variable $t$ replaces $x$. This
is a bottom-up reading of the rule, where the substitution
is applied to the premise. Similarly to $\forall_e$ there is a
side condition which states that $t$ must be free when substituted
for $x$ in $P$.

Note how this rule is essentially
the converse of $\forall$ elimination and thus we can
rather directly prove the following:
%\begin{align*}
%\dfrac{P}{\exists x . P[x/t]} \;\; \exists_i
%\end{align*}

\begin{example}
  Prove $\forall x. \rel{P}(x) \vdash \exists x . \rel{P}(x)$ is valid.

  \begin{logicproof}{2}
    \forall x . \rel{P}(x) & premise \\
    \rel{P}(t)             & $\forall_e$ 1 \\
    \exists x . \rel{P}(x) & $\exists_i$ 2 $\qquad \Box$
   \end{logicproof}
 \end{example}
 \vspace{-2em}
 \paragraph{Elimination} Recall disjunction elimination:
 %
\begin{align*}
\setlength{\arraycolsep}{0em}
\dfrac{\begin{array}{c} \\ \\[0.4em] P \vee Q\end{array} \quad
\fbox{$\begin{array}{c} P \\[-0.4em] \vdots \\[-0.25em] R\end{array}$}
\quad
\fbox{$\begin{array}{c} Q \\[-0.4em] \vdots \\[-0.25em] R\end{array}$}}{R}
\;
{\vee_e}
\end{align*}
%
We previously needed two subproofs for each case of the disjunct.
However, with an existential, we have a disjunction of many formula
of the same form $P[a_i/x]$. Rather than requiring a subproof for
each of them, concluding in some common formula $R$, we can capture
all such subproofs by a single subproof replacing $x$ with a new
free variable $x_0$ which is in scope only for the subproof.
This gives us the existential elimination rule:
%
\begin{equation*}
\setlength{\arraycolsep}{0.2em}
\dfrac{\begin{array}{l} \\[2em] \exists x . P\end{array} \quad
\fbox{$\begin{array}{lc} \textcolor{freshVariableColor}{x_0} \;\; & P[x_0/x]
         \\[-0.2em] &  \vdots \\ & R \end{array}$}}{R} \;\; {\exists_e}
   \qquad\; \textit{$x_0$ is free for $x$ in $P$}
\end{equation*}
%
The intuition is that the subproof represents a case for every
part of the long disjunction by using
an arbitrary variable $x_0$ that we know nothing about to represent
each atom in the universe. Similarly to disjunction elimination, if we
can then conclude $R$ (but without using $x_0$ since the scope box
ends here) then we can conclude $R$ overall. This box is
both a subproof \emph{and} a variable scope for $x_0$;  it introduces
a variable in scope but also has assumptions.

\begin{example}
Prove $\exists x . \rel{P}(x) \wedge \rel{Q}(x)
\vdash \exists y . \rel{P}(y)$ is valid.

\begin{logicproof}{2}
\exists x . \rel{P}(x) \wedge \rel{Q}(x)  & premise \\
\begin{subproof}
\hspace{-0.5em}\textcolor{freshVariableColor}{x_0}
\;\; \rel{P}(x_0) \wedge \rel{Q}(x_0) & assumption \\
\quad\, \rel{P}(x_0) & $\wedge_{e1}$ 2 \\
\quad\, \exists y  . \rel{P}(y) & $\exists_i$ 3
\end{subproof}
\exists y . \rel{P}(y) & $\exists_e$ 1, 2-4 $\qquad \Box$
\end{logicproof}
\end{example}

%\begin{restatable}{exc}{allExists}
%  Prove $\forall x . (\rel{P}(x) \rightarrow \rel{Q}(x)), \exists x
%  . \rel{P}(x) \vdash \exists x . \rel{Q}(x)$
%\end{restatable}

\begin{example}
  Prove $\exists x . \rel{P}(x),
  \forall x . \rel{P}(x) \rightarrow \rel{Q}(x) \vdash \exists x . \rel{Q}(x)$
  is valid.

  \begin{logicproof}{2}
    \exists x . \rel{P}(x) & premise \\
    \forall x . \rel{P}(x) \rightarrow \rel{Q}(x) & premise \\
    \begin{subproof}
      \hspace{-0.5em}\textcolor{freshVariableColor}{x_0} \;\; \rel{P}(x_0) &
      assumption \\
      \quad \rel{P}(x_0) \rightarrow \rel{Q}(x_0) &
      $\forall_e$ 2 \\
      \quad \rel{Q}(x_0) & $\rightarrow_e$ 4, 3 \\
      \quad \exists x . \rel{Q}(x) & $\exists_i$ 5
    \end{subproof}
\exists x . \rel{Q}(x) & $\exists_e$ 1, 3-6 $\qquad \Box$
\end{logicproof}
\end{example}
\vspace{-1em}
\begin{restatable}{exc}{existsOr}
Prove $\exists x . P \vee Q \vdash \exists x . P \vee \exists x . Q$
is valid.
\end{restatable}

\paragraph{Proving equational rules}

We can use our natural deduction rules to prove
 the equational rules of first-order logic (from
 Section~\ref{sec:fo-eqn-reasoning}).
 We highlight just two here.

\begin{example}
  Prove $\exists x . \neg P \vdash \neg \forall x . P$ is valid.

  \begin{logicproof}{2}
    \exists x . \neg P & assumption \\
    \begin{subproof}
      \forall x . P & assumption \\
      \begin{subproof}
        \hspace{-0.5em}{\textcolor{freshVariableColor}{x_0}}
        \;\; \neg P[x_0/x] & assumption \\
        \quad P[x_0/x] & $\forall_e$ 2 \\
        \quad \bot & $\neg_e$ 4, 3
      \end{subproof}
      \bot & $\exists_e$ 1, 3-5
    \end{subproof}
    \neg \forall x . P & $\neg_i$ 2-6 $\qquad \Box$
  \end{logicproof}
\end{example}

\begin{restatable}{exc}{duality}
  Prove $\neg \forall x . P \vdash \exists x . \neg P$ is valid.
\end{restatable}
%
\noindent
The above example and exercise together give us the following equality
on first-order formula:
%
\begin{align}
\label{eq:quantifier-dual}
 \neg \forall x . P \, = \, \exists x . \neg P
\end{align}
This is known as a \emph{duality} property. There are two duality
properties, and the second can be derived from the above using
double negation elimination:
%
\begin{restatable}{exc}{dualityTwo}
  Prove $\neg \exists x . P = \forall x . \neg P$ is valid using
  the equations for quantifier duality
  $\neg \forall x . P = \exists x . \neg P$ and double negation
  idempotence $P = \neg\neg P$.
\end{restatable}
%



\section{Collected rules for first-order logic}

\noindent
Natural deduction rules for first-order logic include
all those of propositional logic plus:

\begin{center}
\setlength{\tabcolsep}{1.54em}
\renewcommand{\arraystretch}{1}
\begin{tabular}{r||c|c}
 & \textit{Introduction} & \textit{Elimination} \\[0.5em] \hline \hline
  $\forall$
& \rule{0cm}{2.25cm} $\setlength{\arraycolsep}{0.2em}
\dfrac{\fbox{$\begin{array}{lc} \textcolor{freshVariableColor}{x_0} \;\; & \\ & \vdots \\ & P[x_0/x] \end{array}$}}
{\forall x . P}
\; {\forall_i}$
& $\begin{array}{l}\dfrac{\forall x . P}
  {P [t/x]} \; {\forall_e} \\[2.5em]\end{array}$ \\ & & \\[-0.5em] \hline
$\exists$
&
\rule{0cm}{0.75cm}
$\begin{array}{l}\dfrac{P[t/x]}{\exists x . P} \;\exists_i\\[2.5em]\end{array}$
&
\rule{0cm}{2.25cm}
$\setlength{\arraycolsep}{0.2em}
\dfrac{\begin{array}{l} \\[2em] \exists x . P\end{array} \quad
\fbox{$\begin{array}{lc} \textcolor{freshVariableColor}{x_0} \;\; & P[x_0/x]
 \\ &  \vdots \\ & R \end{array}$}}{R}\;\exists_e$
\end{tabular}
\end{center}
In each of the rules we have the requirement that the substitution
is \emph{capture avoiding}, \eg{}, \emph{$x_0$ is free for $x$ in $P$} and
\emph{$t$ is free for $x$ in $P$} in the relevant rules.

\section{Exercises}

\forallAndElim*
\existsOr*
\duality*
\dualityTwo*