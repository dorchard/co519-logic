\documentclass{article}

\usepackage{amsmath}
\usepackage{amsthm}
\usepackage{amssymb}
\usepackage[table,x11names]{xcolor}
\usepackage{logicproof}
\usepackage{url}
\usepackage{fancyhdr}
\usepackage{enumitem}

% For restatables
\usepackage{thmtools}
\usepackage{thm-restate}
\usepackage[hidelinks]{hyperref}
\usepackage{cleveref}

\theoremstyle{definition}
\newtheorem{definition}{Definition}
\newtheorem{lemma}{Lemma}
\newtheorem{example}{Example}
\newtheorem*{remark}{Remark}
\declaretheorem[name=Exercise,numberwithin=section]{exc}

\newcommand{\highlight}[1]{%
  \colorbox{yellow!50}{$\displaystyle#1$}}
\newcommand{\highlightG}[1]{%
  \colorbox{green!30}{$\displaystyle#1$}}
\newcommand{\highlightR}[1]{%
  \colorbox{red!20}{$\displaystyle#1$}}

\newcommand{\rel}[1]{\mathsf{#1}}

% Writing
\newcommand{\ie}{\emph{i.e.}}
\newcommand{\eg}{\emph{e.g.}}

\title{\vspace{-3em}CO519 - Theory of Computing - Logic \\
  {\large{Part D : First-order logic and its natural deduction
      proofs}}}
\author{Dominic Orchard \\
  {\small{School of Computing, University of Kent}}}

\date{Last updated on \today}

\begin{document}
\maketitle

\noindent
If you spot any errors or have suggested edits, the notes are written
in LaTeX and are available on GitHub at
\url{http://github.com/dorchard/co519-logic}. Please fork and submit a
pull request with any suggested changes.

\section{Natural deduction for first-order logic}

First-order logic (also known as \emph{predicate logic}) extends
propositional logic with quantification: existential quantification
$\exists$ (``\emph{there exists....}'') and universal quantification
$\forall$ (``\emph{for all...}'').  First-order logic also adds
relations, predicates (unary relations or \emph{classifiers}), and
functions. The relations, predicates, and functions are
domain-specific for whatever purpose the logic is being used and may
be defined externally. Usually some underlying ``universe'' is fixed
over which quantified variables range and on which predicates,
relations, and functions are defined.

Consider the following sentence:
%
\begin{equation*}
  \textit{Not all birds can fly}
\end{equation*}
%
We can capture this in first-order logic using quantification and
unary predicates. Let's define abstractly two predicates:
%
\begin{align*}
  \rel{B}(x) & : \textit{$x$ is a bird} \\
  \rel{F}(x) & : \textit{$x$ can fly}
\end{align*}
%
Our universe here might be ``animals'' or just general objects. As
with propositional logic, we are studying the process and framework of
logic rather than any connections of certain logical statements to
physical reality; it is up to us how we assign the semantics of
$\rel{B}$ and $\rel{F}$ above, but the semantics of quantification and
logical operators is fixed by the definition of first-order logic.

We can then express the above sentence in predicate logic as:
%
\begin{equation}
  \neg (\forall x . \rel{B}(x) \rightarrow \rel{F}(x))
  \label{eq:nonflying1}
\end{equation}
%
We can read this exactly as \emph{it is not true that for all $x$, if
  $x$ is a bird then $x$ can fly}. Another way to write this is 
that there are some birds which cannot fly:
%
\begin{equation}
  \exists x . \rel{B}(x) \wedge \neg \rel{F}(x)
    \label{eq:nonflying2}
\end{equation}
%
If we have a semantics for $\rel{B}$ and $\rel{F}$ that includes, for
example, penguins then both \eqref{eq:nonflying1} and
\eqref{eq:nonflying2} will be true.

We can prove that these two statements (\eqref{eq:nonflying1} and
\eqref{eq:nonflying2}) are equivalent in predicate logic via two
proofs, one for:
$$\neg (\forall x . \rel{B}(x) \rightarrow \rel{F}(x)) \vdash
\exists x . \rel{B}(x) \wedge \neg \rel{F}(x)$$
and one for:
$$
\exists x . \rel{B}(x) \wedge \neg \rel{F}(x) \vdash
\neg (\forall x . \rel{B}(x) \rightarrow \rel{F}(x))
$$
We will do this later once we have explained more about the
meta-theory of the logic, as well as the introduction and elimination
rules for quantification.

\subsection{Names and binding}

In propositional logic, variables range over propositions, \ie{},
their ``type'' is a proposition. For example, $x \wedge y$ has two
propositional variables $x$ and $y$ which could be replaced with true
or false, or with any other formula. In predicate logic, universal and
existential quantifiers provide \emph{variable bindings}, which
introduce variables ranging over objects in some universe rather than
over propositions. For example, the formula $\forall x . P$ binds
a variable $x$ in the \emph{scope} of $P$. That is, $x$
is available within $P$, but not outside of it. A variable which does
not have a binding in scope is called \emph{free}.

For example, the following formula has free variables $x$ and $y$
and bound variables $u$ and $v$:
%
\begin{equation*}
\rel{P}(x) \, \vee  \, \forall u . \, (\, \rel{Q}(y) \, \wedge \,
\rel{R}(u) \, \rightarrow \exists v . \, (\, \rel{P}(v) \, \wedge \, \rel{Q}(x) \,))
\end{equation*}
%
The following repeats the formula and 
highlights the binders in yellow, the bound variables
in green, and the free variables in red:
%
\begin{equation*}
\highlightR{\rel{P}(x)} \vee \highlight{\forall u} . (\highlightR{\rel{Q}(y)} \wedge \highlightG{\rel{R}(u)}
\rightarrow \highlight{\exists v} .
(\highlightG{\rel{P}(v)} \wedge \highlightR{\rel{Q}(x)}))
\end{equation*}
%
%
In the following formula, there are two syntactic occurrences of a
variable called $x$, but semantically these are different variables:
%
\begin{equation*}
\rel{Q}(x) \wedge (\forall x . \rel{P}(x))
\end{equation*}
%
The $x$ on the left (used with a predicate $\mathsf{Q}$) is free,
whilst the $x$ used with the predicate $\rel{P}$ is bound by the
universal quantifier. Thus, these are semantically two different
variables.

\paragraph{Alpha renaming}
The above formula is
semantically equivalent to the following formula obtained by
consistently renaming bound variables:
%
\begin{equation*}
\rel{Q}(x) \wedge (\forall y . \rel{P}(y))
\end{equation*}
%
Renaming variables is a meta-level operation we can apply to any
formula: we can rename a bound variable as long as we do not rename it
to clash with any other free or bound variable names, and as long as
we rename the variable consistently. This principle is more generally
known as $\alpha$-renaming (alpha renaming) and equality up-to
renaming (equality that accounts for renaming) is known as
$\alpha$-equality. For example, writing $\alpha$-equality as
$=_{\alpha}$ the following equality and inequality hold:
%
\begin{equation*}
\exists x . \rel{P}(x) \rightarrow \rel{P}(y)
  \;\;\; =_{\alpha} \;\;\; 
\exists z . \rel{P}(z) \rightarrow \rel{P}(y)
 \;\;\; \neq_{\alpha} \;\;\; \exists y . \rel{P}(y) \rightarrow \rel{P}(y)
\end{equation*}
%
The middle formula can be obtained from the left formula by
renaming $x$ to a fresh variable $z$. However, in the right-hand
formula, we have renamed $x$ to $y$ which conflates the bound
variable with the free variable $y$ on the right of the implication.
The formula on the right has a different meaning to the left two.

\subsection{Substitution}

Recall in Part C, we used the function $\textit{replace}$ in the DPLL
algorithm where $\textit{replace}(x, Q, P)$ rewrites formula $P$ such
that any occurrences of variable $x$ are replaced with formula
$Q$. This is more generally called \emph{substitution}.

From now on we will use a more compact syntax for substitution
written $$P[t/x]$$ which means: \emph{replace variable $x$
with the variable $t$ in formula $P$} (akin to $\textit{replace}(x, t,
P)$). We will only replace variables with other variables.
Note however that in predicate logic we have to careful
about free and bound variables. Thus, $P[t/x]$ means replace any
\emph{free} occurrences of $x$ in $P$ with object $t$. (One way to remember
this notation is to observe that the letters used in the general
form above are in alphabetical order: $P$ then $t$ then $x$ to
give $P[t/x]$ for replacing $x$ with $t$ in $P$).  This is a common
notation which is also used in the course textbook.

We must be careful to replace only the free
occurrences of variables, that is, those variables which are not in
the scope of a variable binding of the same name. For example, in the
following we have a free $x$ and a bound $x$, so substitution only
affects the free $x$ as such:
%
\begin{equation*}
(\rel{P}(x) \wedge \forall x . \rel{P}(x))[t/x] 
= \rel{P}(t) \wedge \forall x . \rel{P}(x)
\end{equation*}
%
In general, it is best practise to give bound variables a different name
to all other free and bound names in a formula, in order to avoid confusion.

\subsection{Natural deduction rules}

As with propositional logic, we will have elimination and introduction
rules for the new logical operators in first-order logic.

\subsubsection{Universal quantification (\emph{for all})}

Universal quantification essentially generalises conjunction.  That
is, if the objects in the universe over which we are quantifying are
$a_0, a_1, \ldots a_n \in \mathcal{U}$ then universal quantification
of $x$ over a formula $P$ is equivalent to taking the repeated
conjunction of $P$ substituting each object for $x$, \ie{}
%
\begin{equation*}
\forall x . P = P[a_0/x] \wedge P[a_1/x] \wedge
\ldots \wedge P[a_{n}/x]
\end{equation*}
%
(Note that there may be an infinite number of such objects). 
This perspective helps us to understand elimination and introduction
for universal quantification as a generalisation of elimination and
introduction for conjunction:
%
\begin{equation*}
\dfrac{P \wedge Q}
        {P} \; {\wedge_{e1}}
\;\;
\dfrac{P \wedge Q}
  {Q} \; {\wedge_{e2}}
\;\;
\dfrac{P \qquad Q}
         {P \wedge Q} \; {\wedge_i}
\end{equation*}
%
The elimination rule for universal quantification is then as follows:
%
\begin{equation*}
  \dfrac{\forall x . P}
  {P [t/x]} \; {\forall_e} \quad \textit{where $t$ is free when
    replacing $x$ in $P$}
\end{equation*}
%
The intuition is that we can eliminate a $\forall$ by replacing the
bound variable $x$ with an arbitrary variable $t$ which represents \emph{any} of
things ranged over by the $\forall$. This is similar to conjunction
elimination where we eliminate to either of the left or right-hand sides of
the conjunction.

The side condition requires that the variable $t$ is a free variable
 when it is substituted for $x$ in $P$. We'll see an example of
why this is needed. \footnote{The textbook (Huth and Ryan) uses the
  phrase ``\emph{$t$ is free for $x$ in $P$}'' which is a little
  confusing, but means the same as the above side condition for
  $\forall$ elimination: that the variable $t$ remains a free variable
  even once it replaces $x$ inside of $P$.}

Consider the following statement about integers, that for
every integer there exists a bigger integer:
%
\begin{equation*}
  \forall x . \exists y . (x < y)
\end{equation*}
In a natural deduction proof, we can eliminate the $\forall$ with
$t = x_0$ (some fresh variable) to get $\exists y . (x_0 < y)$. We
know nothing about $x_0$, it is just a variable representing any
object (integer) in the set of things ranged over by the $\forall$.
If we instead performed the elimination with the substitution where
$t = y$ (violating the side condition) then we would get
$\exists y . (y < y)$ which is no longer true: it says that there
exists a number which is greater than itself. The problem is that by
performing the substitution $(\exists y. (x < y))[y/x]$ we have
``captured'' the binding of $y$ with the $y$ we are substituting in,
which then changed the meaning of the formula. This is why we have the
side condition.

Next we will see introduction. This uses boxes like we used for
subproofs previously, but the boxes are no longer subproofs, but
instead mark out the scope of a variable. The rule is as follows:
%
\begin{equation*}
\setlength{\arraycolsep}{0.2em}
\dfrac{\fbox{$\begin{array}{lc} \textcolor{blue}{x_0} \;\; & \\ & \vdots \\ & P[x_0/x] \end{array}$}}
{\forall x . P}
\; {\forall_i}
\end{equation*}
%
This says that there is a \emph{scope} (not subproof) that has a
variable $x_0$ (marked in blue in the top corner) which appears in a 
proof concluding with $P[x_0/x]$. If this variable $x_0$ is only used
in this scope, then we can leave the scope and conclude $\forall x
. P$. Thus, the proof inside is of $P$ but where $x$ is replaced by
$x_0$. Here's an example to make this idea more clear:
%
\begin{example}
Prove $\forall x . (\rel{P}(x) \rightarrow \rel{Q}(x)), \forall y
. \rel{P}(y) \vdash \forall z . \rel{Q}(z)$ is valid.

That is, for all $x$ such that $\rel{P}(x)$ implies $\rel{Q}(x)$ and
for all $y$ that $\rel{P}(y)$ then $\rel{Q}(z)$ holds for all $z$. Note
how I have used different names for the bound variables for the sake
of clarity, but the formula would have the exact same meaning if each
bound variable was called $x$ here.

The proof proceeds:
  \begin{logicproof}{2}
  \forall x . (\rel{P}(x) \rightarrow \rel{Q}(x)) & premise \\
  \forall y . \rel{P}(y)                          & premise \\
  \begin{subproof}
    \hspace{-1em}\textcolor{blue}{x_0}
    \;\; \rel{P}(x_0) \rightarrow \rel{Q}(x_0) & $\forall e$ 1 \\
    \;\; \rel{P}(x_0)                          & $\forall e$ 2 \\
    \;\; \rel{Q}(x_0)                          & $\rightarrow_e$ 3, 4
  \end{subproof}
  \forall z . \rel{Q}(z)                       & $\forall i$ 3-5
  \end{logicproof}
\end{example}
%
\noindent
We start with the two premises. The proof then proceeds
with a scope box on lines 3-5: \emph{but remember this is not
a subproof}, it merely serves to delimit the scope of 
the fresh variable $x_0$ which starts on line 3 when
the ``for all'' on line 1 is eliminated.

When we eliminate $\forall y . \rel{P}(y)$ on line 4 we use the same
$x_0$. Since the quantification is universal we can pick the same
object $x_0$ to eliminate line 4 that we ``picked'' for the
elimination on line 3. Line 5 uses modus ponens to get $\rel{Q}(x_0)$
which gives us the formula on which we apply $\forall
i$.

Philosophically we are applying universal quantification introduction
only on $\rel{Q}(x_0)$, but we state the lines in which $x_0$ is
in scope. Once we close this scope box, we can't do anything with
$x_0$. The proof has shown that if we pick an arbitrary object $x_0$
then we get $Q(x_0)$ and so we can conclude that for all
objects $z$ we have $\rel{Q}(z)$ (expressed as $\forall z
. \rel{Q}(z)$).

Existential quantification is more restrictive.

\subsubsection{Existential quantification (\emph{there exists})}

Whilst universal quantification generalises conjunction,
existential quantification generalises disjunction. 
If existential quantification binds variables ranging
over the objects $a_0, a_1, \ldots a_n$ then:
%
\begin{equation}
\exists x . P = P[a_0/x] \vee P[a_1/x] \vee
\ldots \vee P[a_{n}/x]
\label{eq:exists-meaning}
\end{equation}
%
That is, existential quantification is equivalent to the disjunction
of $P$ for every object in the universe, replacing $x$.

Recall the introduction and elimination rules for disjunction:
%
\begin{align*}
\begin{array}{c}\dfrac{P}
  {P \vee Q} \; {\vee_{i1}}
  \qquad
    \dfrac{Q}
  {P \vee Q} \; {\vee_{i2}}\\[3.25em]\end{array}
\qquad \setlength{\arraycolsep}{0em}
\dfrac{\begin{array}{c} \\ \\[0.7em] P \vee Q\end{array} \quad
\fbox{$\begin{array}{c} P \\ \vdots \\ R\end{array}$}
\quad
\fbox{$\begin{array}{c} Q \\ \vdots \\ R\end{array}$}}{R}
\;
{\vee_e}
\end{align*}
%
Disjunction introduction generalises to the following existential
introduction rule:
%
\begin{align*}
\dfrac{P[t/x]}{\exists x . P} \; \exists_i \;\;
\textit{$t$ is free for $x$ in $P$}
\end{align*}
%
That is, we can introduce $\exists x . P$ if we have
$P$ where some other variable $t$ replaces $x$. This
is a bottom-up reading of the rule, where the substitution
is applied to the premise. Note how this rule is essentially
the converse of $\forall elimination$ and thus we can
rather directly prove the following:
%\begin{align*}
%\dfrac{P}{\exists x . P[x/t]} \;\; \exists_i
%\end{align*}

\begin{example}
  Prove $\forall x. \rel{P}(x) \vdash \exists x . \rel{P}(x)$ is valid.

  \begin{logicproof}{2}
    \forall x . \rel{P}(x) & premise \\
    \rel{P}(t)             & $\forall_e$ \\
    \exists x . \rel{P}(x) & $\exists_i$ $\qquad \Box$
   \end{logicproof}
 \end{example}
%
Existential elimination resembles disjunction elimination
but now combines the notion of a variable scope box with a subproof
box:
%
\begin{equation*}
\setlength{\arraycolsep}{0.2em}
\dfrac{\begin{array}{l} \\[2em] \exists x . P\end{array} \quad 
\fbox{$\begin{array}{lc} \textcolor{blue}{x_0} \;\; & P[x_0/x]
 \\ &  \vdots \\ & Q \end{array}$}}{Q}
\end{equation*}
%
The intuition is that the subproof here represents a case for every
single part of the disjunction in~\eqref{eq:exists-meaning} by using
an arbitrary variable $x_0$ that we know nothing about to represent
each atom in the universe. Similarly to disjunction elimination, if we
can then conclude $Q$ (but without using $x_0$ since the scope box
ends here) then we can conclude $Q$ overall.


\begin{restatable}{exc}{duality}
  $\neg \forall x . \rel{P} (x) \vdash \exists x . \neg \rel{P} (x)$
\end{restatable}

\section{Collected rules of natural deduction for
first-order logic}

\vspace{2em}

\setlength{\tabcolsep}{1.54em}
\renewcommand{\arraystretch}{1}
\begin{tabular}{r||c|c}
 & \textit{Introduction} & \textit{Elimination} \\[0.5em] \hline \hline
  $\forall$
& \rule{0cm}{2.25cm} $\setlength{\arraycolsep}{0.2em}
\dfrac{\fbox{$\begin{array}{lc} \textcolor{blue}{x_0} \;\; & \\ & \vdots \\ & P[x_0/x] \end{array}$}}
{\forall x . P}
\; {\forall_i}$
& $\dfrac{\forall x . P}
  {P [t/x]} \; {\forall e} \;$ {\small{$\begin{array}{c}\textit{where $t$ is
                                  free when} \\
\textit{replacing $x$ in $P$}\end{array}$}} \\[1.25em] \hline
$\exists$ 
&
\rule{0cm}{0.75cm}
$\dfrac{P[t/x]}{\exists x . P} \;\exists_i$ {\small{$\begin{array}{c}\textit{where $t$ is
                                  free when} \\
\textit{replacing $x$ in $P$}\end{array}$}}
&
\rule{0cm}{2.25cm}
$\setlength{\arraycolsep}{0.2em}
\dfrac{\begin{array}{l} \\[2em] \exists x . P\end{array} \quad 
\fbox{$\begin{array}{lc} \textcolor{blue}{x_0} \;\; & P[x_0/x]
 \\ &  \vdots \\ & Q \end{array}$}}{Q}$
\end{tabular}

\section{Exercises}

\duality*

\end{document}
